% filepath: /SmartRecommender-Project-Django-React/.docs/latex/olko/attachment.tex

\newpage
\section*{Załącznik A}
\addcontentsline{toc}{section}{Załącznik A: Szczegółowe scenariusze użycia}
\section*{Szczegółowe scenariusze użycia}

Poniżej przedstawiono szczegółowe opisy scenariuszy użycia systemu odpowiadających przypadkom użycia zaprezentowanym na diagramie w rozdziale 3.3. Scenariusze zostały opisane zgodnie z notacją: Aktor, Warunki początkowe, Przebieg scenariusza głównego, Warunki końcowe oraz Scenariusze alternatywne (jeśli występują).

\subsection*{Scenariusz 1: Przeglądanie i wyszukiwanie produktów}

\textbf{Aktor:} Gość (użytkownik niezalogowany)

\textbf{Warunki początkowe:} Użytkownik znajduje się na stronie głównej aplikacji

\textbf{Przebieg scenariusza głównego:}
\begin{enumerate}
\item Użytkownik otwiera stronę główną aplikacji w przeglądarce
\item System wyświetla katalog produktów z możliwością filtrowania i sortowania
\item Użytkownik wybiera kategorię ,,Laptopy'' z menu nawigacyjnego
\item System filtruje produkty i wyświetla tylko laptopy
\item Użytkownik wybiera opcję sortowania ,,Najlepszy sentyment''
\item System sortuje produkty według wartości \texttt{sentiment\_score} malejąco
\item Użytkownik przegląda listę produktów
\item Użytkownik klika na wybrany produkt
\item System wyświetla szczegółową stronę produktu z opisem, specyfikacjami, zdjęciami i opiniami
\end{enumerate}

\textbf{Warunki końcowe:} Użytkownik widzi szczegółowe informacje o wybranym produkcie

\textbf{Scenariusze alternatywne:}
\begin{itemize}
\item \textbf{1a.} Użytkownik wpisuje zapytanie w wyszukiwarkę zamiast wybierać kategorię:
  \begin{enumerate}
  \item[1.] Użytkownik wpisuje frazę ,,laptop gaming'' w pole wyszukiwania
  \item[2.] System wykonuje wyszukiwanie full-text w polach nazwa i opis produktu
  \item[3.] System wyświetla przefiltrowane wyniki z możliwością dalszego sortowania
  \item[4.] Powrót do kroku 7 scenariusza głównego
  \end{enumerate}
\end{itemize}

\subsection*{Scenariusz 2: Dodawanie produktów do koszyka z rekomendacjami}

\textbf{Aktor:} Gość (użytkownik niezalogowany)

\textbf{Warunki początkowe:} Użytkownik przegląda szczegółową stronę produktu (laptop)

\textbf{Przebieg scenariusza głównego:}
\begin{enumerate}
\item Użytkownik klika przycisk ,,Dodaj do koszyka''
\item System dodaje produkt do koszyka i wyświetla powiadomienie o sukcesie
\item Użytkownik klika ikonę koszyka w nawigacji
\item System wyświetla widok koszyka z dodanymi produktami
\item System automatycznie wywołuje algorytm Apriori dla produktów w koszyku
\item System wyświetla sekcję ,,Frequently Bought Together'' z trzema produktami rekomendowanymi przez algorytm Apriori (torba na laptopa, mysz bezprzewodowa, hub USB-C)
\item Każda rekomendacja zawiera metryki: lift, confidence oraz przycisk ,,Dodaj do koszyka''
\item Użytkownik przegląda rekomendacje
\item Użytkownik klika ,,Dodaj do koszyka'' przy torbie i myszy
\item System dodaje wybrane produkty do koszyka i aktualizuje sumę całkowitą
\end{enumerate}

\textbf{Warunki końcowe:} Koszyk zawiera laptop, torbę i mysz, wyświetlona jest zaktualizowana suma

\subsection*{Scenariusz 3: Rejestracja nowego użytkownika}

\textbf{Aktor:} Gość (użytkownik niezalogowany)

\textbf{Warunki początkowe:} Użytkownik znajduje się na stronie głównej

\textbf{Przebieg scenariusza głównego:}
\begin{enumerate}
\item Użytkownik klika przycisk ,,Zarejestruj się'' w nawigacji
\item System wyświetla formularz rejestracji
\item Użytkownik wypełnia pola: imię, nazwisko, adres e-mail, hasło, potwierdzenie hasła
\item Użytkownik klika przycisk ,,Zarejestruj''
\item System waliduje dane: sprawdza czy e-mail jest unikalny, czy hasła się zgadzają, czy hasło spełnia wymagania bezpieczeństwa
\item System tworzy nowe konto użytkownika w bazie danych
\item System wyświetla komunikat o pomyślnej rejestracji
\item System automatycznie loguje użytkownika i generuje token JWT
\item System przekierowuje użytkownika do panelu klienta
\end{enumerate}

\textbf{Warunki końcowe:} Użytkownik jest zalogowany i posiada aktywne konto w systemie

\textbf{Scenariusze alternatywne:}
\begin{itemize}
\item \textbf{5a.} Podany adres e-mail już istnieje w systemie:
  \begin{enumerate}
  \item[1.] System wyświetla komunikat błędu: ,,Użytkownik z tym adresem e-mail już istnieje''
  \item[2.] Powrót do kroku 3 scenariusza głównego
  \end{enumerate}
\item \textbf{5b.} Hasła nie są identyczne:
  \begin{enumerate}
  \item[1.] System wyświetla komunikat błędu: ,,Hasła muszą być identyczne''
  \item[2.] Powrót do kroku 3 scenariusza głównego
  \end{enumerate}
\end{itemize}

\subsection*{Scenariusz 4: Logowanie do systemu}

\textbf{Aktor:} Gość (użytkownik niezalogowany, posiadający konto)

\textbf{Warunki początkowe:} Użytkownik znajduje się na stronie głównej

\textbf{Przebieg scenariusza głównego:}
\begin{enumerate}
\item Użytkownik klika przycisk ,,Zaloguj się'' w nawigacji
\item System wyświetla formularz logowania
\item Użytkownik wpisuje adres e-mail i hasło
\item Użytkownik klika przycisk ,,Zaloguj''
\item System weryfikuje dane logowania w bazie danych
\item System generuje token JWT i zwraca go do klienta
\item Frontend zapisuje token w localStorage przeglądarki
\item System przekierowuje użytkownika do panelu klienta
\end{enumerate}

\textbf{Warunki końcowe:} Użytkownik jest zalogowany, posiada ważny token JWT

\textbf{Scenariusze alternatywne:}
\begin{itemize}
\item \textbf{5a.} Nieprawidłowy e-mail lub hasło:
  \begin{enumerate}
  \item[1.] System wyświetla komunikat błędu: ,,Nieprawidłowe dane logowania''
  \item[2.] Powrót do kroku 3 scenariusza głównego
  \end{enumerate}
\end{itemize}

\subsection*{Scenariusz 5: Składanie zamówienia}

\textbf{Aktor:} Klient (użytkownik zalogowany)

\textbf{Warunki początkowe:} Użytkownik jest zalogowany, koszyk zawiera co najmniej jeden produkt

\textbf{Przebieg scenariusza głównego:}
\begin{enumerate}
\item Użytkownik otwiera widok koszyka
\item Użytkownik przegląda produkty w koszyku i sumę całkowitą
\item Użytkownik klika przycisk ,,Przejdź do kasy''
\item System wyświetla formularz zamówienia z danymi dostawy
\item Użytkownik wypełnia/potwierdza dane: imię, nazwisko, adres dostawy, numer telefonu
\item Użytkownik wybiera metodę płatności (przelew, karta, przy odbiorze)
\item Użytkownik klika ,,Złóż zamówienie''
\item System waliduje dane zamówienia
\item System tworzy zamówienie w bazie danych ze statusem ,,oczekujące''
\item System tworzy rekordy OrderItem dla każdego produktu w koszyku
\item System czyści koszyk użytkownika
\item System wyświetla potwierdzenie zamówienia z numerem zamówienia
\item System przekierowuje użytkownika do panelu klienta z zakładką ,,My Orders''
\end{enumerate}

\textbf{Warunki końcowe:} Zamówienie zostało utworzone w systemie, koszyk jest pusty

\textbf{Scenariusze alternatywne:}
\begin{itemize}
\item \textbf{8a.} Brak wymaganych danych dostawy:
  \begin{enumerate}
  \item[1.] System wyświetla komunikat błędu z listą brakujących pól
  \item[2.] Powrót do kroku 5 scenariusza głównego
  \end{enumerate}
\end{itemize}

\subsection*{Scenariusz 6: Śledzenie statusu zamówienia}

\textbf{Aktor:} Klient (użytkownik zalogowany)

\textbf{Warunki początkowe:} Użytkownik jest zalogowany, posiada co najmniej jedno zamówienie

\textbf{Przebieg scenariusza głównego:}
\begin{enumerate}
\item Użytkownik otwiera panel klienta
\item Użytkownik klika zakładkę ,,My Orders''
\item System pobiera wszystkie zamówienia użytkownika z bazy danych
\item System wyświetla listę zamówień posortowaną od najnowszych
\item Dla każdego zamówienia wyświetlane są: numer zamówienia, data złożenia, suma całkowita, status (oczekujące/w realizacji/zakończone/anulowane)
\item Użytkownik klika na wybrane zamówienie
\item System wyświetla szczegóły zamówienia: listę produktów, ceny, adres dostawy, historię zmian statusu
\end{enumerate}

\textbf{Warunki końcowe:} Użytkownik widzi szczegółowe informacje o wybranym zamówieniu

\subsection*{Scenariusz 7: Wystawianie opinii o produkcie}

\textbf{Aktor:} Klient (użytkownik zalogowany)

\textbf{Warunki początkowe:} Użytkownik posiada zakończone zamówienie zawierające produkt

\textbf{Przebieg scenariusza głównego:}
\begin{enumerate}
\item Użytkownik otwiera panel klienta, zakładkę ,,My Orders''
\item Użytkownik znajduje zakończone zamówienie
\item Użytkownik klika przycisk ,,Add Review'' przy zakupionym produkcie
\item System wyświetla formularz opinii
\item Użytkownik wybiera ocenę gwiazdkową (1-5 gwiazdek)
\item Użytkownik wpisuje treść tekstową opinii
\item Użytkownik klika ,,Wyślij opinię''
\item System zapisuje opinię w bazie danych z powiązaniem: użytkownik, produkt, zamówienie
\item System automatycznie wywołuje algorytm analizy sentymentu dla nowej opinii
\item Algorytm tokenizuje tekst, wyszukuje słowa w słowniku pozytywnym/negatywnym
\item Algorytm oblicza sentiment\_score dla opinii zgodnie ze wzorem (2)
\item System przelicza zagregowany sentyment produktu zgodnie ze wzorem (3) (wieloźródłowa agregacja z 5 źródeł)
\item System aktualizuje pole \texttt{sentiment\_score} w rekordzie produktu
\item System wyświetla komunikat o pomyślnym dodaniu opinii
\item System odświeża widok produktu - zaktualizowany sentyment jest widoczny
\end{enumerate}

\textbf{Warunki końcowe:} Opinia została dodana, sentyment produktu zaktualizowany

\textbf{Scenariusze alternatywne:}
\begin{itemize}
\item \textbf{3a.} Użytkownik już wystawił opinię dla tego produktu w tym zamówieniu:
  \begin{enumerate}
  \item[1.] System wyświetla komunikat: ,,Już wystawiłeś opinię dla tego produktu''
  \item[2.] Scenariusz kończy się
  \end{enumerate}
\end{itemize}

\subsection*{Scenariusz 8: Zarządzanie kontem użytkownika}

\textbf{Aktor:} Klient (użytkownik zalogowany)

\textbf{Warunki początkowe:} Użytkownik jest zalogowany

\textbf{Przebieg scenariusza głównego:}
\begin{enumerate}
\item Użytkownik otwiera panel klienta
\item Użytkownik klika zakładkę ,,Profile''
\item System wyświetla formularz z danymi użytkownika: imię, nazwisko, e-mail
\item Użytkownik edytuje wybrane pola (np. zmienia nazwisko)
\item Użytkownik klika ,,Zapisz zmiany''
\item System waliduje dane (sprawdza czy e-mail jest unikalny, jeśli został zmieniony)
\item System aktualizuje rekord użytkownika w bazie danych
\item System wyświetla komunikat o pomyślnym zapisaniu zmian
\end{enumerate}

\textbf{Warunki końcowe:} Dane użytkownika zostały zaktualizowane

\subsection*{Scenariusz 9: Administrator zarządza produktami}

\textbf{Aktor:} Administrator

\textbf{Warunki początkowe:} Administrator jest zalogowany, posiada uprawnienia administratora

\textbf{Przebieg scenariusza głównego - dodawanie produktu:}
\begin{enumerate}
\item Administrator otwiera panel administracyjny
\item Administrator klika sekcję ,,Products''
\item System wyświetla listę wszystkich produktów
\item Administrator klika przycisk ,,Add New Product''
\item System wyświetla formularz dodawania produktu
\item Administrator wypełnia pola: nazwa, opis, cena, kategorie, specyfikacje techniczne, przesyła zdjęcie
\item Administrator klika ,,Zapisz produkt''
\item System waliduje dane (sprawdza czy wszystkie wymagane pola są wypełnione)
\item System tworzy nowy rekord produktu w bazie danych
\item System automatycznie wywołuje algorytm analizy sentymentu dla nowego produktu
\item Algorytm oblicza sentiment\_score na podstawie nazwy i opisu (brak jeszcze opinii)
\item System zapisuje obliczony sentyment w rekordzie produktu
\item System wyświetla komunikat o pomyślnym dodaniu produktu
\item System odświeża listę produktów
\end{enumerate}

\textbf{Warunki końcowe:} Nowy produkt został dodany do katalogu z obliczonym sentymentem

\textbf{Przebieg scenariusza alternatywnego - edycja produktu:}
\begin{enumerate}
\item Administrator w liście produktów klika ikonę ,,Edytuj'' przy wybranym produkcie
\item System wyświetla formularz edycji z wypełnionymi aktualnymi danymi
\item Administrator modyfikuje wybrane pola (np. zmienia cenę)
\item Administrator klika ,,Zapisz zmiany''
\item System aktualizuje rekord produktu w bazie danych
\item Jeśli zmieniono opis lub nazwę, system przelicza sentyment
\item System wyświetla komunikat o pomyślnym zapisaniu zmian
\end{enumerate}

\textbf{Przebieg scenariusza alternatywnego - usuwanie produktu:}
\begin{enumerate}
\item Administrator w liście produktów klika ikonę ,,Usuń'' przy wybranym produkcie
\item System wyświetla dialog potwierdzenia: ,,Czy na pewno chcesz usunąć ten produkt?''
\item Administrator klika ,,Potwierdź''
\item System sprawdza czy produkt nie jest używany w aktywnych zamówieniach
\item System oznacza produkt jako nieaktywny lub usuwa rekord z bazy danych
\item System wyświetla komunikat o pomyślnym usunięciu
\item System odświeża listę produktów
\end{enumerate}

\subsection*{Scenariusz 10: Administrator zarządza zamówieniami}

\textbf{Aktor:} Administrator

\textbf{Warunki początkowe:} Administrator jest zalogowany

\textbf{Przebieg scenariusza głównego:}
\begin{enumerate}
\item Administrator otwiera panel administracyjny
\item Administrator klika sekcję ,,Orders''
\item System wyświetla listę wszystkich zamówień w systemie
\item Dla każdego zamówienia wyświetlane są: numer, użytkownik, data, suma, aktualny status
\item Administrator klika na wybrane zamówienie ze statusem ,,oczekujące''
\item System wyświetla szczegóły zamówienia: produkty, dane dostawy, historia statusów
\item Administrator klika przycisk ,,Zmień status''
\item System wyświetla listę dostępnych statusów (w realizacji, zakończone, anulowane)
\item Administrator wybiera ,,w realizacji''
\item System aktualizuje status zamówienia w bazie danych
\item System zapisuje zdarzenie zmiany statusu z timestampem
\item System wyświetla komunikat o pomyślnej zmianie statusu
\item Użytkownik właściciel zamówienia widzi zaktualizowany status w swoim panelu
\end{enumerate}

\textbf{Warunki końcowe:} Status zamówienia został zmieniony, użytkownik ma dostęp do aktualnej informacji

\subsection*{Scenariusz 11: Administrator zarządza użytkownikami}

\textbf{Aktor:} Administrator

\textbf{Warunki początkowe:} Administrator jest zalogowany

\textbf{Przebieg scenariusza głównego:}
\begin{enumerate}
\item Administrator otwiera panel administracyjny
\item Administrator klika sekcję ,,Users''
\item System wyświetla listę wszystkich użytkowników
\item Dla każdego użytkownika wyświetlane są: ID, imię, nazwisko, e-mail, rola, data rejestracji
\item Administrator może filtrować użytkowników według roli (klient/administrator)
\item Administrator klika na wybranego użytkownika
\item System wyświetla szczegóły użytkownika: profil, historia zamówień, statystyki
\item Administrator może zmienić rolę użytkownika (nadać uprawnienia administratora) lub usunąć konto
\end{enumerate}

\textbf{Warunki końcowe:} Administrator ma pełny wgląd w dane użytkowników i może zarządzać uprawnieniami

\subsection*{Scenariusz 12: Administrator przegląda statystyki i dashboard}

\textbf{Aktor:} Administrator

\textbf{Warunki początkowe:} Administrator jest zalogowany

\textbf{Przebieg scenariusza głównego:}
\begin{enumerate}
\item Administrator otwiera panel administracyjny
\item System automatycznie wyświetla dashboard ze statystykami
\item Dashboard zawiera następujące kluczowe metryki:
  \begin{itemize}
  \item Całkowita liczba produktów w katalogu
  \item Całkowita liczba zarejestrowanych użytkowników
  \item Całkowita liczba zamówień (wszystkie statusy)
  \item Całkowity przychód (suma wartości zamówień zakończonych)
  \end{itemize}
\item System wyświetla wykresy:
  \begin{itemize}
  \item Wykres słupkowy: miesięczny obrót (ostatnie 12 miesięcy)
  \item Wykres kołowy: rozkład zamówień według kategorii produktów
  \item Wykres liniowy: liczba nowych użytkowników (ostatnie 30 dni)
  \end{itemize}
\item Administrator klika sekcję ,,Statistics'' dla bardziej szczegółowych analiz
\item System wyświetla zaawansowane raporty: najpopularniejsze produkty, najlepsi klienci, współczynnik konwersji
\end{enumerate}

\textbf{Warunki końcowe:} Administrator ma pełny wgląd w kluczowe wskaźniki biznesowe systemu

\subsection*{Scenariusz 13: Administrator debuguje algorytmy rekomendacji}

\textbf{Aktor:} Administrator

\textbf{Warunki początkowe:} Administrator jest zalogowany, algorytmy rekomendacji zostały przeliczone

\textbf{Przebieg scenariusza głównego - Collaborative Filtering:}
\begin{enumerate}
\item Administrator otwiera panel administracyjny
\item Administrator klika sekcję ,,Debug ML''
\item Administrator wybiera zakładkę ,,Collaborative Filtering''
\item System wyświetla panel debugowania Collaborative Filtering zawierający:
  \begin{itemize}
  \item Wzór matematyczny Adjusted Cosine Similarity
  \item Statystyki: liczba par produktów, średnie podobieństwo, czas ostatniego przeliczenia
  \item Tabelę z przykładowymi podobieństwami (product\_a, product\_b, similarity\_score)
  \end{itemize}
\item Administrator może wpisać ID produktu i zobaczyć top 10 najbardziej podobnych produktów
\item System pobiera dane z tabeli ProductSimilarity i wyświetla wyniki
\end{enumerate}

\textbf{Warunki końcowe:} Administrator zweryfikował poprawność działania algorytmu Collaborative Filtering

\textbf{Przebieg scenariusza alternatywnego - Apriori:}
\begin{enumerate}
\item Administrator wybiera zakładkę ,,Association Rules (Apriori)''
\item System wyświetla panel debugowania Apriori zawierający:
  \begin{itemize}
  \item Wzory matematyczne: Support, Confidence, Lift
  \item Parametry algorytmu: min\_support, min\_confidence
  \item Statystyki: liczba wygenerowanych reguł, liczba transakcji analizowanych
  \end{itemize}
\item System wyświetla tabelę z regułami asocjacyjnymi posortowaną według lift
\item Dla każdej reguły wyświetlane są: antecedent (produkt A), consequent (produkt B), support, confidence, lift
\item Administrator może filtrować reguły według minimalnego lift
\end{enumerate}

\textbf{Przebieg scenariusza alternatywnego - Sentiment Analysis:}
\begin{enumerate}
\item Administrator wybiera zakładkę ,,Sentiment Analysis''
\item System wyświetla panel debugowania analizy sentymentu zawierający:
  \begin{itemize}
  \item Wzór matematyczny polarności tekstu
  \item Wzór wieloźródłowej agregacji z wagami
  \item Statystyki: liczba opinii w systemie, rozkład opinii (pozytywne/neutralne/negatywne)
  \end{itemize}
\item System wyświetla histogram rozkładu sentiment\_score dla wszystkich produktów
\item Administrator może wpisać ID produktu i zobaczyć szczegółową dekompozycję sentymentu:
  \begin{itemize}
  \item Sentyment z opinii (40\%)
  \item Sentyment z opisu (25\%)
  \item Sentyment z nazwy (15\%)
  \item Sentyment ze specyfikacji (12\%)
  \item Sentyment z kategorii (8\%)
  \item Zagregowany wynik końcowy
  \end{itemize}
\end{enumerate}

\textbf{Warunki końcowe:} Administrator zweryfikował poprawność działania wszystkich trzech algorytmów rekomendacji

\subsection*{Scenariusz 14: Przeglądanie rekomendacji na stronie głównej}

\textbf{Aktor:} Klient (użytkownik zalogowany)

\textbf{Warunki początkowe:} Użytkownik jest zalogowany, posiada historię zakupów

\textbf{Przebieg scenariusza głównego:}
\begin{enumerate}
\item Użytkownik loguje się do systemu
\item System przekierowuje użytkownika do strony głównej
\item System wywołuje algorytm Collaborative Filtering dla zalogowanego użytkownika
\item Algorytm analizuje historię zakupów użytkownika
\item Algorytm pobiera produkty podobne do wcześniej zakupionych (na podstawie macierzy podobieństw)
\item System wyświetla sekcję ,,Rekomendowane dla Ciebie'' z 6 produktami
\item Użytkownik przegląda rekomendacje
\item Użytkownik klika na wybrany produkt
\item System wyświetla szczegółową stronę produktu
\end{enumerate}

\textbf{Warunki końcowe:} Użytkownik widzi spersonalizowane rekomendacje oparte na swojej historii zakupów

\subsection*{Scenariusz 15: Wyszukiwanie produktów z filtrem sentymentu}

\textbf{Aktor:} Gość lub Klient

\textbf{Warunki początkowe:} Użytkownik znajduje się na stronie głównej

\textbf{Przebieg scenariusza głównego:}
\begin{enumerate}
\item Użytkownik otwiera wyszukiwarkę z paska nawigacyjnego
\item Użytkownik wpisuje zapytanie ,,laptop gaming''
\item Użytkownik wybiera filtr ,,Tylko produkty z pozytywnym sentymentem''
\item Użytkownik klika ,,Szukaj''
\item System wykonuje zapytanie do bazy danych:
  \begin{itemize}
  \item Full-text search w polach nazwa i opis produktu
  \item Filtrowanie WHERE \texttt{sentiment\_score} $\geq$ 0.5
  \item Sortowanie ORDER BY \texttt{sentiment\_score} DESC
  \end{itemize}
\item System wyświetla przefiltrowane wyniki (tylko laptopy gaming z pozytywnym sentymentem)
\item Dla każdego produktu wyświetlane są: zdjęcie, nazwa, cena, sentiment\_score, liczba opinii
\item Użytkownik przegląda wyniki i może dalej filtrować według kategorii lub zakresu cenowego
\end{enumerate}

\textbf{Warunki końcowe:} Użytkownik widzi listę produktów spełniających kryteria wyszukiwania i filtrów sentymentu
