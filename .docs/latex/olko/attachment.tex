% filepath: /SmartRecommender-Project-Django-React/.docs/latex/olko/attachment.tex

\newpage
\section*{Załącznik A}
\addcontentsline{toc}{section}{Załącznik A: Szczegółowe scenariusze przypadków użycia}
\section*{Szczegółowe scenariusze przypadków użycia}

Poniżej przedstawiono szczegółowe opisy scenariuszy użycia systemu odpowiadających przypadkom użycia zaprezentowanym na diagramie w rozdziale 3.3. Scenariusze zostały opisane zgodnie z notacją: Aktor, Warunki początkowe, Przebieg scenariusza głównego, Warunki końcowe.

\subsection*{Scenariusz 1: Przeglądanie i wyszukiwanie produktów}

\noindent
\textbf{Aktor:} Gość (użytkownik niezalogowany)

\noindent
\textbf{Warunki początkowe:} Użytkownik znajduje się na stronie głównej aplikacji

\noindent
\textbf{Przebieg scenariusza głównego:}
\begin{enumerate}
\item Użytkownik otwiera stronę główną aplikacji w przeglądarce
\item Użytkownik otwiera pasek wyszukiwania z standardową opcją 
\item Użytkownik wpisuje szukaną frazę
\item System sortuje produkty według wartości \texttt{sentiment\_score} malejąco
\item Użytkownik przegląda listę produktów
\item Użytkownik klika na wybrany produkt
\item System wyświetla szczegółową stronę produktu z opisem, specyfikacjami, zdjęciami i opiniami
\end{enumerate}

\noindent
\textbf{Warunki końcowe:} Użytkownik widzi szczegółowe informacje o wybranym produkcie

\subsection*{Scenariusz 2: Dodawanie produktów do koszyka}

\noindent
\textbf{Aktor:} Gość (użytkownik niezalogowany)

\noindent
\textbf{Warunki początkowe:} Użytkownik przegląda szczegółową stronę produktu (laptop)

\noindent
\textbf{Przebieg scenariusza głównego:}
\begin{enumerate}
\item Użytkownik klika przycisk ,,Dodaj do koszyka''
\item System dodaje produkt do koszyka i wyświetla powiadomienie o sukcesie
\item Użytkownik klika ikonę koszyka w nawigacji
\item System wyświetla widok koszyka z dodanymi produktami
\item System automatycznie wywołuje algorytm Apriori dla produktów w koszyku
\item System wyświetla sekcję ,,Frequently Bought Together'' z trzema produktami rekomendowanymi przez algorytm Apriori (torba na laptopa, mysz bezprzewodowa, hub USB-C)
\item Każda rekomendacja zawiera metryki: lift, confidence oraz przycisk ,,Dodaj do koszyka''
\item Użytkownik przegląda rekomendacje
\item Użytkownik klika ,,Dodaj do koszyka'' przy torbie i myszy
\item System dodaje wybrane produkty do koszyka i aktualizuje sumę całkowitą
\end{enumerate}

\textbf{Warunki końcowe:} Koszyk zawiera laptop, torbę i mysz, wyświetlona jest zaktualizowana suma

\subsection*{Scenariusz 3: Rejestracja nowego użytkownika}

\noindent
\textbf{Aktor:} Gość (użytkownik niezalogowany)

\noindent
\textbf{Warunki początkowe:} Użytkownik znajduje się na stronie głównej

\noindent
\textbf{Przebieg scenariusza głównego:}
\begin{enumerate}
\item Użytkownik klika przycisk ,,Zarejestruj się'' w nawigacji
\item System wyświetla formularz rejestracji
\item Użytkownik wypełnia pola: imię, nazwisko, adres e-mail, hasło, potwierdzenie hasła
\item Użytkownik klika przycisk ,,Zarejestruj''
\item System waliduje dane: sprawdza czy e-mail jest unikalny, czy hasła się zgadzają, czy hasło spełnia wymagania bezpieczeństwa
\item System tworzy nowe konto użytkownika w bazie danych
\item System wyświetla komunikat o pomyślnej rejestracji
\item System automatycznie loguje użytkownika i generuje token JWT
\item System przekierowuje użytkownika do panelu klienta
\end{enumerate}

\textbf{Warunki końcowe:} Użytkownik jest zalogowany i posiada aktywne konto w systemie

\subsection*{Scenariusz 4: Logowanie do systemu}

\noindent
\textbf{Aktor:} Gość (użytkownik niezalogowany, posiadający konto)

\noindent
\textbf{Warunki początkowe:} Użytkownik znajduje się na stronie głównej

\noindent
\textbf{Przebieg scenariusza głównego:}
\begin{enumerate}
\item Użytkownik klika przycisk ,,Zaloguj się'' w nawigacji
\item System wyświetla formularz logowania
\item Użytkownik wpisuje adres e-mail i hasło
\item Użytkownik klika przycisk ,,Zaloguj''
\item System weryfikuje dane logowania w bazie danych
\item System generuje token JWT i zwraca go do klienta
\item Frontend zapisuje token w localStorage przeglądarki
\item System przekierowuje użytkownika do panelu klienta
\end{enumerate}

\noindent
\textbf{Warunki końcowe:} Użytkownik jest zalogowany, posiada ważny token JWT

\subsection*{Scenariusz 5: Śledzenie statusu zamówienia}

\noindent
\textbf{Aktor:} Klient (użytkownik zalogowany)

\noindent
\textbf{Warunki początkowe:} Użytkownik jest zalogowany, posiada co najmniej jedno zamówienie

\noindent
\textbf{Przebieg scenariusza głównego:}
\begin{enumerate}
\item Użytkownik otwiera panel klienta
\item Użytkownik klika zakładkę ,,My Orders''
\item System pobiera wszystkie zamówienia użytkownika z bazy danych
\item System wyświetla listę zamówień posortowaną od najnowszych
\item Dla każdego zamówienia wyświetlane są: numer zamówienia, status (oczekujące/w realizacji/zakończone/anulowane)
\end{enumerate}

\noindent
\textbf{Warunki końcowe:} Użytkownik widzi szczegółowe informacje o statusie wybranego zamówienia

\subsection*{Scenariusz 6: Wystawianie opinii o produkcie}

\noindent
\textbf{Aktor:} Klient (użytkownik zalogowany)

\noindent
\textbf{Warunki początkowe:} Użytkownik posiada zakończone zamówienie zawierające produkt

\noindent
\textbf{Przebieg scenariusza głównego:}
\begin{enumerate}
\item Użytkownik otwiera panel klienta, zakładkę ,,My Orders''
\item Użytkownik znajduje zakończone zamówienie
\item Użytkownik klika przycisk ,,Add Review'' przy zakupionym produkcie
\item System wyświetla formularz opinii
\item Użytkownik wybiera ocenę gwiazdkową (1-5 gwiazdek)
\item Użytkownik wpisuje treść tekstową opinii
\item Użytkownik klika ,,Wyślij opinię''
\item System zapisuje opinię w bazie danych z powiązaniem: użytkownik, produkt, zamówienie
\item System automatycznie wywołuje algorytm analizy sentymentu dla nowej opinii
\item Algorytm tokenizuje tekst, wyszukuje słowa w słowniku pozytywnym/negatywnym
\item Algorytm oblicza sentiment\_score dla opinii zgodnie ze wzorem (2)
\item System przelicza zagregowany sentyment produktu zgodnie ze wzorem (3) (wieloźródłowa agregacja z 5 źródeł)
\item System aktualizuje pole \texttt{sentiment\_score} w rekordzie produktu
\item System wyświetla komunikat o pomyślnym dodaniu opinii
\item System odświeża widok produktu - zaktualizowany sentyment jest widoczny
\end{enumerate}

\noindent
\textbf{Warunki końcowe:} Opinia została dodana, sentyment produktu zaktualizowany

\subsection*{Scenariusz 7: Zarządzanie kontem użytkownika}

\noindent
\textbf{Aktor:} Klient (użytkownik zalogowany)

\noindent
\textbf{Warunki początkowe:} Użytkownik jest zalogowany

\noindent
\textbf{Przebieg scenariusza głównego:}
\begin{enumerate}
\item Użytkownik otwiera panel klienta
\item Użytkownik klika zakładkę ,,Konto''
\item System wyświetla formularz z danymi użytkownika: imię, nazwisko, e-mail
\item Użytkownik edytuje wybrane pola (np. zmienia nazwisko)
\item Użytkownik klika ,,Zapisz zmiany''
\item System waliduje dane (sprawdza czy e-mail jest unikalny, jeśli został zmieniony)
\item System aktualizuje rekord użytkownika w bazie danych
\item System wyświetla komunikat o pomyślnym zapisaniu zmian
\end{enumerate}

\noindent
\textbf{Warunki końcowe:} Dane użytkownika zostały zaktualizowane

\subsection*{Scenariusz 8: Administrator zarządza produktami}

\noindent
\textbf{Aktor:} Administrator

\noindent
\textbf{Warunki początkowe:} Administrator jest zalogowany, posiada uprawnienia administratora

\noindent
\textbf{Przebieg scenariusza głównego - dodawanie produktu:}
\begin{enumerate}
\item Administrator otwiera panel administracyjny
\item Administrator klika sekcję ,,Products''
\item System wyświetla listę wszystkich produktów
\item Administrator klika przycisk ,,Add New Product''
\item System wyświetla formularz dodawania produktu
\item Administrator wypełnia pola: nazwa, opis, cena, kategorie, specyfikacje techniczne, przesyła zdjęcie
\item Administrator klika ,,Zapisz produkt''
\item System waliduje dane (sprawdza czy wszystkie wymagane pola są wypełnione)
\item System tworzy nowy rekord produktu w bazie danych
\item System automatycznie wywołuje algorytm analizy sentymentu dla nowego produktu
\item Algorytm oblicza sentiment\_score na podstawie nazwy i opisu (brak jeszcze opinii)
\item System zapisuje obliczony sentyment w rekordzie produktu
\item System wyświetla komunikat o pomyślnym dodaniu produktu
\item System odświeża listę produktów
\end{enumerate}

\noindent
\textbf{Warunki końcowe:} Nowy produkt został dodany do katalogu z obliczonym sentymentem

\subsection*{Scenariusz 9: Administrator zarządza zamówieniami}

\noindent
\textbf{Aktor:} Administrator

\noindent
\textbf{Warunki początkowe:} Administrator jest zalogowany

\noindent
\textbf{Przebieg scenariusza głównego:}
\begin{enumerate}
\item Administrator otwiera panel administracyjny
\item Administrator klika sekcję ,,Orders''
\item System wyświetla listę wszystkich zamówień w systemie
\item Dla każdego zamówienia wyświetlane są: numer, aktualny status
\item Administrator klika na wybrane zamówienie ze statusem ,,oczekujące''
\item Administrator klika przycisk ,,Zmień status''
\item System wyświetla listę dostępnych statusów (w realizacji, zakończone, anulowane)
\item Administrator wybiera ,,w realizacji''
\item System aktualizuje status zamówienia w bazie danych
\item System zapisuje zdarzenie zmiany statusu z timestampem
\item System wyświetla komunikat o pomyślnej zmianie statusu
\item Użytkownik właściciel zamówienia widzi zaktualizowany status w swoim panelu
\end{enumerate}

\noindent
\textbf{Warunki końcowe:} Status zamówienia został zmieniony, użytkownik ma dostęp do aktualnej informacji

\subsection*{Scenariusz 10: Administrator zarządza użytkownikami}

\noindent
\textbf{Aktor:} Administrator

\noindent
\textbf{Warunki początkowe:} Administrator jest zalogowany

\noindent
\textbf{Przebieg scenariusza głównego:}
\begin{enumerate}
\item Administrator otwiera panel administracyjny
\item Administrator klika sekcję ,,Users''
\item System wyświetla listę wszystkich użytkowników
\item Dla każdego użytkownika wyświetlane są: ID, imię, nazwisko, e-mail, rola, data rejestracji
\item Administrator może filtrować użytkowników według roli (klient/administrator)
\item Administrator klika na wybranego użytkownika
\item System wyświetla szczegóły użytkownika: profil, historia zamówień, statystyki
\item Administrator może zmienić rolę użytkownika (nadać uprawnienia administratora) lub usunąć konto
\end{enumerate}

\noindent
\textbf{Warunki końcowe:} Administrator ma pełny wgląd w dane użytkowników i może zarządzać uprawnieniami

\subsection*{Scenariusz 11: Administrator przegląda statystyki i dashboard}

\noindent
\textbf{Aktor:} Administrator

\noindent
\textbf{Warunki początkowe:} Administrator jest zalogowany

\noindent
\textbf{Przebieg scenariusza głównego:}
\begin{enumerate}
\item Administrator otwiera panel administracyjny
\item System automatycznie wyświetla dashboard ze statystykami
\item Dashboard zawiera następujące kluczowe metryki:
  \begin{itemize}
  \item Całkowita liczba produktów w katalogu
  \item Całkowita liczba zarejestrowanych użytkowników
  \item Całkowita liczba zamówień (wszystkie statusy)
  \item Całkowity przychód (suma wartości zamówień zakończonych)
  \end{itemize}
\item System wyświetla wykresy:
  \begin{itemize}
  \item Wykres słupkowy: miesięczny obrót (ostatnie 12 miesięcy)
  \item Wykres kołowy: rozkład zamówień według kategorii produktów
  \item Wykres liniowy: liczba nowych użytkowników (ostatnie 30 dni)
  \end{itemize}
\item Administrator klika sekcję ,,Statistics'' dla bardziej szczegółowych analiz
\item System wyświetla widok do zmiany rekomentacji, odświeżanie reguł asocjacyjnych
\end{enumerate}

\noindent
\textbf{Warunki końcowe:} Administrator ma pełny wgląd w kluczowe wskaźniki biznesowe systemu

\subsection*{Scenariusz 12: Administrator debuguje algorytmy rekomendacji}

\noindent
\textbf{Aktor:} Administrator

\noindent
\textbf{Warunki początkowe:} Administrator jest zalogowany, algorytmy rekomendacji zostały przeliczone

\noindent
\textbf{Przebieg scenariusza głównego - Collaborative Filtering:}
\begin{enumerate}
\item Administrator otwiera panel administracyjny
\item Administrator klika sekcję ,,Debug''
\item Administrator wybiera zakładkę ,,Collaborative Filtering''
\item System wyświetla panel debugowania Collaborative Filtering zawierający:
  \begin{itemize}
  \item Wzór matematyczny Adjusted Cosine Similarity
  \item Statystyki: liczba par produktów, średnie podobieństwo, czas ostatniego przeliczenia
  \item Tabelę z przykładowymi podobieństwami (product\_a, product\_b, similarity\_score)
  \end{itemize}
\item Administrator może wpisać ID produktu i zobaczyć top 10 najbardziej podobnych produktów
\item System pobiera dane z tabeli ProductSimilarity i wyświetla wyniki
\end{enumerate}

\noindent
\textbf{Warunki końcowe:} Administrator zweryfikował poprawność działania algorytmu Collaborative Filtering

\subsection*{Scenariusz 13: Przeglądanie rekomendacji na stronie głównej}

\noindent
\textbf{Aktor:} Klient (użytkownik zalogowany)

\noindent
\textbf{Warunki początkowe:} Użytkownik jest zalogowany, posiada historię zakupów

\noindent
\textbf{Przebieg scenariusza głównego:}
\begin{enumerate}
\item Użytkownik loguje się do systemu
\item System przekierowuje użytkownika do strony głównej
\item System wywołuje algorytm Collaborative Filtering dla zalogowanego użytkownika
\item Algorytm analizuje historię zakupów użytkownika
\item Algorytm pobiera produkty podobne do wcześniej zakupionych (na podstawie macierzy podobieństw)
\item System wyświetla sekcję ,,Rekomendowane dla Ciebie'' z 6 produktami
\item Użytkownik przegląda rekomendacje
\item Użytkownik klika na wybrany produkt
\item System wyświetla szczegółową stronę produktu
\end{enumerate}

\noindent
\textbf{Warunki końcowe:} Użytkownik widzi spersonalizowane rekomendacje oparte na swojej historii zakupów

\subsection*{Scenariusz 14: Przeglądanie podobnych produktów (Content-Based Filtering)}

\noindent
\textbf{Aktor:} Klient lub Gość

\noindent
\textbf{Warunki początkowe:} Użytkownik przegląda szczegółową stronę produktu (np. laptop Dell XPS 15)

\noindent
\textbf{Przebieg scenariusza głównego:}
\begin{enumerate}
\item Użytkownik otwiera stronę szczegółową produktu
\item System automatycznie wywołuje endpoint \texttt{GET /api/\allowbreak recommendations/\allowbreak content-based/\allowbreak ?product\_id=\{id\}}
\item Backend wywołuje metodę \texttt{ContentBasedAPI} z parametrem \texttt{product\_id}
\item System wykonuje zapytanie do tabeli \texttt{ProductSimilarity} filtrując po \texttt{product1\_id = \{id\}} oraz \texttt{type=\allowbreak 'content\_based'}
\item System sortuje wyniki malejąco według \texttt{similarity\_score}
\item System pobiera Top 10 najbardziej podobnych produktów z bazy danych
\item Backend zwraca JSON z listą produktów zawierającą: \texttt{product\_id}, \texttt{name}, \texttt{price}, \\\texttt{similarity\_score}
\item Frontend wyświetla sekcję „Podobne produkty" z kafelkami produktów
\item Każdy kafelek zawiera zdjęcie, nazwę, cenę oraz wskaźnik podobieństwa (np. „85\% podobny")
\item Użytkownik przegląda rekomendacje
\item Użytkownik klika na wybrany podobny produkt
\item System przekierowuje do strony szczegółowej wybranego produktu
\end{enumerate}

\noindent
\textbf{Warunki końcowe:} Użytkownik widzi produkty podobne do oglądanego, obliczone na podstawie cech (kategorie 40\%, tagi 30\%, cena 20\%, słowa kluczowe 10\%)

\subsection*{Scenariusz 15: Generowanie macierzy podobieństw Content-Based Filtering}

\noindent
\textbf{Aktor:} Administrator

\noindent
\textbf{Warunki początkowe:} Administrator jest zalogowany, katalog zawiera produkty

\noindent
\textbf{Przebieg scenariusza głównego:}
\begin{enumerate}
\item Administrator otwiera panel administracyjny
\item Administrator klika sekcję „Admin Tools"
\item Administrator wybiera zakładkę „Content-Based Filtering"
\item System wyświetla panel zarządzania CBF z przyciskiem „Generate Similarity Matrix"
\item Administrator klika „Generate Similarity Matrix"
\item System wywołuje endpoint \texttt{POST /api/admin/\allowbreak content-based/\allowbreak generate}
\item Backend uruchamia algorytm \texttt{ContentBased\allowbreak Recommendation\allowbreak Engine.\allowbreak generate\_\allowbreak \\similarities()}
\item System pobiera wszystkie produkty z bazy danych
\item Dla każdego produktu system ekstraktuje cechy:
  \begin{itemize}
  \item Kategorie (waga 40\%) - konwersja do wektora one-hot
  \item Tagi (waga 30\%) - ekstrakcja unikalnych tagów
  \item Przedział cenowy (waga 20\%) - normalizacja do zakresu [0,1]
  \item Słowa kluczowe z opisu (waga 10\%) - tokenizacja i TF-IDF
  \end{itemize}
\item System oblicza podobieństwo cosinusowe między każdą parą produktów zgodnie ze wzorem Weighted Cosine Similarity
\item System filtruje pary z podobieństwem $>$ 20\% (próg minimalny)
\item System usuwa stare rekordy z tabeli \texttt{ProductSimilarity} dla \texttt{type='content\_based'}
\item System zapisuje nowe podobieństwa do tabeli \texttt{ProductSimilarity}
\item System wyświetla komunikat o sukcesie: „Wygenerowano 1247 podobieństw dla 500 produktów w czasie 58 sekund"
\item Administrator widzi zaktualizowane statystyki: liczba par produktów, średnie podobieństwo, timestamp ostatniego przeliczenia
\end{enumerate}

\noindent
\textbf{Warunki końcowe:} Macierz podobieństw została wygenerowana i zapisana w bazie danych, rekomendacje CBF są aktualne

\subsection*{Scenariusz 16: Debugowanie algorytmu Content-Based Filtering}

\noindent
\textbf{Aktor:} Administrator

\noindent
\textbf{Warunki początkowe:} Administrator jest zalogowany, macierz podobieństw została wygenerowana

\noindent
\textbf{Przebieg scenariusza głównego:}
\begin{enumerate}
\item Administrator otwiera panel administracyjny
\item Administrator klika sekcję „Debug"
\item Administrator wybiera zakładkę „Content-Based Debug"
\item System wyświetla panel debugowania CBF zawierający:
  \begin{itemize}
  \item Wzór matematyczny Weighted Cosine Similarity z opisem wag cech
  \item Statystyki globalne: liczba produktów w katalogu, liczba wygenerowanych par, średnie podobieństwo, czas ostatniego przeliczenia
  \item Histogram rozkładu podobieństw (przedziały: 20-40\%, 40-60\%, 60-80\%, 80-100\%)
  \end{itemize}
\item Administrator wprowadza ID produktu w pole „Product ID for analysis"
\item Administrator klika „Analyze Product"
\item System wywołuje endpoint \texttt{GET /api/admin/\allowbreak content-based/\allowbreak debug/\allowbreak ?product\_id=\{id\}}
\item System pobiera wybrany produkt oraz jego Top 10 najbardziej podobnych produktów
\item System wyświetla szczegółową analizę dla wybranego produktu:
  \begin{itemize}
  \item Podstawowe dane: nazwa, kategorie, tagi, cena
  \item Ekstrahowane cechy: wektor kategorii, lista tagów, znormalizowana cena, słowa kluczowe
  \item Tabelę Top 10 podobnych produktów z kolumnami: ID, nazwa, similarity\_score, wspólne kategorie, wspólne tagi
  \end{itemize}
\item Administrator analizuje wyniki i weryfikuje poprawność podobieństw
\item Administrator może wyeksportować dane do CSV klikając „Export to CSV"
\end{enumerate}

\noindent
\textbf{Warunki końcowe:} Administrator zweryfikował poprawność działania algorytmu CBF dla wybranego produktu

\subsection*{Scenariusz 17: Przeglądanie rekomendacji Fuzzy Logic w panelu klienta}

\noindent
\textbf{Aktor:} Klient (użytkownik zalogowany)

\noindent
\textbf{Warunki początkowe:} Użytkownik jest zalogowany, posiada historię zakupów

\noindent
\textbf{Przebieg scenariusza głównego:}
\begin{enumerate}
\item Użytkownik loguje się do systemu
\item Użytkownik otwiera panel klienta
\item Użytkownik klika zakładkę „Fuzzy Logic Recommendations"
\item System wywołuje endpoint \texttt{GET /api/\allowbreak fuzzy-logic-\allowbreak recommendations/\allowbreak ?limit=10}
\item Backend wywołuje \texttt{FuzzyLogicAPI} z parametrem \texttt{user\_id} (z tokenu JWT)
\item System inicjalizuje silnik rozmyty \texttt{FuzzyEngine.\allowbreak initialize\_\allowbreak fuzzy\_system(\allowbreak user\_id)}
\item System buduje profil użytkownika \texttt{FuzzyUserProfile}:
  \begin{itemize}
  \item Oblicza parametr \texttt{price\_sensitivity} na podstawie średniej ceny zakupionych produktów
  \item Ekstraktuje preferowane kategorie z historii zakupów (top 3 kategorie)
  \end{itemize}
\item System pobiera wszystkie produkty z bazy danych z metrykami: cena, średni rating, liczba zamówień
\item Dla każdego produktu system wykonuje wnioskowanie rozmyte:
  \begin{itemize}
  \item Fuzzyfikacja - oblicza przynależność ceny, jakości i popularności do zbiorów rozmytych (cheap/medium/expensive, low/medium/high)
  \item Aplikacja 6 reguł IF-THEN typu Mamdani (R1: preferred category match, R2: quality-price balance, R3: price sensitive match, R4: premium match, R5: popular high quality, R6: budget friendly)
  \item Agregacja wyników reguł z wagami (R1: 0.9, R2: 0.85, R3: 0.6, R4: 0.75, R5: 0.8, R6: 0.7)
  \item Defuzzyfikacja - oblicza końcowy \texttt{fuzzy\_score} metodą średniej ważonej
  \end{itemize}
\item System sortuje produkty malejąco według \texttt{fuzzy\_score}
\item System zwraca JSON z Top 10 rekomendacji zawierający: \texttt{product\_id}, \texttt{name}, \texttt{price}, \\\texttt{fuzzy\_score}, \texttt{activated\_rules}
\item Frontend wyświetla trzy zakładki:
  \begin{itemize}
  \item „Fuzzy Recommendations" - lista Top 10 produktów z fuzzy\_score i aktywowanymi regułami
  \item „User Profile" - wyświetla \texttt{price\_sensitivity} (np. 60\% - moderate) oraz preferowane kategorie
  \item „Fuzzy Rules" - opis 6 reguł IF-THEN z wyjaśnieniem logiki
  \end{itemize}
\item Użytkownik przegląda rekomendacje dopasowane do jego profilu cenowego
\item Użytkownik klika przycisk „View Rule Activations" przy wybranym produkcie
\item System wyświetla szczegóły aktywacji reguł: które reguły zostały aktywowane, z jaką siłą ($\alpha_i$), jaki był ich wkład w końcowy wynik
\end{enumerate}

\noindent
\textbf{Warunki końcowe:} Użytkownik widzi spersonalizowane rekomendacje dopasowane do jego wrażliwości cenowej oraz preferowanych kategorii z pełną przejrzystością procesu decyzyjnego

\subsection*{Scenariusz 18: Wyszukiwanie tolerujące błędy (Fuzzy Search)}

\noindent
\textbf{Aktor:} Klient lub Gość

\noindent
\textbf{Warunki początkowe:} Użytkownik znajduje się na stronie głównej

\noindent
\textbf{Przebieg scenariusza głównego:}
\begin{enumerate}
\item Użytkownik otwiera pasek wyszukiwania
\item Użytkownik przełącza opcję wyszukiwania na „Fuzzy Search"
\item System wyświetla suwak „Fuzzy Threshold" (domyślnie 0.5 = 50\%)
\item Użytkownik wpisuje frazę z literówką: „loptap" (zamiast „laptop")
\item Użytkownik ustawia próg podobieństwa na 0.3 (30\%)
\item Użytkownik klika przycisk „Search"
\item System wywołuje endpoint \texttt{GET /api/search/\allowbreak fuzzy/\allowbreak ?query=\allowbreak loptap\&\allowbreak threshold=0.3}
\item Backend iteruje po wszystkich produktach w katalogu
\item Dla każdego produktu system oblicza odległość Levenshteina między zapytaniem a nazwą produktu:
  \begin{itemize}
  \item Algorytm rekurencyjny oblicza minimalną liczbę operacji edycji (wstawienie, usunięcie, zamiana znaku)
  \item System normalizuje odległość do zakresu [0,1] dzieląc przez maksymalną długość
  \item Oblicza \texttt{fuzzy\_match\_score} = 1 - normalized\_distance
  \end{itemize}
\item System filtruje produkty z \texttt{fuzzy\_match\_score} $\geq$ 0.3 (próg użytkownika)
\item System sortuje wyniki malejąco według \texttt{fuzzy\_match\_score}
\item Backend zwraca JSON z produktami: \texttt{product\_id}, \texttt{name}, \texttt{price}, \texttt{fuzzy\_match\_score}
\item Frontend wyświetla wyniki wyszukiwania:
  \begin{itemize}
  \item Laptop Dell XPS 15 - 86\% match (zawiera „Laptop")
  \item Laptop HP Pavilion - 86\% match (zawiera „Laptop")
  \item Desktop Hub USB-C - 45\% match (częściowe dopasowanie)
  \end{itemize}
\item Użytkownik widzi produkty mimo błędu w zapytaniu
\item Użytkownik klika na wybrany produkt
\end{enumerate}

\noindent
\textbf{Warunki końcowe:} Użytkownik znalazł poszukiwane produkty mimo literówki dzięki algorytmowi Levenshteina

\subsection*{Scenariusz 19: Debugowanie algorytmu Fuzzy Logic}

\noindent
\textbf{Aktor:} Administrator

\noindent
\textbf{Warunki początkowe:} Administrator jest zalogowany, system posiada użytkowników z historią zakupów

\noindent
\textbf{Przebieg scenariusza głównego:}
\begin{enumerate}
\item Administrator otwiera panel administracyjny
\item Administrator klika sekcję „Debug"
\item Administrator wybiera zakładkę „Fuzzy Logic Debug"
\item System wyświetla widok ogólny panelu debugowania zawierający:
  \begin{itemize}
  \item Szczegóły algorytmu: metoda Mamdani Fuzzy Inference, liczba reguł (6), T-norma (min), T-conorma (max)
  \item Definicje funkcji przynależności dla price (cheap: 0-500 PLN, medium: 300-1500 PLN, expensive: 1000+ PLN)
  \item Definicje funkcji przynależności dla quality (low: 0-2.5, medium: 2.0-4.0, high: 3.5-5.0)
  \item Definicje funkcji przynależności dla popularity (low: 0-10 zamówień, medium: 5-50, high: 30+ zamówień)
  \item Listę profili użytkowników z parametrem price\_sensitivity
  \end{itemize}
\item Administrator wprowadza ID produktu w pole „Product ID for detailed analysis"
\item Administrator klika „Analyze Product"
\item System wywołuje endpoint \texttt{GET /api/admin/\allowbreak fuzzy-debug/\allowbreak ?product\_id=\{id\}}
\item System pobiera wybrany produkt oraz wykonuje szczegółową ewaluację rozmytą
\item System wyświetla widok szczegółowy dla produktu:
  \begin{itemize}
  \item Dane produktu: nazwa, cena, średni rating, liczba zamówień
  \item Wartości fuzzyfikacji:
    \begin{itemize}
    \item Price: cheap=0.3, medium=0.7, expensive=0.0
    \item Quality: low=0.0, medium=0.2, high=0.8
    \item Popularity: low=0.0, medium=0.6, high=0.4
    \end{itemize}
  \item Aktywacja wszystkich 6 reguł z siłą aktywacji $\alpha_i$:
    \begin{itemize}
    \item R1 (Preferred Category Match): $\alpha_1$ = 0.85, waga = 0.9, wkład = 0.765
    \item R2 (Quality-Price Balance): $\alpha_2$ = 0.7, waga = 0.85, wkład = 0.595
    \item R3 (Price Sensitive Match): $\alpha_3$ = 0.3, waga = 0.6, wkład = 0.18
    \item R4 (Premium Match): $\alpha_4$ = 0.0, waga = 0.75, wkład = 0.0
    \item R5 (Popular High Quality): $\alpha_5$ = 0.6, waga = 0.8, wkład = 0.48
    \item R6 (Budget Friendly): $\alpha_6$ = 0.2, waga = 0.7, wkład = 0.14
    \end{itemize}
  \item Agregacja: suma wkładów = 2.16, fuzzy\_score = 2.16 / 4.6 = 0.47 (47\%)
  \end{itemize}
\item Administrator analizuje które reguły dominują dla danego produktu
\item Administrator może dostroić wagi reguł klikając „Edit Rule Weights"
\item System zapisuje nowe wagi w konfiguracji
\end{enumerate}

\noindent
\textbf{Warunki końcowe:} Administrator zweryfikował proces wnioskowania rozmytego i zoptymalizował wagi reguł

\subsection*{Scenariusz 20: Przeglądanie rekomendacji probabilistycznych w panelu klienta}

\noindent
\textbf{Aktor:} Klient (użytkownik zalogowany)

\noindent
\textbf{Warunki początkowe:} Użytkownik jest zalogowany, posiada historię zakupów, modele probabilistyczne zostały wytrenowane

\noindent
\textbf{Przebieg scenariusza głównego:}
\begin{enumerate}
\item Użytkownik loguje się do systemu
\item Użytkownik otwiera panel klienta
\item Użytkownik klika zakładkę „Probabilistic Recommendations"
\item System wywołuje endpoint \texttt{GET /api/\allowbreak probabilistic-\allowbreak recommendations/\allowbreak \\?user\_id=\{id\}}
\item Backend wywołuje \texttt{ProbabilisticAPI} z parametrem \texttt{user\_id}
\item System pobiera historię zamówień użytkownika z bazy danych wywołując \texttt{get\_user\_\allowbreak \\orders(\allowbreak user\_id)}
\item System ekstraktuje sekwencję kategorii produktów z zamówień uporządkowanych chronologicznie
\item System wywołuje \texttt{MarkovChain.\allowbreak predict\_next\_\allowbreak products(\allowbreak order\_\allowbreak sequence)}:
  \begin{itemize}
  \item Identyfikuje ostatnią zakupioną kategorię (np. „Laptopy")
  \item Odczytuje macierz przejść dla kategorii „Laptopy"
  \item Pobiera Top 3 kategorie z najwyższym prawdopodobieństwem przejścia (np. Akcesoria komputerowe: 0.45, Peryferia: 0.30, Torby: 0.15)
  \item Dla każdej kategorii pobiera Top produkty
  \end{itemize}
\item System wywołuje \texttt{NaiveBayes.\allowbreak predict\_\allowbreak purchase\_\allowbreak probability(\allowbreak user\_profile)}:
  \begin{itemize}
  \item Ekstraktuje 5 cech użytkownika: liczba zamówień, średnia wartość, dni od ostatniego zakupu, ulubiona kategoria, częstotliwość
  \item Stosuje twierdzenie Bayesa dla klasyfikacji binarnej (kupi/nie kupi)
  \item Oblicza prawdopodobieństwo zakupu w najbliższym okresie
  \end{itemize}
\item System agreguje wyniki: 60\% waga dla Markov, 40\% waga dla Naive Bayes
\item Backend zwraca JSON z Top 10 rekomendacji zawierający: \texttt{product\_id}, \texttt{name}, \texttt{price}, \\\texttt{markov\_probability}, \texttt{bayes\_probability}, \texttt{combined\_score}
\item Frontend wyświetla trzy zakładki:
  \begin{itemize}
  \item „Probabilistic Recommendations" - lista Top 10 produktów z prawdopodobieństwami
  \item „Markov Chain Visualization" - graf przejść między kategoriami z prawdopodobieństwami
  \item „Purchase Probability Analysis" - wykres prawdopodobieństwa zakupu oraz analiza cech użytkownika
  \end{itemize}
\item Użytkownik przegląda rekomendacje oparte na sekwencjach zakupowych
\item Użytkownik klika zakładkę „Markov Chain Visualization"
\item System wyświetla interaktywny graf pokazujący możliwe przejścia z obecnej kategorii do kolejnych
\item Użytkownik klika zakładkę „Purchase Probability Analysis"
\item System wyświetla:
  \begin{itemize}
  \item Prawdopodobieństwo zakupu: 73\% (high probability)
  \item Breakdown cech: liczba zamówień: 12 (pozytywny wpływ +0.15), średnia wartość: 450 PLN (pozytywny +0.10), dni od ostatniego: 15 (pozytywny +0.20)
  \item Rekomendacja: „Użytkownik jest aktywny, wysokie prawdopodobieństwo konwersji"
  \end{itemize}
\end{enumerate}

\noindent
\textbf{Warunki końcowe:} Użytkownik widzi rekomendacje przewidujące jego przyszłe zakupy oparte na łańcuchach Markowa i klasyfikatorze Bayesa

\subsection*{Scenariusz 21: Trenowanie modeli probabilistycznych przez administratora}

\noindent
\textbf{Aktor:} Administrator

\noindent
\textbf{Warunki początkowe:} Administrator jest zalogowany, system posiada historię zamówień

\noindent
\textbf{Przebieg scenariusza głównego:}
\begin{enumerate}
\item Administrator otwiera panel administracyjny
\item Administrator klika sekcję „Admin Tools"
\item Administrator wybiera zakładkę „Probabilistic Models"
\item System wyświetla panel zarządzania modelami probabilistycznymi
\item Administrator widzi status modeli:
  \begin{itemize}
  \item Markov Chain: ostatni trening 2024-01-15, liczba stanów: 48 kategorii, wygładzanie Laplace'a: $\alpha$ = 1.0
  \item Naive Bayes: ostatni trening 2024-01-15, liczba próbek treningowych: 200 zamówień, accuracy: 78\%
  \end{itemize}
\item Administrator klika przycisk „Train Markov Chain Model"
\item System wywołuje endpoint \texttt{POST /api/admin/\allowbreak probabilistic/\allowbreak train-markov}
\item Backend uruchamia \texttt{MarkovChainEngine.\allowbreak train()}:
  \begin{itemize}
  \item Pobiera wszystkie zamówienia z bazy danych uporządkowane chronologicznie per użytkownik
  \item Ekstraktuje sekwencje kategorii dla każdego użytkownika
  \item Buduje macierz przejść 48x48 (48 kategorii produktowych)
  \item Oblicza prawdopodobieństwa przejść: $P(s_j | s_i) = \frac{count(s_i \rightarrow s_j) + \alpha}{\sum_k count(s_i \rightarrow s_k) + \alpha \cdot N}$
  \item Stosuje wygładzanie Laplace'a ($\alpha$ = 1.0) zapobiegające zerowemu prawdopodobieństwu
  \item Zapisuje macierz w cache Django z timeout 24h
  \end{itemize}
\item System wyświetla komunikat: „Markov Chain trained successfully. 48 states, 2304 transitions, training time: 3.2s"
\item Administrator klika przycisk „Train Naive Bayes Model"
\item System wywołuje endpoint \texttt{POST /api/admin/\allowbreak probabilistic/\allowbreak train-bayes}
\item Backend uruchamia \texttt{NaiveBayesEngine.\allowbreak train()}:
  \begin{itemize}
  \item Pobiera wszystkich użytkowników z zamówieniami
  \item Dla każdego użytkownika ekstraktuje 5 cech: liczba zamówień, średnia wartość, dni od ostatniego, ulubiona kategoria, częstotliwość
  \item Tworzy etykiety binarne (aktywny/nieaktywny) na podstawie heurystyki (zakup w ostatnich 30 dniach)
  \item Oblicza prawdopodobieństwa warunkowe dla każdej cechy: $P(feature | class)$
  \item Oblicza prawdopodobieństwa a priori: $P(active)$, $P(inactive)$
  \item Waliduje model na zbiorze testowym (80\%/20\% split)
  \item Zapisuje model w tabeli cache
  \end{itemize}
\item System wyświetla komunikat: „Naive Bayes trained successfully. Training samples: 200, Test accuracy: 78\%, training time: 1.8s"
\item Administrator widzi zaktualizowane timestampy i metryki
\end{enumerate}

\noindent
\textbf{Warunki końcowe:} Administrator zweryfikował poprawność działania modeli probabilistycznych (Markov Chain i Naive Bayes)
