\subsubsection*{6.2.1 Panel debugowania Content-Based Filtering}

System oferuje zaawansowany panel debugowania dostępny przez endpoint \texttt{/api/content-based-debug/}. Panel prezentuje:

\textbf{Widok ogólny (bez parametru product\_id)}:

\begin{itemize}
    \item \textbf{Szczegóły algorytmu}: nazwa, metoda (Weighted Feature Vectors + Cosine Similarity), status
    \item \textbf{Wagi cech}: category (40\%), tag (30\%), price (20\%), keywords (10\%)
    \item \textbf{Statystyki bazy danych}: liczba produktów, zapisanych podobieństw, procent pokrycia
    \item \textbf{Status cache}: HIT/MISS, czas wygaśnięcia
    \item \textbf{Top 10 podobieństw}: produkty o najwyższym podobieństwie w systemie
\end{itemize}

\begin{figure}[h!]
  \centering
  \includegraphics[width=0.92\textwidth]{images/contentBasedAdminDebug1.png}
  \caption{Panel debugowania Content-Based Filtering.}
  \label{fig:cbf_debug1}
\end{figure}

\textbf{Widok szczegółowy (z parametrem product\_id)}:

Dla konkretnego produktu panel pokazuje:

\begin{itemize}
    \item Wektor cech produktu z wagami (słownik feature → weight)
    \item Top 10 produktów podobnych z szczegółami obliczeń
    \item Wzór matematyczny dla każdej pary: $\frac{dot\_product}{norm_1 \times norm_2}$
    \item Cechy wspólne między produktami
    \item Breakdown podobieństwa: ile procent z kategorii, ile z tagów, ile z ceny, ile z keywords
\end{itemize}

\begin{figure}[h!]
  \centering
  \includegraphics[width=0.92\textwidth]{images/contentBasedAdminDebug2.png}
  \caption{CBF - szczegółowa analiza podobieństwa produktu.}
  \label{fig:cbf_debug2}
\end{figure}

Panel umożliwia administratorowi:
\begin{itemize}
    \item Monitorowanie pokrycia rekomendacji (ile produktów ma podobieństwa)
    \item Identyfikację produktów bez podobieństw (słabo opisane metadane)
    \item Walidację działania wag (czy kategorie dominują prawidłowo)
    \item Ręczne wyzwalanie przeliczenia macierzy
\end{itemize}

\newpage

\subsubsection*{6.2.2 Interfejs użytkownika - sortowanie według CBF}

Metoda Content-Based Filtering jest wykorzystywana jako jedna z opcji sortowania produktów na stronie głównej sklepu. Administrator może wybrać algorytm CBF w ustawieniach systemu rekomendacji, co powoduje wyświetlanie produktów podobnych do tych, które użytkownik wcześniej przeglądał lub kupił.

\begin{figure}[h!]
  \centering
  \includegraphics[width=0.92\textwidth]{images/recomendationSystem.png}
  \caption{Rekomendacje Content-Based Filtering wyświetlane użytkownikowi na stronie głównej.}
  \label{fig:cbf_recommendations}
\end{figure}

\newpage

Rysunek \ref{fig:cbf_recommendations} przedstawia sekcję "Recommended For You (Content-Based)" wyświetlaną na stronie głównej aplikacji po wyborze algorytmu CBF przez administratora. System prezentuje 4 produkty podobne do wcześniej przeglądanych lub zakupionych artykułów, wybrane na podstawie analizy cech (kategoria, tagi, cena, słowa kluczowe). Każdy produkt zawiera zdjęcie, nazwę, cenę oraz przycisk dodania do koszyka.

Rekomendacje CBF są również dostępne w panelu klienta, gdzie użytkownik może zobaczyć produkty podobne do swoich poprzednich zakupów. System automatycznie identyfikuje produkty z wysokim współczynnikiem podobieństwa (powyżej 20\%) i prezentuje je w sekcji spersonalizowanych rekomendacji. Szczegółowy opis przepływu danych w systemie CBF przedstawiono w rozdziale 5.1.

\newpage

\section*{Rozdzia\l{} 6}
