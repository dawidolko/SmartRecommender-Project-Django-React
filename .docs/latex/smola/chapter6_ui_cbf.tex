\subsubsection*{6.2.1 Panel debugowania Content-Based Filtering}

System oferuje zaawansowany panel debugowania dostępny przez endpoint \texttt{/api/content-based-debug/}. Panel prezentuje:

\textbf{Widok ogólny (bez parametru product\_id)}:

\begin{itemize}
    \item \textbf{Szczegóły algorytmu}: nazwa, metoda (Weighted Feature Vectors + Cosine Similarity), status
    \item \textbf{Wagi cech}: category (40\%), tag (30\%), price (20\%), keywords (10\%)
    \item \textbf{Statystyki bazy danych}: liczba produktów, zapisanych podobieństw, procent pokrycia
    \item \textbf{Status cache}: HIT/MISS, czas wygaśnięcia
    \item \textbf{Top 10 podobieństw}: produkty o najwyższym podobieństwie w systemie
\end{itemize}

\begin{figure}[h!]
  \centering
  \includegraphics[width=0.92\textwidth]{images/contentBasedAdminDebug1.jpg}
  \caption{Panel debugowania Content-Based Filtering.}
  \label{fig:cbf_debug1}
\end{figure}

\textbf{Widok szczegółowy (z parametrem product\_id)}:

Dla konkretnego produktu panel pokazuje:

\begin{itemize}
    \item Wektor cech produktu z wagami (słownik feature → weight)
    \item Top 10 produktów podobnych z szczegółami obliczeń
    \item Wzór matematyczny dla każdej pary: $\frac{dot\_product}{norm_1 \times norm_2}$
    \item Cechy wspólne między produktami
    \item Breakdown podobieństwa: ile % z kategorii, ile z tagów, ile z ceny, ile z keywords
\end{itemize}

\begin{figure}[h!]
  \centering
  \includegraphics[width=0.92\textwidth]{images/contentBasedAdminDebug2.jpg}
  \caption{CBF - szczegółowa analiza podobieństwa produktu.}
  \label{fig:cbf_debug2}
\end{figure}

\textbf{Przykładowe dane z panelu debugowania}:

\begin{verbatim}
Algorithm: Content-Based Filtering (Cosine Similarity)
Formula: cos(θ) = (A·B) / (||A|| × ||B||)

Database Statistics:
- Total Products: 500
- Saved Similarities: 16038
- Percentage Saved: 6.43%
- Threshold: 20% (Only similarities > 20% are saved to database)

Feature Weights:
- Category: 40%
- Tag: 30%
- Price: 20%
- Keywords: 10%
\end{verbatim}

\textbf{Przykład wektora cech produktu} (ACEFAST Powerbank MagSafe M10 10000 mAh):

\begin{verbatim}
Feature Vector (12 features):
- category_accessories.powerbanks: 0.400
- tag_budget:                      0.300
- tag_portable:                    0.300
- tag_fast charging:               0.300
- tag_magsafe compatible:          0.300
- tag_wireless:                    0.300
- price_low:                       0.200
- keyword_charging:                0.020
- keyword_power:                   0.020
- keyword_fast:                    0.020
- keyword_devices:                 0.020
- keyword_magsafe:                 0.020

Top 10 Similar Products:
#1 Baseus Magnetic Mini Wireless Charging 20W 20000mAh z MagSafe: 99.90%
#2 Baseus Magnetic Mini Wireless Charging 20W 20000mAh z MagSafe: 99.90%
#3 Belkin Magnetic Wireless 5000mAh MagSafe + Stand: 99.90%
#4 Baseus mini 5000mAh 20W (magnetyczny): 92.70%
#5 Belkin 20000mAh (15W, USB-C, USB-A): 84.90%

Detailed Calculation (#1):
- Dot Product: 0.6512
- Norm Product 1: 0.8075
- Norm Product 2: 0.8075
- Formula: 0.6512 / (0.8075 × 0.8075) = 0.9988
- Verification: Stored: 0.999 | Calculated: 0.9988 ✓
- Common Features: 10 total
\end{verbatim}

Panel umożliwia administratorowi:
\begin{itemize}
    \item Monitorowanie pokrycia rekomendacji (ile produktów ma podobieństwa)
    \item Identyfikację produktów bez podobieństw (słabo opisane metadane)
    \item Walidację działania wag (czy kategorie dominują prawidłowo)
    \item Ręczne wyzwalanie przeliczenia macierzy
\end{itemize}

\subsubsection*{6.2.2 Interfejs użytkownika - sortowanie według CBF}

Metoda Content-Based Filtering jest wykorzystywana jako jedna z opcji sortowania produktów na stronie głównej sklepu. Administrator może wybrać algorytm CBF w ustawieniach systemu rekomendacji, co powoduje wyświetlanie produktów podobnych do tych, które użytkownik wcześniej przeglądał lub kupił.

\begin{figure}[h!]
  \centering
  \includegraphics[width=0.92\textwidth]{images/contentBasedAdmin1.jpg}
  \caption{Panel administratora - rekomendacje CBF.}
  \label{fig:cbf_admin}
\end{figure}

\begin{figure}[h!]
  \centering
  \includegraphics[width=0.92\textwidth]{images/fuzzyLogicAdmin1.jpg}
  \caption{Panel administratora - rekomendacje Fuzzy Logic.}
  \label{fig:fuzzy_admin}
\end{figure}

Porównanie rysunków \ref{fig:cbf_admin} i \ref{fig:fuzzy_admin} demonstruje kluczową różnicę między metodami: przełączenie algorytmu z Content-Based Filtering na Fuzzy Logic powoduje natychmiastową zmianę rekomendowanych produktów. CBF poleca produkty o podobnych cechach (kategoria, tagi, cena), podczas gdy Fuzzy Logic uwzględnia dodatkowo rozmyty profil użytkownika i reguły wnioskowania.

\begin{figure}[h!]
  \centering
  \includegraphics[width=\textwidth]{images/contentBasedSequenceDiagram.png}
  \caption{Diagram sekwencji - Content-Based Filtering.}
  \label{fig:cbf_sequence}
\end{figure}

Proces generowania rekomendacji CBF przebiega następująco:
\begin{enumerate}
    \item Klient przegląda produkt na stronie ProductPage
    \item Frontend wysyła żądanie GET do \newline
    \texttt{/api/recommendations/content-based/?product\_id=\{id\}}
    \item Backend (ContentBasedAPI) wykonuje zapytanie do modelu ProductSimilarity
    \item Model pobiera top 10 podobnych produktów z bazy danych (sortowanie po \texttt{-similarity\_score})
    \item Backend zwraca odpowiedź 200 OK z listą rekomendacji (product\_id, name, price, similarity\_score)
    \item Frontend wyświetla sekcję "Podobne produkty" na stronie produktu
\end{enumerate}

Rekomendacje CBF są również dostępne w panelu klienta, gdzie użytkownik może zobaczyć produkty podobne do swoich poprzednich zakupów. System automatycznie identyfikuje produkty z wysokim współczynnikiem podobieństwa (powyżej 20\%) i prezentuje je w sekcji spersonalizowanych rekomendacji.

\newpage

\section*{Rozdzia\l{} 6}
