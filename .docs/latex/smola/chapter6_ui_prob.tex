% filepath: /SmartRecommender-Project-Django-React/.docs/latex/smola/chapter6_ui_prob.tex

\subsubsection*{6.4.1 Panel debugowania modeli probabilistycznych}

Panel debugowania prezentuje szczegółowe informacje o obu modelach:

\begin{figure}[H]
  \centering
  \includegraphics[width=0.92\textwidth]{images/probabilisticMethodsAdminDebug1.png}
  \caption{Panel debugowania - Markov Chain.}
  \label{fig:prob_debug1}
\end{figure}

\textbf{Statystyki Markov Chain} (z panelu debugowania):
\begin{itemize}
    \item Rząd łańcucha (Order): 1 (first-order Markov Chain)
    \item Liczba stanów (kategorii): 48
    \item Liczba przejść (transitions): 48
\end{itemize}

\begin{figure}[h!]
  \centering
  \includegraphics[width=0.92\textwidth]{images/probabilisticMethodsAdminDebug2.png}
  \caption{Panel debugowania - Naive Bayes.}
  \label{fig:prob_debug2}
\end{figure}

\newpage

\textbf{Statystyki Naive Bayes} (z panelu debugowania):

\textit{Purchase Prediction}:
\begin{itemize}
    \item Trained: Yes
    \item Number of Features: 3
    \item Classes: will\_not\_purchase
    \item Class Priors: will\_not\_purchase = 1.0
\end{itemize}

\textit{Churn Prediction}:
\begin{itemize}
    \item Trained: Yes
    \item Number of Features: 3
    \item Classes: will\_churn, will\_not\_churn
    \item Class Priors: will\_churn = 0.95, will\_not\_churn = 0.05
\end{itemize}

\subsubsection*{6.4.2 Interfejs użytkownika - rekomendacje probabilistyczne}

Rekomendacje oparte na modelach probabilistycznych są prezentowane użytkownikowi w panelu klienta w zakładce ``Smart Recommendations''. Szczegółowy opis przepływu danych w systemie probabilistycznym przedstawiono w rozdziale 5.3. System wyświetla dwie podzakładki:
\begin{itemize}
    \item \textbf{Next Purchase (Markov)}: produkty z kategorii przewidywanych przez łańcuch Markowa jako najbardziej prawdopodobne do zakupu
    \item \textbf{Behavior Insights (Bayesian)}: analiza zachowań zakupowych użytkownika z wykorzystaniem Naive Bayes
\end{itemize}

\begin{figure}[h!]
  \centering
  \includegraphics[width=0.92\textwidth]{images/probabilisticMethodsClient1.png}
  \caption{Zakładka "Next Purchase (Markov)".}
  \label{fig:prob_client1}
\end{figure}

Zakładka "Next Purchase (Markov)" prezentuje:
\begin{itemize}
    \item \textbf{Next Purchase Probability}: prawdopodobieństwo zakupu w ciągu 30 dni (np. 50\%)
    \item \textbf{Expected Days Until Next Purchase}: przewidywany czas do następnego zakupu
    \item \textbf{Likely Next Products}: lista produktów z najwyższym Prediction Score (np. Imou Cruiser 2 5MP: 13\%, A4Tech HD PK-910P: 13\%)
    \item \textbf{Your Shopping Patterns}: najczęstsza sekwencja zakupów i długość cyklu (np. power.strips → laptop.hubs → office.accessories, 10 products per cycle)
\end{itemize}

\begin{figure}[h!]
  \centering
  \includegraphics[width=0.92\textwidth]{images/probabilisticMethodsClient2.png}
  \caption{Zakładka "Behavior Insights (Bayesian)".}
  \label{fig:prob_client2}
\end{figure}

Zakładka ``Behavior Insights (Bayesian)'' wykorzystuje model Naive Bayes do analizy preferencji zakupowych:
\begin{itemize}
    \item \textbf{Purchase Likelihood}: wykres słupkowy prawdopodobieństwa zakupu dla każdej kategorii
    \item Kategorie z najwyższym prawdopodobieństwem: electronics.phones (10\%), power.strips (9\%), accessories.cables (9\%), office.accessories (9\%)
    \item Model uczy się na podstawie historii zakupów wszystkich użytkowników i tworzy profil behawioralny
\end{itemize}

\begin{figure}[h!]
  \centering
  \includegraphics[width=0.92\textwidth]{images/probabilisticMethodsClient3.png}
  \caption{Zakładka "Churn Risk Analysis".}
  \label{fig:prob_client3}
\end{figure}

Zakładka ``Churn Risk Analysis'' (Rysunek \ref{fig:prob_client3}) prezentuje:
\begin{itemize}
    \item \textbf{Churn Risk}: poziom ryzyka rezygnacji klienta (np. 25\% - LOW RISK)
    \item \textbf{Shopping Behavior Analysis}: analiza wzorców zakupowych użytkownika
    \item \textbf{Personalized Suggestions}: spersonalizowane sugestie produktów
\end{itemize}

\subsubsection*{6.4.3 Panel administracyjny - widoki probabilistyczne}

Panel administracyjny systemu rekomendacji probabilistycznych prezentuje zaawansowane analizy dla administratora:

\begin{figure}[h!]
  \centering
  \includegraphics[width=0.92\textwidth]{images/probabilisticMethodsAdmin1.png}
  \caption{Panel administracyjny - Markov Chain Analysis.}
  \label{fig:prob_admin1}
\end{figure}

\newpage

Rysunek \ref{fig:prob_admin1} prezentuje panel ``Markov Chain Analysis'' zawierający:
\begin{itemize}
    \item \textbf{Sales Forecast}: wykres prognozy sprzedaży w czasie
    \item \textbf{Detailed Forecast}: tabela z szczegółowymi predykcjami dla poszczególnych okresów
\end{itemize}

\begin{figure}[h!]
  \centering
  \includegraphics[width=0.92\textwidth]{images/probabilisticMethodsAdmin2.png}
  \caption{Panel administracyjny - Bayesian Analysis.}
  \label{fig:prob_admin2}
\end{figure}

\newpage

Rysunek \ref{fig:prob_admin2} prezentuje panel ``Bayesian Analysis'' zawierający zaawansowane narzędzia analityczne dla administratora. Panel umożliwia analizę oczekiwanego popytu na produkty, identyfikację kategorii produktowych preferowanych przez użytkowników oraz monitorowanie metryk wydajności modelu Naive Bayes. System generuje rekomendacje dotyczące poziomu zapasów oraz identyfikuje produkty wymagające uzupełnienia magazynu na podstawie predykcji probabilistycznych.

\newpage

\section*{Rozdzia\l{} 8}
