% filepath: /SmartRecommender-Project-Django-React/.docs/latex/smola/chapter6_ui_fuzzy.tex

\subsubsection*{6.3.1 Panel debugowania Fuzzy Logic}

Panel debugowania dostępny przez endpoint \texttt{/api/fuzzy-debug/} prezentuje:

\begin{figure}[h!]
  \centering
  \includegraphics[width=0.92\textwidth]{images/fuzzyLogicAdminDebug1.png}
  \caption{Panel debugowania Fuzzy Logic.}
  \label{fig:fuzzy_debug1}
\end{figure}

\textbf{Widok ogólny}:

\begin{itemize}
    \item \textbf{Szczegóły algorytmu}: metoda (Mamdani Fuzzy Inference), liczba reguł (6), T-norma (min), T-conorma (max)
    \item \textbf{Funkcje przynależności}: definicje dla price, quality, popularity z progami
    \item \textbf{Statystyki}: średni fuzzy\_score, rozkład wyników, aktywacja reguł
    \item \textbf{Profil użytkownika}: jeśli podany user\_id — szczegóły profilu rozmytego
\end{itemize}

\textbf{Widok produktu} (z parametrem product\_id):

\begin{itemize}
    \item Wartości fuzzyfikacji (wszystkie przynależności)
    \item Aktywacja każdej z 6 reguł z wyjaśnieniem
    \item Obliczenie końcowe z breakdownem
    \item Porównanie z innymi produktami
\end{itemize}

\begin{figure}[h!]
  \centering
  \includegraphics[width=0.92\textwidth]{images/fuzzyLogicAdminDebug2.png}
  \caption{Fuzzy Logic - ewaluacja produktu.}
  \label{fig:fuzzy_debug2}
\end{figure}

\subsubsection*{6.3.2 Interfejs użytkownika - rekomendacje Fuzzy Logic}

System Fuzzy Logic jest dostępny dla użytkowników w dedykowanej zakładce panelu klienta. Interfejs składa się z trzech podzakładek: "Fuzzy Recommendations" (rekomendacje produktów), "User Profile" (profil użytkownika) oraz "Fuzzy Rules" (reguły wnioskowania). Szczegółowy opis przepływu danych w systemie Fuzzy Logic przedstawiono w rozdziale 5.2.

\newpage

\textbf{Zakładka "Fuzzy Recommendations"}

\begin{figure}[h!]
  \centering
  \includegraphics[width=0.92\textwidth]{images/fuzzyLogicClient1.png}
  \caption{Panel klienta - rekomendacje Fuzzy Logic.}
  \label{fig:fuzzy_client}
\end{figure}

Rysunek \ref{fig:fuzzy_client} przedstawia główny widok rekomendacji Fuzzy Logic. U góry strony znajdują się trzy zakładki nawigacyjne: "Fuzzy Recommendations", "User Profile" oraz "Fuzzy Rules", umożliwiające przełączanie między widokami. Sekcja "Recommended Products" wyświetla 4 produkty wybrane przez silnik wnioskowania rozmytego Mamdani. Każdy produkt zawiera:

\begin{itemize}
    \item \textbf{Fuzzy Score}: całkowity wynik rekomendacji wyrażony w procentach (np. 53.9\%, 53.7\%, 53.6\%) - wynik agregacji wszystkich 6 reguł rozmytych z uwzględnieniem ich wag (suma ważona aktywacji reguł)
    \item \textbf{Category Match}: stopień dopasowania kategorii produktu do ulubionych kategorii użytkownika wyrażony w procentach (np. 63.3\%, 62.6\%, 62.4\%)
    \item \textbf{View Rule Activations}: niebieski przycisk umożliwiający podgląd szczegółowej aktywacji wszystkich 6 reguł IF-THEN dla danego produktu wraz z wyjaśnieniem, dlaczego produkt został polecony
\end{itemize}

System prezentuje przykładowe rekomendacje z interfejsu:
\begin{itemize}
    \item Baseus Magnetic Mini Wireless Charging 20W 20000mAh 2x MagSafe (\$69.99): Fuzzy Score 53.9\%, Category Match 63.3\%
    \item TP-Link Archer T3U Plus (1300Mb/s a/b/g/n/ac) DualBand (\$24.99): Fuzzy Score 53.7\%, Category Match 62.6\%
    \item UGREEN Power Strip 65W USB/USB C + 3x AC Outlets CD268 (\$49.99): Fuzzy Score 53.6\%, Category Match 62.4\%
    \item Hama Premium Surge Protector - 4 Sockets, 1.5m (\$39.99): Fuzzy Score 53.6\%, Category Match 62.4\%
\end{itemize}

Wszystkie produkty mają podobny Fuzzy Score (~53-54\%), co oznacza, że system wnioskowania Mamdani ocenił je jako równie dopasowane do profilu użytkownika. Category Match (~62-63\%) wskazuje na wysoki stopień dopasowania kategorii produktów do historycznych preferencji zakupowych użytkownika.

\medskip

\textbf{Zakładka "User Profile"}

\begin{figure}[h!]
  \centering
  \includegraphics[width=0.92\textwidth]{images/fuzzyLogicClient2.png}
  \caption{Profil użytkownika Fuzzy Logic.}
  \label{fig:fuzzy_client2}
\end{figure}

Zakładka "User Profile" (Rysunek \ref{fig:fuzzy_client2}) prezentuje rozmyty profil użytkownika zbudowany przez klasę \texttt{FuzzyUserProfile} na podstawie historii zakupów. Profil jest wykorzystywany przez system wnioskowania Mamdani do obliczania spersonalizowanych rekomendacji. Wyświetlane informacje:

\begin{itemize}
    \item \textbf{Profile Type}: typ profilu użytkownika\\
    -- \textit{authenticated} - użytkownik zalogowany z pełną historią zakupów (jak na zrzucie ekranu)\\
    -- \textit{guest} - użytkownik bez historii zakupów (profil domyślny z globalnych statystyk)

    \item \textbf{Price Sensitivity}: wrażliwość cenowa użytkownika wyrażona w procentach\\
    -- Przykład: 60\% oznacza umiarkowaną wrażliwość cenową ("Moderate price sensitivity")\\
    -- Wartość obliczana na podstawie średniej ceny zakupionych produktów względem średniej w systemie\\
    -- Wpływa na aktywację reguł R3 (Price Sensitive Match) i R5 (Premium Match)

    \item \textbf{Favorite Categories}: lista ulubionych kategorii produktowych z procentowym udziałem w historii zakupów\\
    Przykład z interfejsu:\\
    -- \texttt{components.graphics} 13\% (karty graficzne)\\
    -- \texttt{accessories.powerBanks} 8\% (powerbanki)\\
    -- \texttt{networking.networkCards} 7\% (karty sieciowe)\\
    -- \texttt{power.strips} 6\% (listwy zasilające)\\
    -- \texttt{components.cases} 5\% (obudowy komputerowe)
\end{itemize}

System automatycznie aktualizuje profil po każdym nowym zamówieniu użytkownika, co pozwala na dynamiczną adaptację rekomendacji do zmieniających się preferencji. Kategorie są zapisywane w formacie hierarchicznym (kategoria\_główna.podkategoria) i przechowywane w polu JSON modelu \texttt{RecommendationSettings}.

\medskip

\textbf{Zakładka "Fuzzy Rules"}

\begin{figure}[h!]
  \centering
  \includegraphics[width=0.92\textwidth]{images/fuzzyLogicClient3.png}
  \caption{Reguły wnioskowania Fuzzy Logic.}
  \label{fig:fuzzy_client3}
\end{figure}

Zakładka "Fuzzy Rules" (Rysunek \ref{fig:fuzzy_client3}) prezentuje szczegółowy opis 6 reguł wnioskowania IF-THEN wykorzystywanych przez system Mamdani. Strona wyświetla nagłówek "How Recommendations Are Made" oraz wyjaśnienie: "Our system uses 6 intelligent rules to find the best products for you. Each rule looks at different aspects like price, quality, and category match." Każda reguła jest opisana w zrozumiały dla użytkownika sposób:

\begin{itemize}
    \item \textbf{Rule 1 - R1: High Quality Bargain}\\
    \textit{"Products with excellent ratings at reasonable prices get high recommendations"}\\
    Warunek: IF quality is HIGH AND (price is CHEAP OR MEDIUM)\\
    THEN recommendation degree is HIGH (weight: 0.9)\\
    Interpretacja: Produkty wysokiej jakości w atrakcyjnej cenie otrzymują najwyższą wagę rekomendacji

    \item \textbf{Rule 2 - R2: Popular in Category}\\
    \textit{"Trending products in categories the user likes"}\\
    Warunek: IF category matches user interest AND product is HIGHLY POPULAR\\
    THEN recommendation degree is MEDIUM-HIGH (weight: 0.7)\\
    Interpretacja: Popularne produkty z ulubionych kategorii użytkownika

    \item \textbf{Rule 3 - R3: Price Sensitive Match}\\
    \textit{"Budget-conscious users get cheap product recommendations boosted"}\\
    Warunek: IF user is PRICE SENSITIVE AND product is CHEAP\\
    THEN boost recommendation (weight: 0.6)\\
    Interpretacja: Użytkownicy wrażliwi cenowo otrzymują wzmocnione rekomendacje tanich produktów

    \item \textbf{Rule 4 - R4: Category Quality Match}\\
    \textit{"Good quality products in preferred categories"}\\
    Warunek: IF category matches strongly AND quality is MEDIUM-HIGH\\
    THEN recommendation degree is HIGH (weight: 0.85)\\
    Interpretacja: Wysokiej jakości produkty z preferowanych kategorii użytkownika

    \item \textbf{Rule 5 - R5: Premium Match}\\
    \textit{"Premium users get expensive high-quality products boosted"}\\
    Warunek: IF user is NOT price sensitive AND product is EXPENSIVE AND quality is HIGH\\
    THEN boost recommendation (weight: 0.8)\\
    Interpretacja: Użytkownicy premium (niewrażliwi cenowo) otrzymują wzmocnione rekomendacje drogich produktów wysokiej jakości

    \item \textbf{Rule 6 - R6: Budget-Friendly Popular}\\
    Warunek: IF product is CHEAP AND POPULAR\\
    THEN boost recommendation (weight: 0.75)\\
    Interpretacja: Tanie i popularne produkty dla użytkowników wrażliwych cenowo
\end{itemize}

System używa tych reguł do obliczenia końcowego Fuzzy Score dla każdego produktu zgodnie ze wzorem agregacji przedstawionym w rozdziale 5.2.5. Wagi reguł (0.6-0.9) określają ich wpływ na ostateczną rekomendację - reguła R1 (0.9) ma największy wpływ, podczas gdy R3 (0.6) najmniejszy. Przycisk "View Rule Activations" w zakładce "Fuzzy Recommendations" pokazuje, które reguły zostały aktywowane dla konkretnego produktu, z jaką siłą (stopień aktywacji $\alpha_i$) oraz jaki był ich wkład w końcowy wynik.

\subsubsection*{6.3.3 Wyszukiwanie rozmyte (Fuzzy Search)}

Wyszukiwanie rozmyte (Fuzzy Search) wykorzystuje algorytm odległości Levensteina do wyszukiwania produktów z tolerancją na literówki i błędy pisowni. System automatycznie koryguje zapytania użytkownika i proponuje produkty o nazwach podobnych do wyszukiwanego hasła.

\begin{figure}[h!]
  \centering
  \includegraphics[width=0.92\textwidth]{images/fuzzySearch1.jpg}
  \caption{Wyszukiwarka rozmyta (Fuzzy Search).}
  \label{fig:fuzzy_search}
\end{figure}

Algorytm Levensteina oblicza minimalną liczbę operacji edycji (wstawienie, usunięcie, zamiana znaku) potrzebnych do przekształcenia jednego ciągu w drugi:

\begin{equation}
lev(a,b) = \begin{cases}
|a| & \text{jeśli } |b| = 0 \\
|b| & \text{jeśli } |a| = 0 \\
lev(tail(a), tail(b)) & \text{jeśli } a[0] = b[0] \\
1 + \min \begin{cases}
lev(tail(a), b) \\
lev(a, tail(b)) \\
lev(tail(a), tail(b))
\end{cases} & \text{w przeciwnym wypadku}
\end{cases}
\end{equation}

System wyszukiwania rozmytego zwraca produkty, dla których odległość Levensteina między zapytaniem a nazwą produktu jest mniejsza niż ustalony próg (domyślnie 0.5).

\newpage

\section*{Rozdzia\l{} 7}
