\subsubsection*{6.3.1 Panel debugowania Fuzzy Logic}

Panel debugowania dostępny przez endpoint \texttt{/api/fuzzy-debug/} prezentuje:

\begin{figure}[h!]
  \centering
  \includegraphics[width=0.92\textwidth]{images/fuzzyLogicAdminDebug1.jpg}
  \caption{Panel debugowania Fuzzy Logic.}
  \label{fig:fuzzy_debug1}
\end{figure}

\textbf{Widok ogólny}:

\begin{itemize}
    \item \textbf{Szczegóły algorytmu}: metoda (Mamdani Fuzzy Inference), liczba reguł (6), T-norma (min), T-conorma (max)
    \item \textbf{Funkcje przynależności}: definicje dla price, quality, popularity z progami
    \item \textbf{Statystyki}: średni fuzzy\_score, rozkład wyników, aktywacja reguł
    \item \textbf{Profil użytkownika}: jeśli podany user\_id — szczegóły profilu rozmytego
\end{itemize}

\textbf{Widok produktu} (z parametrem product\_id):

\begin{itemize}
    \item Wartości fuzzyfikacji (wszystkie przynależności)
    \item Aktywacja każdej z 6 reguł z wyjaśnieniem
    \item Obliczenie końcowe z breakdownem
    \item Porównanie z innymi produktami
\end{itemize}

\begin{figure}[h!]
  \centering
  \includegraphics[width=0.92\textwidth]{images/fuzzyLogicAdminDebug2.jpg}
  \caption{Fuzzy Logic - ewaluacja produktu.}
  \label{fig:fuzzy_debug2}
\end{figure}

\textbf{Przykładowe dane z panelu debugowania}:

\begin{verbatim}
Algorithm: Fuzzy Logic Inference System (Mamdani-style)
Description: System rekomendacji oparty na logice rozmytej
z uproszczona defuzzyfikacja

User Profile:
- User: admin2 (ID: 2)
- Profile Type: authenticated
- Price Sensitivity: 0.6 - Medium
- Tracked Categories: 26

Category Interests:
- wearables.watches: 0.132
- networking.networkCards: 0.093
- peripherals.microphones: 0.066
- monitoring.cameras: 0.06
- gadgets: 0.06

Membership Functions:
Price Functions:
- CHEAP: μ = 1.0 dla ceny ≤ 100 PLN, spada do 0 przy 500 PLN
- MEDIUM: μ = 1.0 dla ceny 500-1200 PLN
- EXPENSIVE: μ = 1.0 dla ceny ≥ 2000 PLN
\end{verbatim}

\textbf{Przykład fuzzyfikacji produktu} (AMD Ryzen 9 7900X, cena: 400 PLN, rating: 3, views: 1):

\begin{verbatim}
Selected Product:
- ID: 96
- Name: AMD Ryzen 9 7900X
- Price: 400 PLN
- Rating: 3
- View Count: 1
- Categories: components.processors

Fuzzification - Membership Degrees:
Price: 400 PLN
- Cheap: μ = 0.25
- Medium: μ = 0.5
- Expensive: μ = 0
- Dominant: MEDIUM

Quality: 3
- Low: μ = 0.5
- Medium: μ = 0.5
- High: μ = 0
- Dominant: LOW

Popularity: 1 views
- Low: μ = 1
- Medium: μ = 0
- High: μ = 0
- Dominant: LOW

Category Matching:
- Max Match: 0.394
- components.processors: 0.394
\end{verbatim}

\subsubsection*{6.3.2 Interfejs użytkownika - rekomendacje Fuzzy Logic}

\begin{figure}[h!]
  \centering
  \includegraphics[width=\textwidth]{images/fuzzyLogicSequenceDiagram.png}
  \caption{Diagram sekwencji - Fuzzy Logic.}
  \label{fig:fuzzy_sequence}
\end{figure}

Rekomendacje oparte na logice rozmytej są prezentowane użytkownikowi w panelu klienta w sekcji "Recommended For You (Fuzzy Logic)". System wyświetla wykres kołowy przedstawiający rozkład kategorii w historii zakupów użytkownika oraz listę rekomendowanych produktów.

\begin{figure}[h!]
  \centering
  \includegraphics[width=0.92\textwidth]{images/fuzzyLogicClient1.jpg}
  \caption{Panel klienta - rekomendacje Fuzzy Logic.}
  \label{fig:fuzzy_client}
\end{figure}

Wykres "Category Distribution" pokazuje procentowy udział kategorii w historii zakupów użytkownika (np. electronics.phones, accessories.powerBanks, wearables.watches, peripherals.printers, office.accessories). Na podstawie tych danych system buduje rozmyty profil użytkownika i generuje spersonalizowane rekomendacje.

Sekcja "Recommended For You (Fuzzy Logic)" prezentuje produkty z najwyższym wynikiem fuzzy\_score, uwzględniając:
\begin{itemize}
    \item Dopasowanie do kategorii zainteresowań użytkownika
    \item Wrażliwość cenową użytkownika
    \item Jakość produktu (rating)
    \item Popularność produktu (view\_count)
\end{itemize}

\begin{figure}[h!]
  \centering
  \includegraphics[width=0.92\textwidth]{images/fuzzyLogicClient2.jpg}
  \caption{Zakładka "Fuzzy Recommendations".}
  \label{fig:fuzzy_client2}
\end{figure}

Zakładka "Fuzzy Recommendations" wyświetla produkty wraz z:
\begin{itemize}
    \item \textbf{Fuzzy Score}: całkowity wynik rekomendacji (np. 58.1\%, 56.3\%)
    \item \textbf{Category Match}: stopień dopasowania kategorii do preferencji użytkownika (np. 65.9\%, 60.6\%)
    \item \textbf{View Rule Activations}: przycisk do podglądu aktywacji wszystkich 6 reguł rozmytych
\end{itemize}

Przykładowe rekomendacje z interfejsu:
\begin{itemize}
    \item Motorola edge 40 neo 5G (\$549.99): Fuzzy Score 58.1\%, Category Match 65.9\%
    \item Apple iPad Air 11" M2 (\$749.99): Fuzzy Score 56.3\%, Category Match 60.6\%
    \item Roborock Q8 Max+ White (\$649.99): Fuzzy Score 56.3\%, Category Match 60.6\%
    \item JoyRoom Powerbank 10000mAh (\$49.99): Fuzzy Score 54.3\%, Category Match 64.3\%
\end{itemize}

\begin{figure}[h!]
  \centering
  \includegraphics[width=0.92\textwidth]{images/fuzzyLogicClient3.jpg}
  \caption{Zakładka "Your Fuzzy User Profile".}
  \label{fig:fuzzy_client3}
\end{figure}

Zakładka ``Your Fuzzy User Profile'' (Rysunek \ref{fig:fuzzy_client3}) prezentuje rozmyty profil użytkownika zbudowany na podstawie historii zakupów:
\begin{itemize}
    \item \textbf{Profile Type}: typ profilu (authenticated/guest)
    \item \textbf{Price Sensitivity}: wrażliwość cenowa w procentach (np. 60\%)
    \item \textbf{Favorite Categories}: ulubione kategorie z wagami zainteresowania
\end{itemize}

\subsubsection*{6.3.3 Wyszukiwanie rozmyte (Fuzzy Search)}

Wyszukiwanie rozmyte (Fuzzy Search) wykorzystuje algorytm odległości Levensteina do wyszukiwania produktów z tolerancją na literówki i błędy pisowni. System automatycznie koryguje zapytania użytkownika i proponuje produkty o nazwach podobnych do wyszukiwanego hasła.

\begin{figure}[h!]
  \centering
  \includegraphics[width=0.92\textwidth]{images/fuzzySearch1.jpg}
  \caption{Wyszukiwarka rozmyta (Fuzzy Search).}
  \label{fig:fuzzy_search}
\end{figure}

Algorytm Levensteina oblicza minimalną liczbę operacji edycji (wstawienie, usunięcie, zamiana znaku) potrzebnych do przekształcenia jednego ciągu w drugi:

\begin{equation}
lev(a,b) = \begin{cases}
|a| & \text{jeśli } |b| = 0 \\
|b| & \text{jeśli } |a| = 0 \\
lev(tail(a), tail(b)) & \text{jeśli } a[0] = b[0] \\
1 + \min \begin{cases}
lev(tail(a), b) \\
lev(a, tail(b)) \\
lev(tail(a), tail(b))
\end{cases} & \text{w przeciwnym wypadku}
\end{cases}
\end{equation}

System wyszukiwania rozmytego zwraca produkty, dla których odległość Levensteina między zapytaniem a nazwą produktu jest mniejsza niż ustalony próg (domyślnie 3).

\newpage

\section*{Rozdzia\l{} 7}
