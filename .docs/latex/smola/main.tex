% filepath: /SmartRecommender-Project-Django-React/.docs/latex/smola/main.tex

\documentclass[a4paper,12pt,twoside]{article}

\usepackage[utf8]{inputenc}
\usepackage[T1]{fontenc}
\usepackage[polish]{babel}
\usepackage{amsmath, amssymb}
\usepackage{amsthm}
\usepackage{graphicx}
\usepackage{geometry}
\usepackage{setspace}
\usepackage{indentfirst}
\usepackage{url}
\usepackage{titlesec}
\usepackage{tocloft}
\usepackage{pdfpages}
\usepackage{listings}
\usepackage{xcolor}
\usepackage{placeins}
\usepackage{float}
\usepackage{booktabs}
\usepackage{longtable}
\usepackage{array}

% Marginesy zgodnie z wytycznymi
\geometry{left=3.5cm, right=2.5cm, top=2.5cm, bottom=2.5cm}

% Numeracja stron u dołu, wyrównana do zewnętrznego marginesu
\usepackage{fancyhdr}
\pagestyle{fancy}
\fancyhf{}
\fancyfoot[LE,RO]{\thepage}
\renewcommand{\headrulewidth}{0pt}

% Interlinia 1,5
\onehalfspacing

% Wcięcia akapitów
\setlength{\parindent}{1cm}

% Zapobieganie dużym odstępom pionowym - strony nie muszą być wyrównane do dołu
\raggedbottom

% Tytuły - czcionka pogrubiona
\titleformat{\section}[block]{\bfseries\Large\raggedright}{}{1em}{}
\titleformat{\subsection}[block]{\bfseries\large\raggedright}{}{1em}{}

\lstset{
  basicstyle=\ttfamily\small,
  breaklines=true,
  frame=single,
  numbers=left,
  numberstyle=\tiny,
  keywordstyle=\color{blue},
  commentstyle=\color{gray},
  stringstyle=\color{red}
}

% Styl dla pseudokodu
\lstdefinelanguage{Pseudocode}{
  morekeywords={FUNKCJA, ZWROC, DLA, KAZDEGO, W, JEZELI, WTEDY, INACZEJ, KONIEC, LUB, I, NIE, DOPOKI, WYKONUJ, ORAZ, ZWRACA, GDZIE, OD, DO, KROK},
  sensitive=false,
  morecomment=[l]{//},
  morestring=[b]"
}

\lstdefinestyle{pseudocode}{
  language=Pseudocode,
  basicstyle=\ttfamily\small,
  keywordstyle=\bfseries\color{blue},
  commentstyle=\color{gray},
  frame=single,
  numbers=left,
  numberstyle=\tiny,
  breaklines=true,
  xleftmargin=2em,
  framexleftmargin=1.5em,
  literate={ą}{{\k{a}}}1 {ć}{{\'c}}1 {ę}{{\k{e}}}1 {ł}{{\l{}}}1 {ń}{{\'n}}1 {ó}{{\'o}}1 {ś}{{\'s}}1 {ź}{{\'z}}1 {ż}{{\.z}}1
           {Ą}{{\k{A}}}1 {Ć}{{\'C}}1 {Ę}{{\k{E}}}1 {Ł}{{\L{}}}1 {Ń}{{\'N}}1 {Ó}{{\'O}}1 {Ś}{{\'S}}1 {Ź}{{\'Z}}1 {Ż}{{\.Z}}1
}

\begin{document}

\begin{titlepage}

\begin{minipage}{0.7\textwidth}
    {\large\bf UNIWERSYTET RZESZOWSKI}\\
    {\large\bf Wydział Nauk Ścisłych i Technicznych}
\end{minipage}
\hfill
\begin{minipage}{0.25\textwidth}
    \centering
    \includegraphics[width=8em]{images/UR.png}
\end{minipage}


\vspace{3cm}

\begin{center}
    {\Large Piotr Smoła} \\
    {\large nr albumu: 125162} \\
    {\large Kierunek: Informatyka}
\end{center}

\vspace{2cm}

\begin{center}
    {\LARGE\bf System rekomendacji produktów wykorzystujący filtrację opartą na treści, logikę rozmytą i modele probabilistyczne}
\end{center}

\vspace{1.5cm}

\begin{center}
    {\large Praca inżynierska}
\end{center}

\vspace{1.5cm}

\begin{flushright}
    {\large Praca wykonana pod kierunkiem}\\
    {\large dr inż. Piotra Grochowalskiego}
\end{flushright}

\vspace{3cm}

\begin{center}
    {\large Rzesz\'ow, 2026}
\end{center}

\end{titlepage}

% Spis treści
\tableofcontents
\newpage


\section*{Wstęp}
\addcontentsline{toc}{section}{Wstęp}

Nowoczesne platformy e-commerce oferują tysiące lub dziesiątki tysięcy produktów, co stanowi istotne wyzwanie zarówno dla klientów, jak i właścicieli sklepów internetowych. Użytkownik poszukujący produktu staje przed wyborem setek opcji, co często prowadzi do rezygnacji z zakupu. Bez wsparcia inteligentnych systemów rekomendacyjnych proces zakupowy staje się czasochłonny i frustrujący. Z perspektywy biznesowej oznacza to utratę potencjalnych klientów oraz sytuacje, w których nabywcy nie odkrywają produktów optymalnie dopasowanych do ich potrzeb.

Systemy rekomendacyjne stanowią rozwiązanie tego problemu poprzez automatyczną analizę preferencji użytkowników w celu proponowania produktów dopasowanych do indywidualnych potrzeb, zwiększając jednocześnie konwersję i wartość sprzedaży \cite{ricci2015recommender}.

System został zrealizowany jako część większego projektu e-commerce powstałego we współpracy dwuosobowej, w której funkcjonalności systemu rekomendacji zostały podzielone między dwóch studentów. Piotr Smoła (niniejsza praca) odpowiadał za implementację trzech metod: filtracji opartej na treści (Content-Based Filtering), logiki rozmytej (Fuzzy Logic) oraz modeli probabilistycznych (łańcuch Markowa i naiwny klasyfikator Bayesa). Dawid Olko (praca równoległa) odpowiadał za pozostałe trzy metody: algorytm Apriori (reguły asocjacyjne), analizę sentymentu opinii użytkowników oraz collaborative filtering (filtrację kolaboratywną). Niniejsza praca koncentruje się szczegółowo na aspektach algorytmicznych i jakościowych trzech pierwszych metod, z uwzględnieniem problemu zimnego startu, interpretowalności wyników oraz personalizacji doświadczenia użytkownika.

\medskip
\noindent\textbf{Motywacja i kontekst problemu}

Problem rekomendacji w platformach e-commerce jest złożony i wieloaspektowy. Preferencje użytkowników są subiektywne i trudne do modelowania matematycznego. Dane o użytkownikach są często niekompletne lub niedostępne, zwłaszcza dla nowych użytkowników i produktów - zjawisko znane jako problem zimnego startu (cold start problem). Katalogi produktów dynamicznie się zmieniają, co wymaga ciągłej aktualizacji modeli rekomendacyjnych. Dodatkowo, użytkownicy oczekują nie tylko trafnych rekomendacji, ale również ich wyjaśnienia - dlaczego dany produkt został zaproponowany.

Istniejące rozwiązania komercyjne (Amazon Personalize, Google Recommendations AI) oferują zaawansowane mechanizmy rekomendacji, jednak działają jako czarne skrzynki (black-box), nie pozwalając na kontrolę nad algorytmami ani ich dostosowanie do specyficznych wymagań biznesowych. Biblioteki open-source (Apache Mahout, Surprise) rozwiązują problem kosztów, ale nie oferują implementacji logiki rozmytej ani zaawansowanych modeli probabilistycznych w jednym spójnym systemie.

Rzeczona praca odpowiada na powyższe wyzwania poprzez zaprojektowanie i implementację modułowego systemu rekomendacyjnego łączącego trzy komplementarne podejścia: filtrację opartą na treści (Content-Based Filtering, CBF), logikę rozmytą (Fuzzy Logic) oraz modele probabilistyczne (łańcuchy Markowa i naiwny klasyfikator Bayesa). Każda z tych metod rozwiązuje inne aspekty problemu rekomendacji i oferuje unikalne możliwości personalizacji. System został zaimplementowany od podstaw, bez wykorzystania zewnętrznych bibliotek uczenia maszynowego, co zapewnia pełną kontrolę nad algorytmami oraz możliwość ich dostosowania do specyficznych wymagań platformy e-commerce.

\medskip
\noindent\textbf{Cel i zakres pracy}

Głównym celem pracy jest zaprojektowanie i implementacja kompleksowego systemu rekomendacji produktów dla platformy e-commerce. Aplikacja została zrealizowana jako część większego projektu e-commerce powstałego we współpracy dwuosobowej, w której funkcjonalności systemu rekomendacji zostały podzielone między dwóch studentów. Niniejsza praca odpowiada za implementację trzech metod: filtracji opartej na treści (Content-Based Filtering), logiki rozmytej (Fuzzy Logic) oraz modeli probabilistycznych (łańcuch Markowa i naiwny klasyfikator Bayesa). Współautor projektu (Dawid Olko) odpowiadał za pozostałe trzy metody: algorytm Apriori (reguły asocjacyjne), analizę sentymentu opinii użytkowników oraz collaborative filtering (filtrację kolaboratywną). Łącznie system integruje sześć komplementarnych metod rekomendacyjnych, co zapewnia wszechstronne wsparcie procesu zakupowego użytkownika.

W ramach realizacji celu głównego zdefiniowano następujące cele szczegółowe:

\begin{itemize}
    \item Zaprojektowanie architektury modułowego systemu rekomendacyjnego z architekturą trójwarstwową (Django + React + PostgreSQL)
    \item Implementacja algorytmu filtracji opartej na treści (Content-Based Filtering) z wykorzystaniem ważonych wektorów cech (kategorie 40\%, tagi 30\%, cena 20\%, słowa kluczowe 10\%) i miary podobieństwa kosinusowego
    \item Opracowanie systemu wnioskowania rozmytego typu Mamdani z regułami IF-THEN i funkcjami przynależności (trójkątne i trapezoidalne) dla trzech wymiarów: wrażliwość cenowa, preferencje kategorialne, preferencja jakości
    \item Zbudowanie modeli probabilistycznych: łańcucha Markowa pierwszego rzędu dla predykcji sekwencji zakupowych oraz naiwnego klasyfikatora Bayesa dla predykcji odejścia klienta (ang. \textit{churn}) i prawdopodobieństwa zakupu
    \item Optymalizacja wydajności systemu dla wdrożenia produkcyjnego (pamięć podręczna, operacje zbiorcze, indeksy bazodanowe, Docker)
    \item Przeprowadzenie ewaluacji jakości rekomendacji na rzeczywistych danych (500 produktów, 20 użytkowników, 265 zamówień)
\end{itemize}

Zakres pracy obejmuje trzy główne obszary: podstawy teoretyczne metod rekomendacyjnych, projekt i implementację systemu oraz ewaluację jego działania. System został zintegrowany z aplikacją webową e-commerce (Django 5.1.4 + React 18 + PostgreSQL 14), umożliwiającą weryfikację zaimplementowanych algorytmów w warunkach zbliżonych do rzeczywistych.

Niniejsza praca koncentruje się szczegółowo na aspektach algorytmicznych i jakościowych trzech metod, z uwzględnieniem problemu zimnego startu, interpretowalności wyników oraz personalizacji doświadczenia użytkownika:
\begin{itemize}
    \item \textbf{Content-Based Filtering} - rozwiązuje problem zimnego startu dla nowych produktów poprzez analizę cech produktowych
    \item \textbf{Fuzzy Logic} - oferuje najwyższą interpretowalność decyzji algorytmu poprzez reguły IF-THEN
    \item \textbf{Modele probabilistyczne} - przewidują sekwencje zakupowe i ryzyko odejścia klienta na podstawie historii transakcji
\end{itemize}

\newpage

\section*{Rozdzia\l{} 1}
\addcontentsline{toc}{section}{Rozdział 1: Teoretyczne podstawy metod rekomendacyjnych}
\section*{Teoretyczne podstawy metod rekomendacyjnych}

System rekomendacji produktów został oparty na trzech zaawansowanych metodach uczenia maszynowego. Niniejszy rozdział prezentuje podstawy teoretyczne: filtracji opartej na treści (Content-Based Filtering), logiki rozmytej (Fuzzy Logic) oraz modeli probabilistycznych (Markov Chain, Naive Bayes).

\medskip
\noindent\textbf{Historia i ewolucja systemów rekomendacyjnych}

\medskip

Systemy rekomendacyjne powstały jako rozwiązanie problemu wyboru spośród dziesiątek tysięcy produktów dostępnych w sklepach internetowych. Rozwój tej dziedziny datuje się na lata 90. XX wieku, kiedy pojawiły się pierwsze platformy handlu elektronicznego. Wczesne komercyjne zastosowania systemów rekomendacji w Amazon.com (Linden et al., 2003) bazowały na algorytmie Item-Based Collaborative Filtering, który analizował wzorce zakupowe użytkowników w celu identyfikacji produktów często kupowanych razem. System rekomendacji ``Customers who bought this item also bought...'' (Klienci, którzy kupili ten produkt, kupili również...) przyczynił się do wzrostu sprzedaży o 29\% w pierwszym roku wdrożenia, demonstrując realne korzyści biznesowe personalizacji. Kluczowym elementem było wykorzystanie macierzy podobieństw między produktami obliczanej w trybie offline (przetwarzanie wsadowe wykonywane periodycznie), co pozwalało na generowanie rekomendacji w czasie rzeczywistym nawet dla katalogów liczących miliony produktów. Amazon zastosował metrykę Adjusted Cosine Similarity (skorygowane podobieństwo kosinusowe) do obliczania podobieństwa produktów na podstawie wektorów ocen użytkowników, normalizując różnice w sposobie oceniania przez poszczególnych użytkowników.

Istotnym katalizatorem rozwoju zaawansowanych technik rekomendacji były konkursy badawcze, szczególnie Netflix Prize organizowany w latach 2006-2009. Konkurs ten, z pulą nagród wynoszącą milion dolarów, przyciągnął uwagę środowiska akademickiego oraz przemysłowego, znacząco przyspieszając rozwój algorytmów faktoryzacji macierzy, zespołów modeli oraz metod deep learning. Współcześnie systemy rekomendacyjne stanowią fundament funkcjonowania wiodących platform e-commerce (handlu elektronicznego), serwisów VOD (Video on Demand - wideo na żądanie), platform muzycznych oraz mediów społecznościowych.

Ewolucja systemów rekomendacyjnych przebiegała od prostych metod statystycznych (najbardziej popularne produkty, najlepiej ocenione) przez collaborative filtering (analiza wzorców zakupowych użytkowników), content-based filtering (analiza cech produktów), aż po zaawansowane metody hybrydowe łączące multiple podejścia. Obecne trendy obejmują wykorzystanie deep learning (sieci neuronowe), contextual bandits (algorytmy balansujące eksplorację i eksploatację), explainable AI (interpretowalność decyzji algorytmicznych) oraz personalizację w czasie rzeczywistym.

\medskip
\noindent\textbf{Content-Based Filtering -- podstawy teoretyczne}

\medskip

Content-Based Filtering (CBF, filtracja oparta na treści) jest jedną z fundamentalnych metod systemów rekomendacyjnych. W przeciwieństwie do Collaborative Filtering, CBF analizuje cechy samych produktów, a nie wzorce zachowań użytkowników. Metoda została szczegółowo opisana w literaturze \cite{pazzani2007content}.

\textbf{Zasada działania}: System buduje profil cech każdego produktu (wektor cech) i oblicza podobieństwo między produktami na podstawie ich cech. Użytkownikowi rekomendowane są produkty podobne do tych, które wcześniej przeglądał lub kupił.

\textbf{Reprezentacja produktu jako wektora cech}

Każdy produkt $p$ jest reprezentowany jako wektor w wielowymiarowej przestrzeni cech:

\begin{equation}
\vec{p} = (f_1, f_2, ..., f_n)
\end{equation}

gdzie $f_i$ to waga cechy $i$ (np. należenie do kategorii, posiadanie tagu, przedział cenowy). W ogólnym przypadku stosuje się wagi różnicujące znaczenie poszczególnych cech:

\begin{equation}
\vec{p} = \sum_{i} w_i \cdot f_i(p)
\end{equation}

gdzie $w_i$ to waga cechy $i$, a $f_i(p)$ to wartość cechy dla produktu $p$. Funkcja indykatorowa $\mathbf{1}_{feature}(p)$ przyjmuje wartość 1 jeśli produkt posiada daną cechę, 0 w przeciwnym razie.

\medskip

\textbf{Zalety CBF}:
\begin{itemize}
    \item Brak problemu zimnego startu dla nowych produktów -- wystarczy opis i cechy
    \item Przejrzystość rekomendacji -- można wyjaśnić dlaczego produkt został polecony ("podobna kategoria", "podobne tagi")
    \item Niezależność od innych użytkowników -- działa nawet dla pierwszego klienta w systemie
    \item Szybka aktualizacja -- dodanie nowego produktu nie wymaga przeliczenia całej macierzy
\end{itemize}

\newpage

\textbf{Wady CBF}:
\begin{itemize}
    \item Problem bańki filtrującej (ang. \textit{filter bubble})~-- rekomenduje tylko podobne produkty, użytkownik nie odkrywa nowych kategorii
    \item Wymaga dobrze opisanych cech produktów -- jakość rekomendacji zależy od jakości metadanych
    \item Nie odkrywa nieoczywistych powiązań między produktami (np. "użytkownicy kupujący kawę często kupują cukier")
    \item Ograniczenie do podobieństwa cech -- nie uwzględnia kontekstu użytkownika
\end{itemize}

\textbf{Uzasadnienie wyboru podobieństwa kosinusowego i TF-IDF:} W metodzie Content-Based Filtering produkty są reprezentowane jako wektory w wielowymiarowej przestrzeni cech, gdzie każdy wymiar odpowiada jednej cesze (kategoria, tag, cena). Do porównania takich wektorowych reprezentacji produktów wybrano podobieństwo kosinusowe (ang. \textit{Cosine Similarity}) ze względu na dwie kluczowe właściwości: (1) mierzy kąt między wektorami, a nie ich długość, co czyni je odporne na różnice w liczbie cech między produktami (laptop z 20 tagami vs laptop z 5 tagami), (2) normalizuje wynik do zakresu $[0, 1]$, co umożliwia intuicyjną interpretację jako procent podobieństwa. Schemat ważenia TF-IDF (ang. \textit{Term Frequency - Inverse Document Frequency}) został wybrany do reprezentacji tekstowych cech (tagi, słowa kluczowe), ponieważ automatycznie przypisuje niższą wagę cechom powszechnym (np. tag "elektronika" występujący w 80\% produktów) i wyższą wagę cechom rzadkim, wyróżniającym (np. tag "gaming 4K" występujący tylko w konkretnych laptopach), co zwiększa precyzję rekomendacji. Oba te wybory są standardem w systemach rekomendacyjnych opartych na treści \cite{salton1989automatic}.

\textbf{Podobieństwo kosinusowe} -- dla dwóch wektorów $\vec{A}$ i $\vec{B}$:

\begin{equation}
\text{cos}(\theta) = \frac{\vec{A} \cdot \vec{B}}{||\vec{A}|| \times ||\vec{B}||} = \frac{\sum_{i=1}^{n} A_i B_i}{\sqrt{\sum_{i=1}^{n} A_i^2} \times \sqrt{\sum_{i=1}^{n} B_i^2}}
\end{equation}

gdzie $\vec{A}$ i $\vec{B}$ to wektory cech dwóch produktów. Wynik mieści się w przedziale $[0, 1]$ dla nieujemnych wektorów (w kontekście TF-IDF i wag binarnych).

\medskip

\textbf{Interpretacja podobieństwa kosinusowego}:
\begin{itemize}
    \item $cos(\theta) = 1$ -- wektory identyczne (produkty mają te same cechy)
    \item $cos(\theta) = 0$ -- wektory ortogonalne (brak wspólnych cech)
    \item $cos(\theta) \in (0, 1)$ -- częściowe podobieństwo
\end{itemize}

\textbf{TF-IDF (Term Frequency - Inverse Document Frequency)}

W kontekście ekstrakcji słów kluczowych z opisów tekstowych stosuje się miarę TF-IDF (Term Frequency - Inverse Document Frequency) \cite{salton1989automatic}:

\begin{equation}
TF(t, d) = \frac{count(t, d)}{|d|}
\end{equation}

gdzie $count(t, d)$ to liczba wystąpień terminu $t$ w dokumencie $d$, a $|d|$ to długość dokumentu (liczba słów).

Pełna wersja TF-IDF uwzględnia rzadkość terminu w całym korpusie:

\begin{equation}
TF\text{-}IDF(t, d, D) = TF(t, d) \times IDF(t, D)
\end{equation}

gdzie $IDF(t, D) = \log\frac{|D|}{|\{d \in D : t \in d\}|}$ to odwrócona częstość dokumentowa, penalizująca terminy występujące w wielu dokumentach.

\medskip
\noindent\textbf{Logika rozmyta -- podstawy teoretyczne}

\medskip

Logika rozmyta (Fuzzy Logic) została wprowadzona przez Lotfi Zadeha w przełomowej pracy \cite{zadeh1965fuzzy}. Rozszerza klasyczną logikę dwuwartościową (prawda/fałsz) o stopnie przynależności w przedziale $[0, 1]$.

\textbf{Motywacja}: Klasyczna logika wymaga precyzyjnych granic. Pytanie "Czy produkt za 450 PLN jest tani?"~nie ma jednoznacznej odpowiedzi -- zależy od kontekstu, kategorii produktu i preferencji użytkownika. Logika rozmyta pozwala odpowiedzieć: "Produkt jest tani ze stopniem 0.3 i średnio drogi ze stopniem 0.7".

\textbf{Zbiory rozmyte} (Fuzzy Sets): W klasycznej teorii zbiorów element należy lub nie należy do zbioru. W zbiorach rozmytych element ma stopień przynależności $\mu(x) \in [0, 1]$. Formalnie, zbiór rozmyty $A$ na uniwersum $X$ jest zdefiniowany przez funkcję przynależności:

\begin{equation}
\mu_A : X \rightarrow [0, 1]
\end{equation}

gdzie $\mu_A(x)$ oznacza stopień przynależności elementu $x$ do zbioru $A$.

\textbf{Przykład}: Dla zmiennej "cena"~możemy zdefiniować trzy zbiory rozmyte:
\begin{itemize}
    \item \textbf{cheap}: ceny niskie (pełna przynależność dla cen < 100 PLN)
    \item \textbf{medium}: ceny średnie (pełna przynależność dla cen 500-1200 PLN)
    \item \textbf{expensive}: ceny wysokie (pełna przynależność dla cen > 2000 PLN)
\end{itemize}

Produkt za 350 PLN może mieć: $\mu_{cheap}(350) = 0.3$, $\mu_{medium}(350) = 0.5$, $\mu_{expensive}(350) = 0.0$.

\textbf{Funkcje przynależności} (Membership Functions) definiują stopień przynależności elementu do zbioru rozmytego. Najczęściej stosowane typy:

\textit{Funkcja trójkątna} (Triangular MF):
\begin{equation}
\mu_{triangle}(x; a, b, c) = \max\left(0, \min\left(\frac{x-a}{b-a}, \frac{c-x}{c-b}\right)\right)
\end{equation}

gdzie $a$ to dolna granica, $b$ to punkt maksymalny ($\mu=1$), $c$ to górna granica.

\textit{Funkcja trapezoidalna} (Trapezoidal MF):
\begin{equation}
\mu_{trapezoid}(x; a, b, c, d) = \max\left(0, \min\left(\frac{x-a}{b-a}, 1, \frac{d-x}{d-c}\right)\right)
\end{equation}

gdzie przedział $[b, c]$ ma pełną przynależność ($\mu=1$), a $[a, b)$ i $(c, d]$ to obszary przejściowe.

\textit{Funkcja gaussowska} (Gaussian MF):
\begin{equation}
\mu_{gaussian}(x; c, \sigma) = e^{-\frac{(x-c)^2}{2\sigma^2}}
\end{equation}

gdzie $c$ to środek (mean), a $\sigma$ to odchylenie standardowe kontrolujące szerokość.

\textbf{Operacje na zbiorach rozmytych}

\textit{Uzupełnienie} (Negacja):
\begin{equation}
\mu_{\bar{A}}(x) = 1 - \mu_A(x)
\end{equation}

\textit{Przecięcie} (AND) -- T-norma:
\begin{equation}
\mu_{A \cap B}(x) = T(\mu_A(x), \mu_B(x))
\end{equation}

Najczęściej używane T-normy:
\begin{itemize}
    \item Minimum (Gödel): $T_{min}(a, b) = \min(a, b)$
    \item Iloczyn algebraiczny: $T_{prod}(a, b) = a \cdot b$
    \item Łukasiewicz: $T_L(a, b) = \max(0, a + b - 1)$
\end{itemize}

\textit{Suma} (OR) -- T-conorma (S-norma):
\begin{equation}
\mu_{A \cup B}(x) = S(\mu_A(x), \mu_B(x))
\end{equation}

Najczęściej używane T-conormy:
\begin{itemize}
    \item Maksimum: $S_{max}(a, b) = \max(a, b)$
    \item Suma algebraiczna: $S_{sum}(a, b) = a + b - a \cdot b$
    \item Łukasiewicz: $S_L(a, b) = \min(1, a + b)$
\end{itemize}

\textbf{System wnioskowania Mamdani} \cite{mamdani1975experiment} jest najbardziej rozpowszechnioną metodą wnioskowania rozmytego. Składa się z czterech etapów:

\begin{enumerate}
    \item \textbf{Fuzzyfikacja} -- przekształcenie wartości wejściowych na stopnie przynależności do zbiorów rozmytych. Przykład: cena 450 PLN → $\mu_{cheap}=0.1$, $\mu_{medium}=0.6$, $\mu_{expensive}=0.0$.

    \item \textbf{Ewaluacja reguł} -- obliczenie aktywacji reguł IF-THEN za pomocą T-norm. Dla reguły "IF price IS cheap AND quality IS high THEN recommendation IS strong":
    \begin{equation}
    \alpha = T(\mu_{cheap}(price), \mu_{high}(quality)) = \min(\mu_{cheap}, \mu_{high})
    \end{equation}

    \item \textbf{Agregacja} -- połączenie wyników wszystkich reguł za pomocą T-conormy. Jeśli wiele reguł prowadzi do tego samego wniosku (następnika):
    \begin{equation}
    \mu_{output} = S(\alpha_1, \alpha_2, ..., \alpha_n) = \max(\alpha_1, \alpha_2, ..., \alpha_n)
    \end{equation}

    \item \textbf{Defuzzyfikacja} -- przekształcenie wyniku rozmytego na wartość liczbową.
\end{enumerate}

\textbf{Metody defuzzyfikacji}:

\textit{Centroid} (środek ciężkości):
\begin{equation}
y^* = \frac{\int y \cdot \mu(y) dy}{\int \mu(y) dy}
\end{equation}

\textit{Średnia ważona} (Weighted Average) -- uproszczona metoda używana w implementacji:
\begin{equation}
y^* = \frac{\sum_{i=1}^{n} \alpha_i \cdot w_i}{\sum_{i=1}^{n} w_i}
\end{equation}

gdzie $\alpha_i$ to aktywacja reguły $i$, a $w_i$ to waga reguły.

\textit{Mean of Maximum} (MoM):
\begin{equation}
y^* = \frac{1}{|M|} \sum_{y \in M} y, \quad M = \{y : \mu(y) = \max_z \mu(z)\}
\end{equation}

\textbf{Reguły rozmyte IF-THEN}

Reguła rozmyta ma postać \cite{ross2010fuzzy}:
\begin{equation}
\text{IF } x_1 \text{ IS } A_1 \text{ AND } x_2 \text{ IS } A_2 \text{ THEN } y \text{ IS } B
\end{equation}

gdzie $A_1$, $A_2$, $B$ to zbiory rozmyte definiujące poprzedniki (warunki) i następnik (wniosek) reguły. Operator AND realizowany jest przez T-normę, najczęściej minimum lub iloczyn algebraiczny.

\medskip
\noindent\textbf{Modele probabilistyczne -- podstawy teoretyczne}

\medskip

\textbf{Łańcuchy Markowa} (Markov Chains) zostały wprowadzone przez Andrieja Markowa w 1906 roku. Są procesami stochastycznymi spełniającymi własność Markowa -- przyszły stan zależy tylko od stanu obecnego, nie od historii \cite{rabiner1989tutorial}.

\textit{Definicja formalna}: Łańcuch Markowa to ciąg zmiennych losowych $X_0, X_1, X_2, ...$ przyjmujących wartości ze zbioru stanów $S = \{s_1, s_2, ..., s_n\}$, spełniający własność Markowa:

\begin{equation}
P(X_{t+1} = s_{j} | X_t = s_i, X_{t-1} = s_{i-1}, ..., X_0 = s_0) = P(X_{t+1} = s_j | X_t = s_i)
\end{equation}

Oznacza to, że prawdopodobieństwo przejścia do stanu $s_j$ zależy tylko od obecnego stanu $s_i$, nie od tego jak do niego dotarliśmy.

\textit{Macierz przejść} (Transition Matrix) $P$ zawiera prawdopodobieństwa przejść między stanami:

\begin{equation}
P_{ij} = P(X_{t+1} = s_j | X_t = s_i)
\end{equation}

Macierz $P$ spełnia warunki:
\begin{itemize}
    \item $P_{ij} \geq 0$ dla wszystkich $i, j$
    \item $\sum_j P_{ij} = 1$ dla wszystkich $i$ (wiersze sumują się do 1)
\end{itemize}

\textit{Estymacja prawdopodobieństw przejść} z danych:
\begin{equation}
\hat{P}_{ij} = \frac{count(s_i \rightarrow s_j)}{\sum_k count(s_i \rightarrow s_k)}
\end{equation}

gdzie $count(s_i \rightarrow s_j)$ to liczba obserwowanych przejść ze stanu $s_i$ do stanu $s_j$.

\textit{Rozkład stacjonarny} (Stationary Distribution) $\pi$ spełnia:
\begin{equation}
\pi = \pi P, \quad \sum_i \pi_i = 1
\end{equation}

Jest to rozkład prawdopodobieństwa, który pozostaje niezmieniony po przejściu -- reprezentuje długoterminowe prawdopodobieństwa przebywania w każdym stanie. Rozkład stacjonarny jest istotny w analizie długoterminowego zachowania systemu.

\textbf{Naiwny klasyfikator Bayesa} (Naive Bayes, NB) opiera się na twierdzeniu Bayesa z założeniem niezależności cech \cite{murphy2012machine}.

\textit{Twierdzenie Bayesa}:
\begin{equation}
P(C | X) = \frac{P(X | C) \cdot P(C)}{P(X)}
\end{equation}

gdzie:
\begin{itemize}
    \item $P(C | X)$ -- prawdopodobieństwo a posteriori klasy $C$ przy cechach $X$
    \item $P(C)$ -- prawdopodobieństwo a priori klasy $C$
    \item $P(X | C)$ -- wiarygodność (likelihood) -- prawdopodobieństwo obserwacji cech $X$ w klasie $C$
    \item $P(X)$ -- prawdopodobieństwo marginalne cech (stałe dla wszystkich klas)
\end{itemize}

\textit{Założenie naiwne} (Naive assumption) -- niezależność warunkowa cech:
\begin{equation}
P(X | C) = P(x_1, x_2, ..., x_n | C) = \prod_{i=1}^{n} P(x_i | C)
\end{equation}

Założenie to jest "naiwne"~bo w rzeczywistości cechy są często skorelowane. Jednak Naive Bayes działa zaskakująco dobrze w praktyce.

\textit{Klasyfikacja}:
\begin{equation}
\hat{C} = \arg\max_C P(C) \prod_{i=1}^{n} P(x_i | C)
\end{equation}

Ponieważ $P(X)$ jest stałe dla wszystkich klas, można je pominąć przy porównywaniu.

\textit{Problem zerowych prawdopodobieństw}: Jeśli cecha $x_i$ nie wystąpiła w klasie $C$ w danych treningowych, to $P(x_i | C) = 0$, co zeruje całe prawdopodobieństwo.

\textbf{Wygładzanie Laplace'a} (Laplace Smoothing / Add-one Smoothing) rozwiązuje ten problem:
\begin{equation}
P(x_i = v | C) = \frac{count(x_i = v, C) + 1}{count(C) + |V_i|}
\end{equation}

gdzie $|V_i|$ to liczba unikalnych wartości cechy $x_i$. Dodanie 1 do licznika i $|V|$ do mianownika zapewnia, że żadne prawdopodobieństwo nie będzie zerowe.

\textit{Logarytm dla stabilności numerycznej}: Iloczyn wielu małych prawdopodobieństw prowadzi do niedomiaru arytmetycznego (ang. \textit{underflow} - sytuacja, gdy wartość liczbowa jest zbyt mała aby można było ją reprezentować w pamięci komputera, co powoduje zaokrąglenie do zera). Rozwiązanie -- praca w przestrzeni logarytmów:
\begin{equation}
\log P(C | X) = \log P(C) + \sum_{i=1}^{n} \log P(x_i | C) + const
\end{equation}

\textit{Warianty Naive Bayes}:
\begin{itemize}
    \item \textbf{Multinomial NB} -- dla danych wyliczeniowych (np. częstość słów)
    \item \textbf{Bernoulli NB} -- dla cech binarnych (obecność/brak)
    \item \textbf{Gaussian NB} -- dla cech ciągłych (zakłada rozkład normalny)
\end{itemize}

Naive Bayes jest szeroko stosowany w klasyfikacji tekstu, filtracji spamu oraz systemach rekomendacyjnych ze względu na prostotę implementacji i niskie wymagania obliczeniowe.

\newpage

\medskip
\noindent\textbf{Metryki oceny systemów rekomendacyjnych}

\medskip

Ewaluacja systemów rekomendacyjnych wymaga odpowiednich metryk jakości. Najpopularniejsze:

\textbf{Precision@K} -- jaka część top K rekomendacji była faktycznie kupiona/polubiona:
\begin{equation}
Precision@K = \frac{|Recommended@K \cap Relevant|}{K}
\end{equation}

\textbf{Recall@K} -- jaka część produktów istotnych dla użytkownika została trafiona:
\begin{equation}
Recall@K = \frac{|Recommended@K \cap Relevant|}{|Relevant|}
\end{equation}

\textbf{F1-Score} -- harmoniczna średnia Precision i Recall:
\begin{equation}
F1@K = 2 \cdot \frac{Precision@K \cdot Recall@K}{Precision@K + Recall@K}
\end{equation}

\textbf{Mean Reciprocal Rank (MRR)} -- pozycja pierwszego trafienia:
\begin{equation}
MRR = \frac{1}{|U|} \sum_{u \in U} \frac{1}{rank_u}
\end{equation}

gdzie $rank_u$ to pozycja pierwszego istotnego produktu w rankingu dla użytkownika $u$.

\textbf{Coverage} -- procent produktów, które system jest w stanie rekomendować:
\begin{equation}
Coverage = \frac{|\text{products with recommendations}|}{|\text{all products}|}
\end{equation}

\newpage

\section*{Rozdzia\l{} 2}
\addcontentsline{toc}{section}{Rozdział 2: Weryfikacja i analiza rozwiązań alternatywnych}
\section*{Weryfikacja i analiza rozwiązań alternatywnych}

W celu uzasadnienia sensowności tworzenia dedykowanego systemu rekomendacji przeprowadzono analizę trzech reprezentatywnych rozwiązań rynkowych. Celem weryfikacji było zidentyfikowanie ograniczeń istniejących narzędzi oraz określenie wymagań dla planowanej aplikacji e-commerce.

\medskip
\noindent\textbf{Amazon Personalize}

\medskip

Amazon Personalize to zarządzana usługa AWS oferująca systemy rekomendacji oparte na algorytmach stosowanych w Amazon.com. System wykorzystuje głębokie uczenie (ang. \textit{deep learning} - wielowarstwowe sieci neuronowe) oraz filtrację kolaboratywną (ang. \textit{collaborative filtering}), oferując trzy typy rekomendacji: personalizację użytkownika (ang. \textit{User Personalization}), podobne produkty (ang. \textit{Similar Items}) oraz spersonalizowane rankowanie (ang. \textit{Personalized Ranking}).

\textbf{Możliwości i funkcjonalności:}

\begin{itemize}
\item \textbf{User Personalization} - algorytm HRNN (Hierarchical Recurrent Neural Network - hierarchiczna rekurencyjna sieć neuronowa) generuje spersonalizowane rankingi produktów na podstawie pełnej historii interakcji użytkownika (przeglądanie, dodawanie do koszyka, zakupy). Model automatycznie wykrywa długoterminowe preferencje oraz krótkoterminowe intencje zakupowe, adaptując rekomendacje w czasie rzeczywistym,

\item \textbf{Similar Items} - rekomendacje produktów podobnych wykorzystujące algorytm SIMS (Similar Items), który analizuje wzorce współwystępowania produktów w sesjach użytkowników. Efektywnie rozwiązuje problem zimnego startu dla nowych użytkowników, bazując wyłącznie na podobieństwie produktów bez konieczności znajomości profilu użytkownika,

\item \textbf{Personalized Ranking} - algorytm re-rankingowy przyjmujący listę kandydatów i sortujący je według przewidywanej trafności dla konkretnego użytkownika. Wykorzystywany do personalizacji wyników wyszukiwania, stron kategorii oraz list promocyjnych,

\item \textbf{Automatyczne dostrajanie hiperparametrów} - system automatycznie dobiera hiperparametry modeli (learning rate, liczba warstw ukrytych, wymiarowość embeddingów) oraz skaluje infrastrukturę w zależności od obciążenia,

\item \textbf{Integracja z ekosystemem AWS} - natywne wsparcie dla AWS Lambda, S3, EventBridge oraz Kinesis umożliwia budowanie rozwiązań czasu rzeczywistego z latencją poniżej 100ms dla 99. percentyla zapytań.
\end{itemize}

\textbf{Kluczowe ograniczenia w kontekście planowanego rozwiązania:}

\begin{itemize}
\item \textbf{Wysokie koszty operacyjne} - rozwiązania chmurowe wiążą się z regularnymi opłatami licencyjnymi, które mogą być znaczące dla małych i średnich platform e-commerce. Dla porównania, własna implementacja eliminuje te koszty przy zachowaniu kontroli nad funkcjonalnościami,
\item \textbf{Brak natywnej analizy sentymentu, logiki rozmytej oraz modeli probabilistycznych} - Amazon Personalize koncentruje się wyłącznie na filtracji kolaboratywnej i nie oferuje analizy opinii produktów, systemów rozmytych ani łańcuchów Markova. Wieloźródłowa agregacja sentymentu, logika rozmyta Mamdani oraz modele probabilistyczne wymagają integracji z dodatkowymi usługami AWS lub samodzielnej implementacji,
\item \textbf{Vendor lock-in} (uzależnienie od dostawcy) - głęboka integracja z ekosystemem AWS (S3, Lambda, EventBridge) oznacza, że migracja do innej platformy wymaga przepisania całej architektury systemu,
\item \textbf{Brak kontroli nad algorytmami} - system działa jako czarna skrzynka (ang. \textit{black box}), uniemożliwiając dostosowanie logiki rekomendacji. Przykład: niemożliwe jest zaimplementowanie ważonych wektorów cech dla filtracji opartej na treści (ang. \textit{Content-Based Filtering}), funkcji przynależności dla logiki rozmytej czy zaawansowanych metryk podobieństwa,
\item \textbf{Wymóg dużych zbiorów danych} - według dokumentacji AWS, system wymaga minimum 25000 interakcji dla zapewnienia wysokiej jakości rekomendacji. Dla nowych platform problem zimnego startu (ang. \textit{cold start}) oznacza ograniczoną jakość w początkowym okresie działania.
\end{itemize}

\medskip
\noindent\textbf{Google Recommendations AI (Vertex AI)}

\medskip

Google Recommendations AI to platforma GCP wykorzystująca deep learning oraz multi-armed contextual bandits (wieloramienne bandyty kontekstowe - algorytmy balansujące eksplorację nowych opcji z wykorzystaniem sprawdzonych rozwiązań). System oferuje zaawansowane rekomendacje dla e-commerce, VOD (Video on Demand - wideo na żądanie) oraz platform newsowych, z automatycznym wykrywaniem trendów i sezonowości.

\textbf{Możliwości i funkcjonalności:}

\begin{itemize}
\item \textbf{Wieloramienne bandyty kontekstowe (Multi-Armed Contextual Bandits)} - algorytmy dynamicznie równoważące eksplorację nowych rekomendacji (w celu zbierania informacji o preferencjach użytkowników) z eksploatacją sprawdzonych rozwiązań (maksymalizujących natychmiastową konwersję). System w każdej iteracji podejmuje decyzję czy pokazać użytkownikowi produkt o znanej wysokiej skuteczności (eksploatacja) czy też produkt niedostatecznie przetestowany, ale potencjalnie bardziej dopasowany (eksploracja). Podejście to zapobiega utknięciu systemu w lokalnym optimum rekomendacji,

\item \textbf{Automatyczna detekcja trendów i sezonowości} - modele czasowe (temporal models) wykrywają wzorce zakupowe związane z porą roku, dniami tygodnia oraz wydarzeniami specjalnymi (Black Friday, święta), automatycznie dostosowując wagi rekomendacji,

\item \textbf{Frequently Bought Together} - funkcja analizująca koszyki zakupowe i identyfikująca produkty często kupowane razem, podobna do algorytmu Apriori dla reguł asocjacyjnych,

\item \textbf{Optymalizacja dla różnych celów biznesowych} - system pozwala definiować cele optymalizacji: maksymalizacja współczynnika klikalności (CTR), wartości koszyka (AOV - Average Order Value), retention (utrzymanie użytkowników) lub revenue (przychód),

\item \textbf{Integracja z Google Analytics 4} - natywne wsparcie dla śledzenia zdarzeń e-commerce (product\_view, add\_to\_cart, purchase) umożliwia szybkie wdrożenie bez konieczności budowania własnej infrastruktury analitycznej,

\item \textbf{Obsługa zimnego startu} - algorytmy wykorzystujące metadata produktów (kategoria, marka, cena) oraz sygnały kontekstowe (lokalizacja, urządzenie, pora dnia) zapewniają rekomendacje nawet dla nowych użytkowników bez historii interakcji.
\end{itemize}

\textbf{Kluczowe ograniczenia w kontekście planowanego rozwiązania:}

\begin{itemize}
\item \textbf{Bardzo wysokie koszty operacyjne} - model cenowy oparty na liczbie predykcji (pay-per-prediction) generuje rosnące koszty wraz ze skalą ruchu użytkowników, co może stanowić barierę dla małych i średnich platform e-commerce o ograniczonym budżecie,
\item \textbf{Brak wieloźródłowej agregacji sentymentu, logiki rozmytej i modeli probabilistycznych} - Google Recommendations AI oferuje funkcję często kupowanych razem produktów (ang. \textit{Frequently Bought Together}, podobną do algorytmu Apriori), ale nie wspiera agregacji sentymentu z wielu źródeł tekstowych, systemów rozmytych z funkcjami przynależności ani łańcuchów Markova jak w planowanym systemie,
\item \textbf{Wymóg bardzo dużych zbiorów danych} - rozwiązanie zaprojektowane dla platform o skali YouTube, co czyni je nadmiarowo złożonym dla małych sklepów internetowych,
\item \textbf{Brak interpretowalności} - głęboka czarna skrzynka, gdzie nawet administratorzy z dostępem do Vertex AI nie mogą zobaczyć wag reprezentacji wektorowych (ang. \textit{embeddings}) ani logiki sieci neuronowej, co uniemożliwia debugowanie i optymalizację. W przeciwieństwie do tego, logika rozmyta oferuje pełną przejrzystość reguł IF-THEN.
\end{itemize}

\medskip
\noindent\textbf{Apache Mahout}

\medskip

Apache Mahout to otwartoźródłowy framework (ang. \textit{open-source framework}) implementujący klasyczne algorytmy filtracji kolaboratywnej oraz faktoryzacji macierzy (ang. \textit{matrix factorization} - technika dekompozycji macierzy użytkownik-produkt) - ALS (ang. \textit{Alternating Least Squares} - metoda najmniejszych kwadratów na przemian), SVD (ang. \textit{Singular Value Decomposition} - rozkład według wartości osobliwych). Projekt powstał w 2008 roku, obecnie koncentruje się na algorytmach rozproszonych opartych na Apache Spark (ang. \textit{Spark-based distributed algorithms}).

\textbf{Możliwości i funkcjonalności:}

\begin{itemize}
\item \textbf{Licencja open-source Apache 2.0} - kod źródłowy dostępny bez opłat licencyjnych, możliwość pełnej modyfikacji i dostosowania do specyficznych wymagań biznesowych bez ograniczeń vendor lock-in,

\item \textbf{Collaborative Filtering z ALS (Alternating Least Squares)} - algorytm faktoryzacji macierzy user-item rozkładający macierz interakcji na dwie macierze niższego wymiaru (user factors i item factors). ALS iteracyjnie optymalizuje reprezentacje wektorowe użytkowników i produktów, minimalizując błąd predykcji dla znanych interakcji. Skaluje się efektywnie dla dużych zbiorów danych dzięki możliwości równoległego przetwarzania,

\item \textbf{SVD i SVD++ (Singular Value Decomposition)} - techniki rozkładu macierzy identyfikujące ukryte czynniki (latent factors) wpływające na preferencje użytkowników. SVD++ rozszerza klasyczne SVD o implicit feedback (sygnały niejawne: przeglądanie produktów, czas spędzony na stronie), zwiększając dokładność predykcji,

\item \textbf{Przetwarzanie rozproszone na Apache Spark} - architektura oparta na Spark umożliwia skalowanie dla katalogów produktów liczących miliony pozycji oraz baz użytkowników rzędu dziesiątek milionów. Distributed computing (przetwarzanie rozproszone) skraca czas trenowania modeli z godzin do minut,

\item \textbf{Integracja z ekosystemem Hadoop} - natywne wsparcie dla HDFS (Hadoop Distributed File System), Hive oraz HBase zapewnia spójność z istniejącą infrastrukturą big data w przedsiębiorstwach,

\item \textbf{Modele probabilistyczne} - implementacje Naive Bayes, Logistic Regression oraz Random Forest dla zadań klasyfikacji i predykcji, możliwe do wykorzystania w scenariuszach uzupełniających collaborative filtering.
\end{itemize}

\textbf{Kluczowe ograniczenia w kontekście planowanego rozwiązania:}

\begin{itemize}
\item \textbf{Wymóg zaawansowanej wiedzy technicznej} - konieczność konfiguracji klastra Apache Spark (środowisko przetwarzania rozproszonego), YARN resource manager (zarządca zasobów) oraz monitoringu. Według Stack Overflow Developer Survey 2023, bardzo mała część programistów ma doświadczenie z Apache Spark,
\item \textbf{Koszty infrastruktury} - chociaż licencja Apache 2.0 jest darmowa, utrzymanie klastra Spark wymaga dedykowanych zasobów serwerowych oraz czasu na implementację integracji (REST API, baza danych, cache, frontend), co generuje znaczące koszty operacyjne,
\item \textbf{Brak analizy sentymentu, logiki rozmytej i modeli probabilistycznych} - Apache Mahout nie oferuje sentiment analysis, fuzzy logic ani Markov Chains. Wymagana jest integracja z zewnętrznymi bibliotekami (np. Stanford CoreNLP dla sentymentu) lub samodzielna implementacja słownikowej analizy sentymentu, systemu Mamdani dla logiki rozmytej oraz łańcucha Markowa,
\item \textbf{Wolniejszy rozwój projektu} - aktywność projektu spadła w ostatnich latach (2-3 commity miesięcznie w 2023-2024 vs 20-30 commitów w latach 2012-2014), co skutkuje ograniczoną dokumentacją dla nowszych wersji.
\end{itemize}

\textbf{Alternatywy open-source}: Biblioteka Surprise (Python) oferuje implementacje SVD, SVD++, NMF, KNN z built-in dataset loaders i cross-validation, ale również nie wspiera Content-Based Filtering, Fuzzy Logic, Sentiment Analysis ani Association Rules.

\medskip
\noindent\textbf{Podsumowanie analizy i uzasadnienie własnego rozwiązania}

\medskip

Analiza rozwiązań alternatywnych ujawniła fundamentalny kompromis: \textbf{zaawansowanie technologiczne vs koszty i elastyczność}. Rozwiązania chmurowe od Amazona oraz Google oferują wysoką jakość rekomendacji dzięki algorytmom deep learning, ale wiążą się z wysokimi kosztami operacyjnymi, vendor lock-in (uzależnieniem od dostawcy) oraz brakiem kontroli nad algorytmami. Apache Mahout eliminuje koszty licencyjne, ale wymaga zaawansowanej wiedzy technicznej oraz kosztownej infrastruktury Spark.

\textbf{Uzasadnienie sensowności własnego rozwiązania:}

\begin{itemize}
\item \textbf{Integracja trzech komplementarnych metod w jednym systemie}:

Żadne z analizowanych rozwiązań nie oferuje natywnej integracji Content-Based Filtering, logiki rozmytej oraz modeli probabilistycznych w jednym systemie:
\begin{itemize}
\item Amazon Personalize: brak CBF, Fuzzy Logic i modeli probabilistycznych,
\item Google Recommendations AI: brak CBF, Fuzzy Logic i łańcuchów Markova,
\item Apache Mahout: brak wszystkich trzech metod w najnowszej wersji.
\end{itemize}

Własna implementacja łączy:
\begin{itemize}
\item Content-Based Filtering - rozwiązuje cold start dla nowych produktów,
\item Fuzzy Logic - personalizacja z wysoką interpretowalnością,
\item Modele probabilistyczne (Markov + Naive Bayes) - predykcja sekwencji zakupowych.
\end{itemize}

\item \textbf{Optymalizacja kosztów dla małych i średnich platform}:

Własna implementacja (Django + PostgreSQL) eliminuje wysokie koszty licencyjne rozwiązań chmurowych przy zachowaniu wysokiej jakości rekomendacji. System jest szczególnie atrakcyjny dla małych i średnich platform e-commerce (do 10000 produktów, do 100000 użytkowników), które potrzebują zaawansowanych funkcjonalności rekomendacji przy ograniczonym budżecie.

\item \textbf{Kontrola nad logiką biznesową i możliwość dostosowania}:

Własna implementacja umożliwia unikalne podejścia niedostępne w gotowych rozwiązaniach:
\begin{itemize}
\item \textbf{Ważone wektory cech} dla CBF - kategorie 40\%, tagi 30\%, cena 20\%, słowa kluczowe 10\%,
\item \textbf{System wnioskowania Mamdaniego} dla logiki rozmytej - modelowanie niepewności preferencji użytkowników z wykorzystaniem funkcji przynależności,
\item \textbf{Łańcuch Markowa pierwszego rzędu} - predykcja następnej kategorii zakupu na podstawie macierzy przejść.
\end{itemize}

\item \textbf{Elastyczność technologiczna i brak uzależnienia od dostawcy}:

Aplikacja oparta na Django + React + PostgreSQL może być wdrożona na dowolnej platformie: AWS, GCP, Azure, własne serwery lub localhost. Migracja między platformami wymaga jedynie zmiany parametrów połączenia - logika rekomendacji pozostaje niezmieniona.

Dla porównania: migracja z Amazon Personalize do Google Recommendations AI wymaga przepisania całej integracji (śledzenie zdarzeń, dane treningowe, wywołania API) oraz ponownego uczenia modeli, co może trwać tygodnie i powodować degradację jakości rekomendacji.

\item \textbf{Cel edukacyjny i interpretowalność}:

Praca inżynierska ma charakter badawczy i edukacyjny. Implementacja algorytmów od podstaw (bez zewnętrznych bibliotek uczenia maszynowego) zapewnia głębokie zrozumienie mechanizmów działania każdej metody, co jest niemożliwe przy wykorzystaniu gotowych usług chmurowych działających jako czarne skrzynki.

Logika rozmyta z regułami IF-THEN oferuje pełną przejrzystość decyzji algorytmu (ang. \textit{explainable AI} - wyjaśnialna sztuczna inteligencja), czego brakuje w modelach głębokiego uczenia używanych przez Google i Amazon.
\end{itemize}

\textbf{Podsumowanie}:

Własna implementacja systemu rekomendacji stanowi optymalny wybór dla małych i średnich platform e-commerce, łączący:
\begin{itemize}
\item Wysoką jakość rekomendacji (trzy komplementarne metody pokrywające różne aspekty problemu),
\item Pełną kontrolę nad algorytmami i możliwość dostosowania do specyfiki biznesowej,
\item Niskie koszty operacyjne (brak opłat licencyjnych rozwiązań chmurowych),
\item Interpretowalność wyników i możliwość debugowania,
\item Elastyczność technologiczną (brak uzależnienia od konkretnego dostawcy chmury),
\item Wartość edukacyjną (implementacja od podstaw).
\end{itemize}

Rozwiązanie jest szczególnie atrakcyjne dla platform potrzebujących zaawansowanych funkcjonalności rekomendacji przy ograniczonym budżecie oraz możliwości dostosowania logiki do specyficznych wymagań biznesowych.

\newpage

\section*{Rozdzia\l{} 3}
\addcontentsline{toc}{section}{Rozdział 3: Projekt systemu rekomendacyjnego}
\section*{Projekt systemu rekomendacyjnego}

Rozdział przedstawia szczegółowy opis projektowanej aplikacji e-commerce z zaawansowanym systemem rekomendacji produktów. Zaprezentowano wymagania funkcjonalne, diagram przypadków użycia ilustrujący interakcje użytkowników z systemem oraz architekturę funkcjonalną rozwiązania zrealizowanego we współpracy dwuosobowej.

\subsection*{3.1 Cel i zakres aplikacji}
\addcontentsline{toc}{subsection}{3.1 Cel i zakres aplikacji}

Aplikacja stanowi kompleksowe rozwiązanie e-commerce z systemem rekomendacji opartym o trzy metody:
\begin{itemize}
\item \textbf{Content-Based Filtering} - odkrywanie produktów podobnych na podstawie cech produktowych (kategorie, tagi, cena, słowa kluczowe),
\item \textbf{Fuzzy Logic} - personalizacja rekomendacji z wykorzystaniem systemu wnioskowania Mamdaniego i funkcji przynależności,
\item \textbf{Modele probabilistyczne} - predykcja sekwencji zakupowych (łańcuch Markowa) oraz prawdopodobieństwa zakupu (Naive Bayes).
\end{itemize}

System został zaprojektowany z myślą o małych i średnich platformach e-commerce (do 10000 produktów, do 100000 użytkowników), zapewniając funkcjonalności rekomendacyjne porównywalne z rozwiązaniami enterprise przy znacznie niższych kosztach operacyjnych i pełnej kontroli nad algorytmami.

\subsection*{3.2 Typy użytkowników systemu}
\addcontentsline{toc}{subsection}{3.2 Typy użytkowników systemu}

System obsługuje trzy typy użytkowników zorganizowanych w hierarchiczną strukturę uprawnień:

\textbf{1. Gość (użytkownik niezalogowany)}

Gość może przeglądać katalog produktów, korzystać z wyszukiwarki, filtrować produkty według kategorii i ceny oraz dodawać produkty do koszyka. Dostęp do rekomendacji jest ograniczony do metody Content-Based Filtering na stronach szczegółów produktów (sekcja „Podobne produkty"). Gość nie może składać zamówień ani przeglądać historii zakupów. Aby sfinalizować zakup, musi się zarejestrować lub zalogować.

\textbf{2. Klient (użytkownik zalogowany)}

Klient dziedziczy wszystkie uprawnienia gościa oraz otrzymuje dodatkowe funkcjonalności: składanie zamówień, przeglądanie historii zamówień, zarządzanie kontem użytkownika (zmiana hasła, danych osobowych), dodawanie opinii i ocen produktów. Klient ma dostęp do panelu klienta z dedykowanymi sekcjami rekomendacji dla wszystkich trzech metod: Content-Based Filtering, Fuzzy Logic oraz Modele Probabilistyczne. System buduje spersonalizowany profil użytkownika na podstawie historii zakupów, co umożliwia personalizację rekomendacji przez algorytm Fuzzy Logic (profil wrażliwości cenowej) oraz modele probabilistyczne (przewidywanie sekwencji zakupowych i prawdopodobieństwa odejścia klienta).

\textbf{3. Administrator}

Administrator dziedziczy wszystkie uprawnienia klienta oraz otrzymuje pełny dostęp do panelu administracyjnego. Funkcjonalności administracyjne obejmują: zarządzanie produktami (dodawanie, edycja, usuwanie), zarządzanie zamówieniami (zmiana statusów, przeglądanie szczegółów), zarządzanie użytkownikami (nadawanie/odbieranie uprawnień, blokowanie kont), generowanie macierzy podobieństw dla Content-Based Filtering, trening modeli probabilistycznych (łańcuch Markowa, Naive Bayes), debugowanie algorytmów rekomendacji (panele CBF Debug, Fuzzy Debug, Probabilistic Debug), konfiguracja parametrów algorytmów oraz dostęp do statystyk i analityki sprzedaży.

Hierarchia uprawnień zapewnia separację odpowiedzialności i bezpieczeństwo systemu. Użytkownicy mogą samodzielnie rejestrować się jako klienci, natomiast role administratora przyznawane są ręcznie przez istniejących administratorów w panelu zarządzania użytkownikami.

\subsection*{3.3 Wymagania funkcjonalne systemu}
\addcontentsline{toc}{subsection}{3.3 Wymagania funkcjonalne systemu}

System został zaprojektowany z uwzględnieniem następujących grup wymagań funkcjonalnych zorganizowanych według obszarów funkcjonalności:

\textbf{Autentykacja i autoryzacja}
\begin{itemize}
\item Logowanie użytkowników z wykorzystaniem email i hasła (JWT - JSON Web Tokens),
\item Rejestracja nowych użytkowników z walidacją danych wejściowych (unikalność email, siła hasła),
\item Zarządzanie rolami użytkowników: Gość (niezalogowany), Klient (zalogowany), Administrator,
\item Autoryzacja dostępu do zasobów na podstawie roli użytkownika (middleware weryfikujący JWT),
\item Zarządzanie kontem użytkownika: edycja danych osobowych, zmiana hasła.
\end{itemize}

\textbf{Zarządzanie katalogiem produktów}
\begin{itemize}
\item Przeglądanie produktów z filtrowaniem według kategorii, przedziału cenowego, oceny produktu (dostępne dla: Gość, Klient, Administrator),
\item Wyszukiwanie produktów z wykorzystaniem algorytmu odległości Levensteina (tolerancja błędów ortograficznych),
\item Sortowanie produktów według: ceny (rosnąco/malejąco), popularności (liczba zamówień), oceny (rating), daty dodania,
\item Wyświetlanie szczegółów produktu: opis, specyfikacja techniczna, opinie użytkowników, średnia ocena,
\item Dodawanie produktów do koszyka (Gość, Klient),
\item Zarządzanie produktami przez administratora: dodawanie nowych produktów, edycja istniejących, usuwanie produktów, zarządzanie kategoriami i tagami (Administrator).
\end{itemize}

\textbf{Obsługa zamówień}
\begin{itemize}
\item Składanie zamówień - wymaga zalogowania, konwersja koszyka gościa na zamówienie po logowaniu (Klient),
\item Śledzenie statusu zamówień w czasie rzeczywistym: oczekujące, w realizacji, wysłane, dostarczone (Klient),
\item Przeglądanie historii zamówień z możliwością filtrowania według daty i statusu (Klient),
\item Zarządzanie zamówieniami przez administratora: przeglądanie wszystkich zamówień, zmiana statusów, anulowanie zamówień (Administrator),
\item Dashboard z statystykami sprzedaży: przychód dzienny/miesięczny, najpopularniejsze produkty, współczynnik konwersji (Administrator).
\end{itemize}

\textbf{System rekomendacji produktów}
\begin{itemize}
\item Wyświetlanie rekomendacji Content-Based Filtering: produkty podobne na podstawie cech (kategoria, tagi, cena, słowa kluczowe) z wykorzystaniem podobieństwa kosinusowego (Klient),
\item Wyświetlanie rekomendacji Fuzzy Logic: personalizacja z wykorzystaniem profilu rozmytego użytkownika, 6 reguł IF-THEN typu Mamdani, funkcje przynależności dla ceny/jakości/popularności (Klient),
\item Wyświetlanie rekomendacji Probabilistic Models: predykcja sekwencji zakupowych (łańcuch Markova), prawdopodobieństwo zakupu i odejścia klienta (Naive Bayes) (Klient),
\item Przeglądanie profilu użytkownika: preferencje kategorialne, wrażliwość cenowa, ulubione tagi (Klient),
\item Przełączanie między metodami rekomendacji w panelu administratora (Administrator).
\end{itemize}

\textbf{Panel administracyjny i debugowanie}
\begin{itemize}
\item Zarządzanie użytkownikami: przeglądanie listy użytkowników, zmiana ról, blokowanie kont (Administrator),
\item Panele debugowania algorytmów rekomendacji (Administrator):
  \begin{itemize}
\item \textit{Content-Based Filtering Debug} - macierz podobieństw produktów, wagi cech (kategorie 40\%, tagi 30\%, cena 20\%, słowa kluczowe 10\%), statystyki pokrycia rekomendacji,
\item \textit{Fuzzy Logic Debug} - profile rozmyte użytkowników, funkcje przynależności, aktywacja reguł IF-THEN dla konkretnych produktów,
\item \textit{Probabilistic Models Debug} - macierz przejść Markova (48x48 kategorii), prawdopodobieństwa zakupu Naive Bayes, predykcja odejścia klienta,
\end{itemize}
\item Generowanie macierzy podobieństw dla algorytmu CBF (ręczne wyzwalanie przeliczenia) (Administrator),
\item Trening modeli probabilistycznych: aktualizacja łańcucha Markowa i klasyfikatorów Naive Bayes na bieżących danych (Administrator),
\item Wizualizacja metryk wydajności algorytmów: średni czas odpowiedzi, liczba wygenerowanych rekomendacji, współczynnik trafienia pamięci podręcznej (Administrator).
\end{itemize}

\subsection*{3.4 Diagram przypadków użycia}
\addcontentsline{toc}{subsection}{3.4 Diagram przypadków użycia}

Diagram przypadków użycia (rys. \ref{fig:use_case_project}) przedstawia kompletny widok funkcjonalności systemu oraz relacji między aktorami a przypadkami użycia. System obsługuje trzy główne typy aktorów: Gościa (użytkownik niezalogowany), Klienta (użytkownik zalogowany) oraz Administratora (zarządzający systemem). Relacje dziedziczenia między aktorami (Gość $\rightarrow$ Klient $\rightarrow$ Administrator) odzwierciedlają hierarchię uprawnień - każdy następny poziom dziedziczy wszystkie funkcjonalności poprzedniego i dodaje nowe, specyficzne dla swojej roli.

\begin{figure}[H]
  \centering
  \includegraphics[width=0.95\textwidth]{images/useCaseDiagram.png}
  \caption{Diagram przypadków użycia systemu.}
  \label{fig:use_case_project}
\end{figure}

System został podzielony na trzy główne obszary funkcjonalne:

\textbf{1. Obszar publiczny} (dostępny dla wszystkich użytkowników) - podstawowe funkcjonalności e-commerce takie jak przeglądanie produktów, wyszukiwanie, dodawanie do koszyka oraz procesy autentykacji (logowanie, rejestracja).

\textbf{2. Obszar klienta} (wymaga zalogowania) - funkcjonalności transakcyjne obejmujące składanie zamówień, śledzenie ich statusu, zarządzanie kontem oraz dostęp do spersonalizowanych rekomendacji.

\textbf{3. Obszar administracyjny} (wymaga uprawnień administratora) - narzędzia do zarządzania całym systemem: produktami, zamówieniami, użytkownikami oraz dostęp do paneli statystycznych i debugowania algorytmów rekomendacji.

\subsection*{3.5 Architektura funkcjonalna systemu}
\addcontentsline{toc}{subsection}{3.5 Architektura funkcjonalna systemu}

System został zaprojektowany w architekturze warstwowej, gdzie każda warstwa odpowiada za konkretny aspekt funkcjonalności. Komunikacja między warstwami odbywa się poprzez RESTful API z uwierzytelnianiem JSON Web Tokens (JWT).

\textbf{Warstwa prezentacji} - interfejsy użytkownika dostosowane do ról (gość, klient, administrator):
\begin{itemize}
\item \textbf{Panel klienta} - dashboard z historią zamówień, sekcje rekomendacji (CBF, Fuzzy, Probabilistic), edycja profilu, śledzenie statusu zamówień,
\item \textbf{Panel administracyjny} - zarządzanie produktami/zamówieniami/użytkownikami, statystyki sprzedaży, panele debugowania algorytmów rekomendacji.
\end{itemize}

\textbf{Warstwa logiki biznesowej} - implementacja algorytmów rekomendacji oraz logiki e-commerce:
\begin{itemize}
\item \textbf{Moduły rekomendacyjne}:
  \begin{itemize}
\item \textit{Moduł Content-Based Filtering} - generowanie macierzy podobieństwa produktów na podstawie cech (kategorie, tagi, cena, słowa kluczowe),
\item \textit{Moduł Fuzzy Logic} - system wnioskowania Mamdani z funkcjami przynależności i regułami IF-THEN,
\item \textit{Moduł Probabilistic Models} - łańcuch Markowa dla predykcji sekwencji zakupowych oraz Naive Bayes dla prawdopodobieństwa zakupu,
\end{itemize}
\item \textbf{Logika transakcyjna} - składanie zamówień, zarządzanie statusami, walidacja danych, uwierzytelnianie JWT.
\end{itemize}

\textbf{Warstwa danych} - relacyjna baza danych PostgreSQL 14 przechowująca:
\begin{itemize}
\item Dane produktów (nazwa, opis, cena, kategorie, specyfikacje, zdjęcia) - 500 produktów w 48 kategoriach,
\item Dane użytkowników (konta, profile, uprawnienia, role) - 20 użytkowników testowych,
\item Dane transakcyjne (zamówienia, produkty w zamówieniach, statusy) - 265 zamówień z 569 pozycjami,
\item Dane opinii (recenzje tekstowe, oceny gwiazdkowe, timestamps) - $\sim$1750 opinii,
\item Wyniki algorytmów (macierze podobieństwa CBF, profile rozmyte, macierz przejść Markova, predykcje Naive Bayes) - przechowywane w dedykowanych tabelach.
\end{itemize}

\textbf{Integracja warstw} odbywa się poprzez RESTful API z automatyczną synchronizacją - zmiana danych w jednej warstwie propaguje aktualizacje do pozostałych. Django Signals zapewniają automatyczne przeliczanie rekomendacji przy dodaniu nowego produktu lub zamówienia.

\newpage

\section*{Rozdzia\l{} 4}
\addcontentsline{toc}{section}{Rozdział 4: Przedstawienie wykorzystanego stosu technologicznego oraz praktycznej realizacji projektu}
\section*{Przedstawienie wykorzystanego stosu technologicznego oraz praktycznej realizacji projektu}

Rozdział opisuje architekturę techniczną zrealizowanego systemu e-commerce wraz z modułem rekomendacyjnym. Przedstawiono wybór i uzasadnienie stosu technologicznego, strukturę warstwy backendowej (Django REST Framework) oraz frontendowej (React), model relacyjnej bazy danych PostgreSQL, a także mechanizmy konteneryzacji i wdrożenia aplikacji z wykorzystaniem Docker Compose.

\subsection*{4.1 Architektura systemu}
\addcontentsline{toc}{subsection}{4.1 Architektura systemu}

Aplikacja została zaprojektowana w architekturze klient-serwer opartej na technologiach Django (backend) oraz React (frontend). Komunikacja odbywa się poprzez RESTful API z uwierzytelnianiem tokenowym JSON Web Tokens (JWT). Struktura aplikacji wyraźnie rozdziela warstwę prezentacji (React SPA), logikę biznesową (widoki Django i serializery) oraz warstwę danych (PostgreSQL).

\textbf{Główne założenia architektoniczne:}

\begin{itemize}
\item \textbf{Separacja frontendu i backendu} - możliwość niezależnego rozwoju i skalowania obu warstw,
\item \textbf{Podejście API-first (API-first approach)} - wszystkie funkcjonalności dostępne przez REST API,
\item \textbf{Uwierzytelnianie bezstanowe (Stateless authentication)} - token JWT eliminuje potrzebę sesji po stronie serwera,
\item \textbf{Modułowa struktura} - każdy algorytm rekomendacji stanowi niezależny moduł.
\end{itemize}

\subsection*{4.2 Stos technologiczny backendu}
\addcontentsline{toc}{subsection}{4.2 Stos technologiczny backendu}

\textbf{Django 5.1.4 (Python 3.11)}

Django stanowi fundament aplikacji serwerowej, zapewniając architekturę MVC, system ORM (Object-Relational Mapping - mapowanie obiektowo-relacyjne) dla abstrakcji bazy danych oraz mechanizmy bezpieczeństwa (CSRF, XSS, SQL injection prevention). Architektura backendu opiera się na wzorcu Model-View-Serializer (MVS). Django zapewnia komponenty ORM, Signals oraz Middleware:

\textbf{Kluczowe komponenty Django:}

\begin{itemize}
\item \textbf{Django ORM} - mapowanie obiektowo-relacyjne umożliwiające operacje na bazie bez SQL,
\item \textbf{Django Signals} - mechanizm automatycznej aktualizacji rekomendacji przy zmianach danych,
\item \textbf{Django Middleware (oprogramowanie pośredniczące)} - obsługa CORS, uwierzytelnienie JWT, pamięć podręczna.
\end{itemize}

\textbf{Django REST Framework 3.15.2}

Rozszerza Django o funkcjonalności API RESTful:

\begin{itemize}
\item \textbf{Serializery} - konwersja obiektów Django na JSON z walidacją,
\item \textbf{ViewSets (zestawy widoków)} - widoki implementujące operacje CRUD,
\item \textbf{Uwierzytelnianie (Authentication)} - wsparcie dla JWT, uwierzytelnianie sesyjne,
\item \textbf{Pagination (paginacja)} - automatyczne stronicowanie wyników.
\end{itemize}

\textbf{Biblioteki uczenia maszynowego (ang. \textit{Machine Learning})}

Do operacji numerycznych i obliczania podobieństw wykorzystano:
\begin{itemize}
\item \textbf{NumPy 1.24} - operacje macierzowe dla wektorów cech i macierzy podobieństwa w CBF i modelu probabilistycznym,
\item \textbf{scikit-learn} - funkcja cosine\_similarity() dla operacji macierzowych.
\end{itemize}

\subsection*{4.3 Stos technologiczny frontendu}
\addcontentsline{toc}{subsection}{4.3 Stos technologiczny frontendu}

\textbf{React 18}

Warstwa prezentacji została zrealizowana jako aplikacja jednostronicowa (Single Page Application - SPA) w technologii React 18. Kluczowe cechy wykorzystanej biblioteki:

\begin{itemize}
\item \textbf{Architektura komponentowa (Component-based)} - reużywalne komponenty UI,
\item \textbf{Virtual DOM (wirtualny DOM)} - optymalizacja renderowania,
\item \textbf{React Hooks} - useState, useEffect, useContext.
\end{itemize}

\textbf{Biblioteki wspierające}

\begin{itemize}
\item \textbf{React Router v6} - trasowanie (routing) dla aplikacji SPA,
\item \textbf{Axios} - komunikacja z API, przechwytywacze JWT (interceptors),
\item \textbf{Framer Motion} - płynne animacje,
\item \textbf{Context API} - zarządzanie stanem (AuthContext, CartContext).
\end{itemize}

\newpage

\subsection*{4.4 Projekt architektury systemu}
\addcontentsline{toc}{subsection}{4.4 Projekt architektury systemu}

Po wyborze stosu technologicznego (Django, React, PostgreSQL, Docker) zaprojektowano szczegółową architekturę aplikacji, definiując strukturę modułów backendu i frontendu oraz mechanizmy optymalizacji wydajności.

\medskip
\noindent\textbf{Struktura backendu Django}

\medskip

Każdy komponent systemu rekomendacyjnego posiada dedykowane pliki realizujące zasadę separacji odpowiedzialności (Separation of Concerns):

\begin{itemize}
\item \textbf{models.py} -- definicje modeli Django ORM reprezentujących tabele bazodanowe (Product, Order, Opinion, ProductSimilarity, FuzzyUserProfile, MarkovTransitions, PurchaseProbability, RecommendationSettings),
\item \textbf{serializers.py} -- klasy Django REST Framework wykonujące konwersję obiektów Django $\leftrightarrow$ JSON z walidacją danych wejściowych,
\item \textbf{views.py} -- widoki API implementujące logikę biznesową dla operacji CRUD na produktach i zamówieniach,
\item \textbf{custom\_recommendation\_engine.py} -- implementacja algorytmów Content-Based Filtering, łańcucha Markowa oraz naiwnego klasyfikatora Bayesa,
\item \textbf{fuzzy\_logic\_engine.py} -- silnik logiki rozmytej z systemem wnioskowania Mamdaniego,
\item \textbf{recommendation\_views.py} -- endpointy API dla CBF: \texttt{/api/content-based-debug/}, \texttt{/api/generate-similarities/},
\item \textbf{fuzzy\_views.py} -- endpointy API dla Fuzzy Logic: \texttt{/api/fuzzy-recommendations/}, \texttt{/api/fuzzy-debug/},
\item \textbf{probabilistic\_views.py} -- endpointy API dla modeli probabilistycznych: \texttt{/api/markov-recommendations/}, \texttt{/api/bayesian-insights/},
\item \textbf{signals.py} -- mechanizm automatycznej aktualizacji rekomendacji przy zmianach danych (nowe zamówienia, nowe produkty) wykorzystujący system sygnałów Django.
\end{itemize}

\medskip
\noindent\textbf{Struktura frontendu React}

\medskip

Aplikacja została zbudowana jako Single Page Application (SPA) z wykorzystaniem wzorców programowania funkcyjnego i React Hooks. Architektura składa się z modułowych komponentów odpowiedzialnych za poszczególne funkcjonalności:

\begin{itemize}
\item \textbf{App.js} -- główny komponent zarządzający routingiem React Router v6 oraz globalnym stanem aplikacji poprzez Context API (AuthContext, CartContext),
\item \textbf{Navbar.jsx} -- responsywna nawigacja z wyszukiwarką produktów, linkami do kluczowych sekcji, przyciskami logowania i rejestracji oraz ikoną koszyka z licznikiem produktów,
\item \textbf{SearchModal.jsx} -- zaawansowany modal wyszukiwania z wykorzystaniem algorytmu odległości Levensteina dla wyszukiwania tolerującego błędy (fuzzy search),
\item \textbf{ShopContent.jsx} -- komponent wyświetlający katalog produktów z dynamicznym filtrowaniem (kategorie, zakres cen, oceny użytkowników) oraz gridową prezentacją kart produktów,
\item \textbf{ProductPage.jsx} -- szczegółowy widok produktu zawierający galerię zdjęć, opis, specyfikacje techniczne, sekcję opinii użytkowników oraz rekomendacje podobnych produktów,
\item \textbf{CartContent.jsx} -- koszyk zakupowy z możliwością modyfikacji ilości produktów oraz obliczania sumarycznej wartości zamówienia,
\item \textbf{ClientPanel.jsx} -- panel klienta z zakładkami: Dashboard (podsumowanie konta), Orders (historia zamówień), Account (ustawienia konta), Recommendations (spersonalizowane rekomendacje CBF, Fuzzy Logic, Probabilistic),
\item \textbf{AdminPanel.jsx} -- panel administracyjny z narzędziami zarządzania systemem oraz debugowania algorytmów uczenia maszynowego.
\end{itemize}

\medskip
\noindent\textbf{Mechanizmy optymalizacji systemu}

\medskip

System rekomendacyjny implementuje szereg mechanizmów optymalizacyjnych zapewniających wydajność i skalowalność dla katalogów liczących setki produktów i dziesiątki użytkowników.

\textbf{1. Pamięć podręczna wyników (Django Cache Framework)}

System wykorzystuje wbudowany mechanizm pamięci podręcznej Django z różnymi poziomami czasu wygaśnięcia:

\begin{itemize}
\item \textbf{CACHE\_TIMEOUT\_SHORT = 300s (5 minut)} - rekomendacje dla zalogowanych użytkowników, wyniki wyszukiwania tolerującego błędy
\item \textbf{CACHE\_TIMEOUT\_MEDIUM = 1800s (30 minut)} - macierze podobieństw, reguły asocjacyjne Apriori
\item \textbf{CACHE\_TIMEOUT\_LONG = 7200s (2 godziny)} - wygenerowane macierze CBF, modele probabilistyczne
\end{itemize}

Pamięć podręczna znacząco redukuje obciążenie bazy danych i przyspiesza odpowiedzi API. Przykładowo, pobranie rekomendacji CBF z pamięci (ang. \textit{cache HIT} - trafienie) trwa 50-100ms, podczas gdy przy braku danych w pamięci (ang. \textit{cache MISS} - chybienie) system musi wykonać zapytanie SQL i obliczyć podobieństwa, co zajmuje 5-10 sekund.

\textbf{2. Operacje zbiorcze (bulk operations)}

Zamiast pojedynczych zapytań INSERT dla każdego podobieństwa, system wykorzystuje metodę \texttt{bulk\_create()} Django ORM do wstawiania rekordów wsadowo. Dla 4000 par produktów bulk insert jest \textbf{80x szybszy} niż pojedyncze inserty (5 sekund vs 400 sekund).

\textbf{3. Prefetching i select\_related}

System minimalizuje liczbę zapytań SQL poprzez wczytywanie wyprzedzające (eager loading) powiązanych obiektów:

\begin{itemize}
\item \texttt{select\_related()} - dla relacji ForeignKey (One-to-Many): łączy tabele przez JOIN
\item \texttt{prefetch\_related()} - dla relacji Many-to-Many: wykonuje dodatkowe zapytanie i łączy w Pythonie
\end{itemize}

Przykład: pobranie 100 produktów z kategoriami bez prefetchingu = 101 zapytań SQL (problem N+1). Z \texttt{prefetch\_related('categories')} = 2 zapytania SQL.

\textbf{4. Przycinanie progowe (threshold pruning)}

Content-Based Filtering zapisuje do bazy danych tylko podobieństwa przekraczające próg 20\% (\texttt{similarity\_threshold = 0.2}):

\begin{itemize}
\item Bez pruning: $500 \times 499 / 2 = 124\,750$ par produktów
\item Z pruning (>20\%): $\sim$4\,000 zapisanych podobieństw
\item \textbf{Redukcja o 97\%} rozmiaru tabeli \texttt{ProductSimilarity}
\end{itemize}

Produkty o podobieństwie <20\% są pomijane, ponieważ nie stanowią wartościowych rekomendacji dla użytkownika.

\textbf{5. Ograniczenie liczby porównań per produkt}

Algorytm CBF limituje liczbę obliczanych podobieństw do maksymalnie 50 najbardziej prawdopodobnych kandydatów per produkt (\texttt{max\_comparisons\_per\_product = 50}). Zamiast porównywać produkt ze wszystkimi 499 innymi produktami, system:

\begin{enumerate}
    \item Filtruje produkty z tej samej kategorii (znacznie mniejszy zbiór)
    \item Sortuje według ceny (produkty o zbliżonej cenie są bardziej podobne)
    \item Oblicza podobieństwo tylko dla top 50 kandydatów
\end{enumerate}

\textbf{6. Indeksowanie bazy danych}

PostgreSQL automatycznie tworzy indeksy B-tree na:
\begin{itemize}
\item Klucze główne (PRIMARY KEY) - wszystkie tabele
\item Klucze obce (FOREIGN KEY) - przyspieszenie operacji JOIN
\item Pola często wyszukiwane: \texttt{product\_id}, \texttt{user\_id}, \texttt{category\_id}
\end{itemize}

Dodatkowo system wykorzystuje indeks GIN dla pełnotekstowego wyszukiwania produktów.

\textbf{7. Optymalizacje algorytmiczne}

\begin{itemize}
\item \textbf{Rzadkie wektory (sparse vectors)} - CBF przechowuje tylko niezerowe cechy w słowniku zamiast pełnych wektorów (redukcja pamięci o 90\%)
\item \textbf{Wygładzanie Laplace'a} - Naive Bayes dodaje +1 do wszystkich liczników, zapobiegając zerowemu prawdopodobieństwu
\item \textbf{Normalizacja} - Fuzzy Logic normalizuje wejścia do zakresu [0, 1]
\end{itemize}

Kombinacja pamięci podręcznej, operacji zbiorczych, przycinania progowego oraz indeksowania pozwala systemowi obsługiwać 500 produktów i generować rekomendacje w czasie rzeczywistym (<200ms) nawet przy braku danych w pamięci podręcznej. Dla większych katalogów (>10\,000 produktów) zalecane są dalsze optymalizacje: przybliżone wyszukiwanie najbliższych sąsiadów (ang. \textit{approximate nearest neighbors} - algorytmy LSH, HNSW), partycjonowanie tabel PostgreSQL, oraz rozproszenie obliczeń przy użyciu kolejki zadań asynchronicznych Celery z brokerem Redis.

\subsection*{4.5 Baza danych PostgreSQL}
\addcontentsline{toc}{subsection}{4.5 Baza danych PostgreSQL}

\textbf{Wybór PostgreSQL 14}

PostgreSQL został wybrany jako system zarządzania bazą danych ze względu na następujące cechy:

\begin{itemize}
\item \textbf{Zaawansowane indeksy} - wsparcie dla B-tree (domyślne), GIN (wyszukiwanie pełnotekstowe), BRIN (optymalizacja dla dużych tabel),
\item \textbf{Typ danych JSONB} - natywne przechowywanie i indeksowanie struktur JSON (wykorzystane w tabeli \texttt{method\_user\_purchase\_pattern} dla sezonowości zakupów),
\item \textbf{Transakcje ACID} - gwarancja atomowości, spójności, izolacji i trwałości operacji krytycznych (zamówienia, płatności),
\item \textbf{Klucze obce i constrainty} - automatyczne wymuszanie integralności referencyjnej oraz walidacji danych (np. rating 1-5 w opiniach),
\item \textbf{Optymalizacja JOIN} - wydajne łączenie tabel w złożonych zapytaniach rekomendacyjnych,
\item \textbf{Full-text search} - wbudowane mechanizmy wyszukiwania tekstowego dla produktów.
\end{itemize}

\textbf{Struktura bazy danych}

Baza składa się z \textbf{25 tabel} podzielonych na 4 moduły funkcjonalne:

\textbf{1. Moduł produktów i użytkowników (12 tabel):}

\begin{itemize}
\item \texttt{db\_product} - dane produktów (ID, nazwa, cena, opis),
\item \texttt{db\_category} - kategorie produktów z hierarchią,
\item \texttt{db\_product\_category} - relacja Many-to-Many produktów i kategorii,
\item \texttt{db\_photo\_product} - ścieżki do zdjęć produktów,
\item \texttt{db\_specification} - szczegółowe parametry techniczne produktów,
\item \texttt{db\_tag} - tagi do filtrowania produktów,
\item \texttt{db\_sale} - promocje i rabaty,
\item \texttt{db\_user} - konta użytkowników (role: admin/client),
\item \texttt{db\_order} - zamówienia z timestampami i statusami,
\item \texttt{db\_order\_product} - produkty w zamówieniach (ilość, cena),
\item \texttt{db\_cart\_item} - koszyk zakupowy przed finalizacją,
\item \texttt{db\_complaint} - reklamacje powiązane z zamówieniami.
\end{itemize}

\textbf{2. Moduł opinii i analizy sentymentu (3 tabele):}

\begin{itemize}
\item \texttt{db\_opinion} - opinie użytkowników (treść, rating 1-5),
\item \texttt{method\_sentiment\_analysis} - wyniki analizy sentymentu dla opinii,
\item \texttt{method\_product\_sentiment\_summary} - zagregowany sentyment produktu.
\end{itemize}

\textbf{3. Moduł metod rekomendacji (5 tabel):}

\begin{itemize}
\item \texttt{method\_product\_similarity} - macierz podobieństw produktów (Collaborative Filtering),
\item \texttt{method\_user\_product\_recommendation} - spersonalizowane rekomendacje użytkowników,
\item \texttt{method\_productassociation} - reguły asocjacyjne Apriori,
\item \texttt{method\_user\_interactions} - historia interakcji użytkowników,
\item \texttt{method\_recommendation\_settings} - konfiguracja algorytmów dla użytkownika.
\end{itemize}

\textbf{4. Moduł analityczny i prognozowanie (5 tabel):}

\begin{itemize}
\item \texttt{method\_purchase\_probability} - prawdopodobieństwo zakupu produktu przez użytkownika,
\item \texttt{method\_sales\_forecast} - prognoza sprzedaży produktów,
\item \texttt{method\_user\_purchase\_pattern} - wzorce zakupowe użytkowników,
\item \texttt{method\_product\_demand\_forecast} - prognoza popytu i poziomy magazynowe,
\item \texttt{method\_risk\_assessment} - ocena ryzyka dla użytkowników i produktów.
\end{itemize}

Wszystkie migracje Django ORM zostały wygenerowane automatycznie na podstawie modeli Python i zarządzane przez system wersjonowania \texttt{django.db.migrations}.

\newpage

\textbf{Diagramy ERD}

\begin{figure}[H]
  \centering
  \includegraphics[width=1\textwidth]{images/appErd.png}
  \caption{Diagram ERD głównych tabel aplikacji.}
  \label{fig:erd1}
\end{figure}

Rysunek \ref{fig:erd1} przedstawia rdzeń aplikacji e-commerce. Kluczowe relacje:

\begin{itemize}
\item \textbf{db\_user $\rightarrow$ db\_order} (1:N) - jeden użytkownik składa wiele zamówień,
\item \textbf{db\_order $\rightarrow$ db\_order\_product} (1:N) - zamówienie zawiera wiele produktów,
\item \textbf{db\_product $\leftrightarrow$ db\_category} (N:M) - produkt należy do wielu kategorii,
\item \textbf{db\_product $\rightarrow$ db\_opinion} (1:N) - produkt ma wiele opinii.
\end{itemize}

\newpage
\begin{figure}[H]
  \centering
  \includegraphics[width=1\textwidth]{images/methodsErd.png}
  \caption{Diagram ERD tabel metod rekomendacyjnych.}
  \label{fig:erd2}
\end{figure}

Diagram \ref{fig:erd2} pokazuje tabele algorytmów ML:

\begin{itemize}
\item \textbf{method\_product\_similarity} - macierz podobieństw Content-Based Filtering (CBF) przechowująca similarity\_score między produktami,
\item \textbf{method\_user\_product\_recommendation} - cache spersonalizowanych rekomendacji użytkownika dla wszystkich algorytmów,
\item \textbf{method\_recommendation\_settings} - ustawienia aktywnego algorytmu rekomendacyjnego (CBF, Fuzzy Logic, Collaborative Filtering),
\item \textbf{method\_purchase\_probability} - predykcje prawdopodobieństwa zakupu obliczone przez klasyfikator Naive Bayes.
\end{itemize}

\subsection*{4.6 Deployment i konteneryzacja Docker}
\addcontentsline{toc}{subsection}{4.6 Deployment i konteneryzacja Docker}

Aplikacja została skonteneryzowana przy użyciu Docker Compose, zapewniając spójność środowiska między środowiskiem deweloperskim (development), testowym (staging) i produkcyjnym (production).

\begin{figure}[H]
  \centering
  \includegraphics[width=0.92\textwidth]{images/dockerView.png}
  \caption{Deployment aplikacji w architekturze Docker Compose.}
  \label{fig:docker_view}
\end{figure}

Architektura składa się z trzech kontenerów (rys. \ref{fig:docker_view}):

\textbf{1. Kontener frontendu (React 18)}
\begin{itemize}
\item Base image: node:18-alpine,
\item Port: 3000,
\item Volumes: montowanie src/ dla automatycznego przeładowania (hot-reload),
\item Environment: REACT\_APP\_API\_URL,
\item Zależności (Dependencies): package.json (React, Axios, React Router, Framer Motion).
\end{itemize}

\textbf{2. Kontener backendu (Django 5.1.4)}
\begin{itemize}
\item Base image: python:3.11-slim,
\item Port: 8000,
\item Volumes: montowanie projektu dla automatycznego przeładowania (hot-reload), wolumen dla plików multimedialnych,
\item Environment: DATABASE\_URL, SECRET\_KEY, DEBUG, ALLOWED\_HOSTS,
\item Zależności (Dependencies): requirements.txt (Django, DRF, psycopg2, NumPy, scikit-learn).
\end{itemize}

\textbf{3. Kontener bazy danych (PostgreSQL 14)}
\begin{itemize}
\item Base image: postgres:14-alpine,
\item Port: 5432,
\item Volumes: named volume postgres\_data (persystencja danych),
\item Environment: POSTGRES\_DB, POSTGRES\_USER, POSTGRES\_PASSWORD,
\item Healthcheck: pg\_isready.
\end{itemize}

\textbf{Zalety konteneryzacji Docker}

\begin{itemize}
\item \textbf{Izolacja} - każdy serwis w osobnym kontenerze, zero konfliktów zależności,
\item \textbf{Przenośność} - obraz zbudowany raz działa na dowolnym serwerze z silnikiem Docker,
\item \textbf{Łatwa konfiguracja} - komenda \texttt{docker-compose up} uruchamia całą aplikację,
\item \textbf{Skalowalność} - możliwość uruchomienia wielu instancji backendu dla równoważenia obciążenia.
\end{itemize}

\newpage

\section*{Rozdzia\l{} 5}
\addcontentsline{toc}{section}{Rozdział 5: Implementacja algorytmów rekomendacji}
\section*{Implementacja algorytmów rekomendacji}

Niniejszy rozdział opisuje szczegółowo implementację trzech metod rekomendacyjnych: Content-Based Filtering (CBF), logikę rozmytą (Fuzzy Logic) oraz modele probabilistyczne (Markov Chain + Naive Bayes). Każda metoda została zaimplementowana od podstaw bez użycia gotowych bibliotek rekomendacyjnych. Przedstawiono pseudokody algorytmów, wzory matematyczne oraz kluczowe aspekty techniczne implementacji. Praktyczne funkcjonowanie systemu oraz interfejsy użytkownika zostały przedstawione w rozdziale 6.

\subsection*{5.1 Content-Based Filtering}
\addcontentsline{toc}{subsection}{5.1 Content-Based Filtering}

\medskip
\noindent\textbf{Architektura systemu CBF}

\medskip

System Content-Based Filtering został zaimplementowany w klasie \\
\texttt{CustomContentBasedFilter} w pliku \texttt{custom\_recommendation\_engine.py}. \\
Architektura składa się z trzech głównych komponentów:

\textbf{1. Ekstraktor cech} (Feature Extractor) -- odpowiada za budowę ważonego wektora cech dla każdego produktu. Analizuje cztery źródła danych:

\begin{itemize}
    \item \textbf{Kategorie} (waga 40\%) -- główna klasyfikacja produktu. Format cechy: \\
    \texttt{category\_\{nazwa\}}. Przykład: \texttt{category\_Electronics}, \texttt{category\_Laptops}.
    \item \textbf{Tagi} (waga 30\%) -- dodatkowe deskryptory (np. "Gaming", "Premium", "Budget"). Format: \texttt{tag\_\{nazwa\}}.
    \item \textbf{Przedział cenowy} (waga 20\%) -- dyskretyzacja ceny: low (<100 PLN), medium (100-500), high (500-1500), premium (>1500).
    \item \textbf{Słowa kluczowe} (waga 10\%) -- top 10 słów z opisu produktu po filtracji stop-words.
\end{itemize}

\textbf{2. Kalkulator podobieństwa} -- oblicza podobieństwo kosinusowe między wektorami cech produktów. Operuje na wektorach rzadkich dla efektywności.

\textbf{3. Generator rekomendacji} -- zwraca top N produktów podobnych do danego produktu, z filtrowaniem według dostępności i progu podobieństwa.

\textbf{Przepływ danych w systemie CBF}:

System rozpoczyna przetwarzanie od pobrania produktów z bazy danych wykorzystując optymalizację \texttt{prefetch\_related()} dla kategorii i tagów w celu minimalizacji liczby zapytań SQL. Następnie dla każdego produktu wykonywana jest ekstrakcja ważonych cech (kategorie 40\%, tagi 30\%, cena 20\%, słowa kluczowe 10\%) do postaci wektorów cech. Algorytm oblicza podobieństwa kosinusowe dla wszystkich par produktów, po czym filtruje wyniki poniżej progu 0.2 (20\%) w celu eliminacji nieistotnych podobieństw. Wygenerowane pary podobieństw są zapisywane do tabeli \texttt{ProductSimilarity} za pomocą operacji zbiorczej \texttt{bulk\_create()} dla optymalizacji wydajności. Wyniki są buforowane w pamięci podręcznej na 2 godziny w celu przyspieszenia kolejnych zapytań.

\medskip
\noindent\textbf{Implementacja ekstrakcji cech}

\medskip

Metoda ekstrakcji cech buduje słownik cech z przypisanymi wagami. Algorytm w pseudokodzie:

\begin{lstlisting}[style=pseudocode]
FUNCTION extract_features(product):
    features = empty_dictionary

    FOR EACH category IN product.categories:
        features["category_" + category.name] = 0.40

    FOR EACH tag IN product.tags:
        features["tag_" + tag.name] = 0.30

    IF product.price < 100 THEN
        features["price_low"] = 0.20
    ELSE IF product.price < 500 THEN
        features["price_medium"] = 0.20
    ELSE IF product.price < 1500 THEN
        features["price_high"] = 0.20
    ELSE
        features["price_premium"] = 0.20
    END IF

    keywords = extract_keywords(product.description)
    FOR EACH word IN keywords[0:5]:
        features["keyword_" + word] = 0.10 / length(keywords)

    RETURN features
END FUNCTION
\end{lstlisting}

\textbf{Ekstrakcja słów kluczowych}:

Metoda \texttt{\_extract\_keywords()} przetwarza opis produktu:

\begin{enumerate}
    \item Konwersja na małe litery
    \item Usunięcie znaków interpunkcyjnych (regex)
    \item Tokenizacja na słowa
    \item Filtracja stop-words (zdefiniowana lista 200+ słów: "the", "and", "is", "a", "to", ...)
    \item Filtracja słów krótszych niż 4 znaki
    \item Zliczenie częstości (collections.Counter)
    \item Wybór top 10 najczęstszych słów
\end{enumerate}

\textbf{Dyskretyzacja ceny}:

Progi cenowe zostały dobrane empirycznie na podstawie rozkładu cen w katalogu:

\begin{itemize}
    \item \texttt{price\_low}: cena < 100 PLN -- akcesoria, kable, drobne peryferia
    \item \texttt{price\_medium}: 100 PLN $\leq$ cena < 500 PLN -- peryferia, komponenty
    \item \texttt{price\_high}: 500 PLN $\leq$ cena < 1500 PLN -- monitory, karty graficzne
    \item \texttt{price\_premium}: cena $\geq$ 1500 PLN -- laptopy, komputery, high-end
\end{itemize}

Dyskretyzacja eliminuje problem dużej wariancji cen i pozwala na porównywanie produktów z różnych kategorii cenowych.

\medskip
\noindent\textbf{Algorytm podobieństwa kosinusowego}

\medskip

Metoda \texttt{calculate\_product\_similarity()} implementuje podobieństwo kosinusowe dla wektorów rzadkich (sparse vectors):

\begin{equation}
\text{similarity}(p_1, p_2) = \frac{\sum_{f \in F_{1} \cap F_{2}} w_1(f) \cdot w_2(f)}{\sqrt{\sum_{f \in F_1} w_1(f)^2} \cdot \sqrt{\sum_{f \in F_2} w_2(f)^2}}
\end{equation}

gdzie $F_1$, $F_2$ to zbiory cech produktów $p_1$ i $p_2$, $w_i(f)$ to waga cechy $f$ dla produktu $p_i$.

\textbf{Implementacja dla wektorów rzadkich} w pseudokodzie:

\begin{lstlisting}[style=pseudocode]
FUNCTION calculate_similarity(features1, features2):
    common_features = intersection(keys(features1), keys(features2))
    dot_product = sum(features1[f] * features2[f] FOR f IN common_features)

    norm1 = sqrt(sum(v^2 FOR v IN values(features1)))
    norm2 = sqrt(sum(v^2 FOR v IN values(features2)))

    IF norm1 = 0 OR norm2 = 0 THEN
        RETURN 0.0
    END IF

    RETURN dot_product / (norm1 * norm2)
END FUNCTION
\end{lstlisting}

\textbf{Optymalizacja dla wektorów rzadkich}:

Zamiast tworzyć pełne wektory o długości równej liczbie wszystkich możliwych cech (potencjalnie tysiące), algorytm operuje na słownikach. Iloczyn skalarny wymaga iteracji tylko po cechach wspólnych (przecięcie zbiorów kluczy).

\medskip

\textbf{Próg podobieństwa}:

System zapisuje tylko podobieństwa większe niż 0.2 (20\%). Uzasadnienie:
\begin{itemize}
    \item Podobieństwo < 0.2 oznacza mniej niż 20\% wspólnych cech -- produkty są praktycznie różne
    \item Redukcja rozmiaru tabeli o 60-80\%
    \item Szybsze zapytania (mniej rekordów do przeszukania)
\end{itemize}

\medskip
\noindent\textbf{Generowanie macierzy podobieństw}

\medskip

Metoda \texttt{generate\_similarities\_for\_all\_products()} oblicza podobieństwa dla wszystkich par produktów:

\textbf{Etap 1: Prefetching danych}

Wykorzystujemy mechanizm \texttt{prefetch\_related()} frameworka Django dla kategorii, tagów i specyfikacji, redukując liczbę zapytań SQL z $O(n \times k)$ do $O(1)$ dla $n$ produktów z $k$ relacjami. Ta technika pobiera wszystkie powiązane obiekty w jednym zapytaniu SQL zamiast osobnego zapytania dla każdego produktu.

\textbf{Etap 2: Ekstrakcja cech}

Dla każdego produktu ekstrahujemy wektor cech i zapisujemy w słowniku, gdzie kluczem jest identyfikator produktu, a wartością jego wektor cech.

\textbf{Etap 3: Obliczenie podobieństw}

Dla każdej pary produktów $(p_i, p_j)$ gdzie $i < j$ obliczamy podobieństwo kosinusowe. Algorytm w pseudokodzie:

\begin{lstlisting}[style=pseudocode]
FOR EACH product1 IN products:
    FOR EACH product2 IN products[index(product1)+1:]:
        similarity = calculate_similarity(features[product1], features[product2])

        IF similarity > 0.2 THEN
            save_similarity(product1, product2, similarity)
            save_similarity(product2, product1, similarity)
        END IF
    END FOR
END FOR
\end{lstlisting}

Zapisywane są oba kierunki relacji (symetryczne), co umożliwia szybkie wyszukiwanie produktów podobnych do dowolnego produktu.

\textbf{Etap 4: Bulk insert}

Zapisywanie podobieństw odbywa się w partiach po 1000 rekordów przy użyciu mechanizmu \texttt{bulk\_create()}, który przyspiesza zapis 50-100x względem pojedynczych operacji INSERT.

\textbf{Etap 5: Cache}

Wynik generowania macierzy podobieństw jest cachowany na 2 godziny (7200 sekund), eliminując potrzebę ponownego obliczania przy każdym żądaniu.

\textbf{Złożoność obliczeniowa}:

Teoretyczna złożoność to $O(n^2)$ dla $n$ produktów (wszystkie pary). W praktyce ograniczamy liczbę porównań przez:
\begin{itemize}
    \item \texttt{max\_comparisons\_per\_product = 50} -- dla każdego produktu obliczamy podobieństwo do max 50 innych
    \item Wczesne odrzucanie produktów bez wspólnych kategorii
\end{itemize}

Dla katalogu 500 produktów:
\begin{itemize}
    \item Teoretycznie: $500 \times 499 / 2 = 124,750$ par
    \item Po optymalizacji: ~25,000 obliczonych podobieństw
    \item Po filtrowaniu (próg 0.2): ~4,000 zapisanych rekordów
\end{itemize}

\newpage

\textbf{Diagram sekwencji: Content-Based Filtering}

\begin{figure}[H]
  \centering
  \includegraphics[width=\textwidth]{images/contentBasedSequenceDiagram.png}
  \caption{Diagram sekwencji - Content-Based Filtering.}
  \label{fig:cbf_sequence}
\end{figure}

Proces generowania rekomendacji Content-Based Filtering rozpoczyna się w momencie gdy żądanie użytkownika trafia do endpointa API \texttt{/api/recommendations/content-based/?product\_id=\{id\}}. Backend, reprezentowany przez klasę \texttt{ContentBasedAPI}, wykonuje zapytanie do modelu \texttt{ProductSimilarity}, pobierając top 10 produktów podobnych do wskazanego, posortowanych według malejącego współczynnika \texttt{similarity\_score}. System sprawdza pamięć podręczną -- w przypadku trafienia (cache HIT) wynik zwracany jest natychmiast, eliminując konieczność dostępu do bazy danych. Backend zwraca odpowiedź HTTP 200 OK z listą rekomendacji w formacie JSON, zawierającą dla każdego produktu: identyfikator, nazwę, cenę oraz współczynnik podobieństwa. Frontend (komponent React) wyświetla otrzymane rekomendacje w sekcji "Similar Products" na stronie produktu.

System automatycznie identyfikuje produkty z wysokim współczynnikiem podobieństwa (powyżej 20\%) i zapisuje wyniki w pamięci podręcznej na 5 minut w celu optymalizacji wydajności.

\subsection*{5.2 Logika rozmyta w systemie rekomendacji}
\addcontentsline{toc}{subsection}{5.2 Logika rozmyta w systemie rekomendacji}

\medskip
\noindent\textbf{Architektura systemu Fuzzy Logic}

\medskip

System logiki rozmytej został zaimplementowany w module \texttt{fuzzy\_logic\_engine.py} i składa się z trzech klas:

\textbf{1. FuzzyMembershipFunctions} -- definiuje funkcje przynależności dla trzech zmiennych wejściowych:

\begin{itemize}
    \item \textbf{Cena} (price): cheap, medium, expensive -- funkcje trójkątne i trapezoidalne z progami dostosowanymi do katalogu e-commerce
    \item \textbf{Jakość} (quality/rating): low, medium, high -- bazuje na średniej ocenie produktu (1-5 gwiazdek)
    \item \textbf{Popularność} (popularity/view\_count): low, medium, high -- bazuje na liczbie zamówień produktu
\end{itemize}

\textbf{2. FuzzyUserProfile} -- buduje rozmyty profil użytkownika na podstawie:

\begin{itemize}
    \item Historii zakupów (dla zalogowanych użytkowników) -- analiza kategorii, średniej ceny
    \item Danych sesji (dla gości) -- ostatnio przeglądane produkty
    \item Profilu domyślnego (fallback) -- gdy brak danych
\end{itemize}

\textbf{3. SimpleFuzzyInference} -- silnik wnioskowania Mamdani z 6 regułami IF-THEN i metodą defuzzyfikacji średniej ważonej.

\medskip

\textbf{Przepływ danych}:

Proces rozpoczyna się od pobrania produktów dostępnych do ewaluacji oraz zbudowania rozmytego profilu użytkownika na podstawie historii zakupów. Dla każdego produktu w katalogu system przeprowadza fuzzyfikację (obliczenie stopni przynależności dla ceny, jakości i popularności), następnie ewaluuje 6 reguł rozmytych zdefiniowanych w bazie wiedzy, agreguje wyniki aktywnych reguł oraz wykonuje defuzzyfikację w celu uzyskania liczbowego wyniku \texttt{fuzzy\_score}. Po przetworzeniu wszystkich produktów system sortuje je według malejącego \texttt{fuzzy\_score} i zwraca top N rekomendacji użytkownikowi.

\medskip
\noindent\textbf{Funkcje przynależności -- szczegóły implementacji}

\medskip

\textbf{Funkcje przynależności dla ceny}

Klasa \texttt{FuzzyMembershipFunctions} definiuje trzy funkcje dla zmiennej "cena":

\textit{Funkcja "cheap"~(tania)} -- trójkątna/trapezoidalna:

\begin{equation}
\mu_{cheap}(price) = \begin{cases}
1.0 & \text{jeśli } price \leq 100 \\
\frac{500 - price}{400} & \text{jeśli } 100 < price < 500 \\
0.0 & \text{jeśli } price \geq 500
\end{cases}
\end{equation}

Interpretacja: Produkty poniżej 100 PLN są w pełni "tanie". Od 100 do 500 PLN stopień "taności"~maleje liniowo.

\textit{Funkcja "medium"~(średnia)} -- trapezoidalna:

\begin{equation}
\mu_{medium}(price) = \begin{cases}
0.0 & \text{jeśli } price < 300 \\
\frac{price - 300}{200} & \text{jeśli } 300 \leq price < 500 \\
1.0 & \text{jeśli } 500 \leq price \leq 1200 \\
\frac{1500 - price}{300} & \text{jeśli } 1200 < price < 1500 \\
0.0 & \text{jeśli } price \geq 1500
\end{cases}
\end{equation}

Interpretacja: Przedział $[500, 1200]$ ma pełną przynależność. Przejścia są płynne -- cena 400 PLN jest częściowo "tania"~i częściowo "średnia".

\textit{Funkcja "expensive"~(droga)}:

\begin{equation}
\mu_{expensive}(price) = \begin{cases}
0.0 & \text{jeśli } price \leq 1000 \\
\frac{price - 1000}{1000} & \text{jeśli } 1000 < price < 2000 \\
1.0 & \text{jeśli } price \geq 2000
\end{cases}
\end{equation}

\bigskip

\textbf{Funkcje przynależności dla jakości (rating)}

Oparte na średniej ocenie produktu (skala 1-5):

\begin{equation}
\mu_{\text{low}}(r) =
\begin{cases}
1.0, & r \leq 2.5 \\
\frac{3.5 - r}{3.5 - 2.5}, & 2.5 < r < 3.5 \\
0.0, & r \geq 3.5
\end{cases}
\end{equation}

\begin{equation}
\mu_{\text{medium}}(r) =
\begin{cases}
0.0, & r < 2.5 \\
\frac{r - 2.5}{3.5 - 2.5}, & 2.5 \leq r < 3.5 \\
1.0, & r = 3.5 \\
\frac{4.5 - r}{4.5 - 3.5}, & 3.5 < r < 4.5 \\
0.0, & r \geq 4.5
\end{cases}
\end{equation}

\begin{equation}
\mu_{\text{high}}(r) =
\begin{cases}
0.0, & r < 3.5 \\
\frac{r - 3.5}{4.5 - 3.5}, & 3.5 \leq r < 4.5 \\
1.0, & r \geq 4.5
\end{cases}
\end{equation}

\bigskip

Interpretacja: Produkty z oceną $\geq 4.5$ (maksymalna jakość) mają pełną przynależność do zbioru "high". Rating 3.5 jest punktem środkowym - produkt należy w równym stopniu do zbiorów rozmytych o sąsiednich kategoriach jakości. Przejścia są płynne, co odzwierciedla niepewność i subiektywność w ocenach użytkowników - produkt z ratingiem 4.0 jest częściowo "średni"~i częściowo "wysoki".

\bigskip

\textbf{Funkcje przynależności dla popularności (order\_count)}

Oparte na liczbie zamówień produktu:

\begin{equation}
\mu_{\text{low}}(v) =
\begin{cases}
1.0, & v \leq 2 \\
\frac{10 - v}{10 - 2}, & 2 < v < 10 \\
0.0, & v \geq 10
\end{cases}
\end{equation}

\begin{equation}
\mu_{\text{medium}}(v) =
\begin{cases}
0.0, & v < 2 \\
1.0, & 2 \leq v \leq 10 \\
\frac{30 - v}{30 - 10}, & 10 < v < 30 \\
0.0, & v \geq 30
\end{cases}
\end{equation}

\begin{equation}
\mu_{\text{high}}(v) =
\begin{cases}
0.0, & v < 10 \\
\frac{v - 10}{30 - 10}, & 10 \leq v < 30 \\
1.0, & v \geq 30
\end{cases}
\end{equation}

\bigskip

Interpretacja: Funkcja trapezoidalna dla "medium"~zapewnia stabilny przedział pełnej przynależności $[2, 10]$ zamówień - produkty w tym przedziale są typowymi artykułami o standardowej popularności. Produkt z 5 zamówieniami jest w pełni "średnio popularny", ale nie jest ani "niszowy"~(low $\leq 2$) ani "bestsellerem"~(high $\geq 30$). Trapez eliminuje oscylacje klasyfikacji dla produktów o podobnej liczbie zamówień, w przeciwieństwie do funkcji trójkątnych używanych dla zbiorów skrajnych.

\bigskip

\textbf{Wizualizacja funkcji przynależności}

Poniższe wykresy przedstawiają graficzną reprezentację zaimplementowanych funkcji przynależności dla trzech zmiennych wejściowych systemu rozmytego.

\begin{figure}[H]
  \centering
  \includegraphics[width=0.85\textwidth]{images/fuzzy_price_membership.png}
  \caption{Funkcje przynależności dla ceny produktu (cheap, medium, expensive). Oś X przedstawia cenę w PLN, oś Y stopień przynależności $\mu$ w przedziale [0, 1].}
  \label{fig:fuzzy_price}
\end{figure}

\begin{figure}[H]
  \centering
  \includegraphics[width=0.85\textwidth]{images/fuzzy_quality_membership.png}
  \caption{Funkcje przynależności dla jakości produktu (low, medium, high). Oś X przedstawia ocenę produktu w skali 1-5, oś Y stopień przynależności $\mu$.}
  \label{fig:fuzzy_quality}
\end{figure}

\begin{figure}[H]
  \centering
  \includegraphics[width=0.85\textwidth]{images/fuzzy_popularity_membership.png}
  \caption{Funkcje przynależności dla popularności produktu (low, medium, high). Oś X przedstawia liczbę zamówień, oś Y stopień przynależności $\mu$.}
  \label{fig:fuzzy_popularity}
\end{figure}

\medskip
\noindent\textbf{Rozmyty profil użytkownika}

\medskip

Klasa \texttt{FuzzyUserProfile} buduje profil preferencji użytkownika jako zbiory rozmyte. Jest to kluczowy element personalizacji rekomendacji.

\textbf{Dla zalogowanych użytkowników}:

\begin{enumerate}
    \item Pobranie historii zamówień z \texttt{prefetch\_related} dla powiązanych produktów i kategorii
    \item Zliczenie kategorii produktów w zamówieniach
    \item Obliczenie stopnia zainteresowania kategorią:
    \begin{equation}
    \mu_{category} = \frac{count_{category}}{total\_items}
    \end{equation}
    \item Obliczenie wrażliwości cenowej na podstawie średniej ceny zakupów
\end{enumerate}

\textbf{Wrażliwość cenowa} (price\_sensitivity):

\begin{equation}
\text{price\_sensitivity} = \begin{cases}
0.9 & \text{jeśli } avg\_price < 300 \text{ PLN (bardzo wrażliwy)} \\
0.6 & \text{jeśli } 300 \leq avg\_price < 700 \text{ (średnio wrażliwy)} \\
0.4 & \text{jeśli } 700 \leq avg\_price < 1500 \text{ (mało wrażliwy)} \\
0.2 & \text{jeśli } avg\_price \geq 1500 \text{ PLN (premium)}
\end{cases}
\end{equation}

Użytkownik kupujący średnio tanie produkty (avg < 300 PLN) ma wysoką wrażliwość cenową (0.9) -- system będzie promował tanie produkty. Użytkownik premium (avg > 1500 PLN) ma niską wrażliwość (0.2) -- system może rekomendować droższe produkty.

\textbf{Dopasowanie kategorii} -- metoda \texttt{fuzzy\_category\_match()}:

Dla każdej kategorii produktu system oblicza stopień dopasowania do profilu użytkownika:

\begin{equation}
\text{match} = 0.6 \cdot \text{similarity}(cat_{user}, cat_{product}) + 0.4 \cdot \mu_{interest}(cat_{user})
\end{equation}

gdzie \texttt{similarity} używa hierarchii kategorii. Przykład:
\begin{itemize}
    \item Kategoria użytkownika: "Electronics.Laptops"
    \item Kategoria produktu: "Electronics.Monitors"
    \item Podobieństwo hierarchiczne: 0.7 (wspólna kategoria nadrzędna "Electronics")
\end{itemize}

\textbf{Profil domyślny} (dla gości/nowych użytkowników):

Dla użytkowników bez historii zakupów system stosuje neutralny profil domyślny:
\begin{itemize}
    \item price\_sensitivity = 0.5 (neutralna wrażliwość cenowa)
    \item category\_preferences = \{\} (brak preferencji kategorii)
    \item quality\_preference = 0.7 (preferuje dobrą jakość)
    \item popularity\_preference = 0.5 (neutralna wobec popularności)
\end{itemize}

\medskip
\noindent\textbf{Baza reguł rozmytych}

\medskip

System wykorzystuje 6 reguł rozmytych typu Mamdani. Każda reguła ma formę IF-THEN z przypisaną wagą określającą jej ważność:

\medskip

\textbf{R1: High Quality Bargain} (waga: 0.9)

\begin{lstlisting}[style=pseudocode]
IF quality IS high AND (price IS cheap OR price IS medium)
THEN recommendation IS strong
\end{lstlisting}

Logika: Wysokiej jakości produkt w rozsądnej cenie to doskonała okazja. Najwyższa waga -- ta reguła najsilniej wpływa na wynik.

\medskip

\textbf{R2: Popular in Category} (waga: 0.7)

\begin{lstlisting}[style=pseudocode]
IF category_match IS high AND (popularity IS medium OR popularity IS high)
THEN recommendation IS medium-high
\end{lstlisting}

Logika: Popularny produkt z kategorii interesującej użytkownika. Popularność = walidacja społeczna.

\medskip

\textbf{R3: Price Sensitive Match} (waga: 0.6)

\begin{lstlisting}[style=pseudocode]
IF user.price_sensitivity > 0.6 AND price IS cheap
THEN recommendation IS moderate
\end{lstlisting}

Logika: Dla użytkowników wrażliwych cenowo (kupujących tanie produkty) promuj tanie opcje.

\medskip

\textbf{R4: Category Quality Match} (waga: 0.85)

\begin{lstlisting}[style=pseudocode]
IF category_match IS high AND (quality IS medium OR quality IS high)
THEN recommendation IS strong
\end{lstlisting}

Logika: Dopasowanie do kategorii + dobra jakość. Wysoka waga -- dopasowanie kategorii jest istotne.

\medskip

\textbf{R5: Premium Match} (waga: 0.8)

\begin{lstlisting}[style=pseudocode]
IF user.price_sensitivity < 0.4 AND price IS expensive AND quality IS high
THEN recommendation IS strong
\end{lstlisting}

Logika: Dla użytkowników premium (nieczułych cenowo) promuj drogie produkty wysokiej jakości.

\medskip

\textbf{R6: Quality-Price Balance} (waga: 0.75)

\begin{lstlisting}[style=pseudocode]
IF (quality IS high AND price IS reasonable) OR
   (quality IS medium AND price IS cheap)
THEN recommendation IS moderate
\end{lstlisting}

Logika: Dobry stosunek jakości do ceny -- "value for money".

\medskip
\noindent\textbf{Wnioskowanie i defuzzyfikacja}

\medskip

Metoda \texttt{evaluate\_product()} implementuje pełny cykl wnioskowania Mamdani:

\textbf{Krok 1: Fuzzyfikacja}

Dla każdej zmiennej wejściowej (cena, jakość, popularność) obliczane są stopnie przynależności do wszystkich zbiorów rozmytych. Wynikiem jest słownik zawierający 9 wartości przynależności:

\begin{itemize}
    \item Cena: $\mu_{cheap}$, $\mu_{medium}$, $\mu_{expensive}$
    \item Jakość: $\mu_{low}$, $\mu_{medium}$, $\mu_{high}$
    \item Popularność: $\mu_{low}$, $\mu_{medium}$, $\mu_{high}$
\end{itemize}

\textbf{Krok 2: Ewaluacja reguł}

Każda reguła jest ewaluowana za pomocą T-normy (minimum) dla operatora AND i T-conormy (maksimum) dla OR. Zgodnie z teorią zbiorów rozmytych \cite{zadeh1965fuzzy}:

\begin{equation}
\alpha_{R1} = \min(\mu_{quality\_high}, \max(\mu_{price\_cheap}, \mu_{price\_medium})) \cdot w_{R1}
\end{equation}

Dla każdej z 6 reguł obliczana jest jej aktywacja $\alpha_i$ poprzez aplikację odpowiednich operatorów rozmytych do wartości przynależności.

\bigskip

\textbf{Krok 3: Agregacja}

Wyniki reguł są agregowane. W uproszczonej implementacji wykorzystana została suma ważona (zamiast pełnej agregacji Mamdani):

\begin{equation}
\text{aggregated} = \sum_{i=1}^{6} \alpha_i
\end{equation}

\textbf{Krok 4: Defuzzyfikacja}

System używa uproszczonej metody średniej ważonej:

\begin{equation}
\text{fuzzy\_score} = \frac{\sum_{i=1}^{6} \alpha_i \cdot w_i}{\sum_{i=1}^{6} w_i}
\end{equation}

gdzie $\alpha_i$ to aktywacja reguły $i$, a $w_i$ to waga reguły.

Wagi reguł wynoszą: $w_{R1} = 0.9$, $w_{R2} = 0.7$, $w_{R3} = 0.6$, $w_{R4} = 0.85$, $w_{R5} = 0.8$, $w_{R6} = 0.75$. Suma wag wynosi 4.65, co zapewnia normalizację wyniku do przedziału $[0, 1]$.

\textbf{Wynik końcowy}:

Metoda zwraca słownik z:
\begin{itemize}
    \item \texttt{fuzzy\_score} -- wartość z przedziału $[0, 1]$ reprezentująca siłę rekomendacji
    \item \texttt{rule\_activations} -- słownik z aktywacją każdej reguły (dla debugowania)
    \item \texttt{category\_match} -- stopień dopasowania kategorii
    \item \texttt{price\_membership} -- przynależności cenowe (cheap, medium, expensive)
\end{itemize}

Algorytm Fuzzy Logic oferuje najwyższą interpretowalność - użytkownik może zobaczyć aktywację każdej reguły i zrozumieć, dlaczego produkt został polecony.

\newpage

\textbf{Diagram sekwencji: Fuzzy Logic}

\begin{figure}[H]
  \centering
  \includegraphics[width=\textwidth]{images/fuzzyLogicSequenceDiagram.png}
  \caption{Diagram sekwencji - Fuzzy Logic.}
  \label{fig:fuzzy_sequence}
\end{figure}

Proces generowania rekomendacji Fuzzy Logic rozpoczyna się gdy żądanie użytkownika trafia do endpointa API \texttt{/api/fuzzy-recommendations/}. Silnik logiki rozmytej (\texttt{FuzzyLogicEngine}) buduje rozmyty profil użytkownika (\texttt{FuzzyUserProfile}) na podstawie historii zakupów, ekstrahując preferencje cenowe, kategorialne oraz jakościowe. Dla każdego produktu w katalogu system przeprowadza proces fuzzyfikacji, obliczając stopnie przynależności do zbiorów rozmytych dla trzech wymiarów: ceny (cheap/medium/expensive), jakości (low/medium/high) oraz popularności (low/medium/high). Następnie system wykonuje wnioskowanie rozmyte -- ewaluuje 6 reguł IF-THEN zdefiniowanych w bazie wiedzy i oblicza aktywację każdej reguły na podstawie T-norm (minimum). Proces defuzzyfikacji agreguje wyniki wszystkich aktywnych reguł do jednej wartości \texttt{fuzzy\_score} metodą średniej ważonej. Backend zwraca listę produktów posortowanych według malejącego \texttt{fuzzy\_score}, wraz z dodatkowymi metadanymi \texttt{rule\_activations} zapewniającymi transparentność decyzji algorytmu. System buforuje wygenerowane rekomendacje w pamięci podręcznej na 5 minut w celu optymalizacji wydajności zapytań powtarzalnych.

\subsection*{5.3 Modele probabilistyczne -- Markov Chain i Naive Bayes}
\addcontentsline{toc}{subsection}{5.3 Modele probabilistyczne -- Markov Chain i Naive Bayes}

\medskip
\noindent\textbf{Architektura systemu probabilistycznego}

\medskip

System probabilistyczny składa się z trzech komponentów zaimplementowanych w \texttt{custom\_recommendation\_engine.py}:

\textbf{1. CustomMarkovChain} -- łańcuch Markowa pierwszego rzędu do predykcji sekwencji zakupowych kategorii produktów. Modeluje pytanie: "Jeśli użytkownik kupił produkt z kategorii A, jaka kategoria jest najbardziej prawdopodobna jako następna?"

\textbf{2. CustomNaiveBayes} -- naiwny klasyfikator Bayesa z wygładzaniem Laplace'a do:
\begin{itemize}
    \item Predykcji prawdopodobieństwa zakupu (will\_purchase / will\_not\_purchase)
    \item Predykcji ryzyka rezygnacji (will\_churn / will\_not\_churn)
\end{itemize}

\textbf{3. ProbabilisticRecommendationEngine} -- silnik łączący oba modele w jeden system rekomendacji z wagami: Markov (60\%) + Naive Bayes (40\%).

\medskip

\textbf{Przepływ danych}:

System rozpoczyna przetwarzanie od pobrania historii zamówień wszystkich użytkowników z bazy danych. Następnie buduje sekwencje kategorii produktowych dla każdego użytkownika (np. Electronics → Clothing → Books). Model Markowa jest trenowany na tych sekwencjach w celu wyuczenia macierzy przejść między kategoriami. Równolegle system ekstrahuje cechy behawioralne użytkowników (liczba zamówień, średnia wartość koszyka, czas od ostatniego zakupu) i trenuje klasyfikator Naive Bayesa na danych historycznych. W fazie predykcji łańcuch Markowa przewiduje następne najbardziej prawdopodobne kategorie zakupów, podczas gdy Naive Bayes ocenia prawdopodobieństwo zakupu oraz ryzyko odejścia klienta. Silnik rekomendacji agreguje wyniki obu modeli (60\% Markov, 40\% NB) i generuje finalne rekomendacje produktowe.

\medskip
\noindent\textbf{Łańcuch Markowa dla sekwencji zakupowych}

\medskip

Klasa \texttt{CustomMarkovChain} modeluje sekwencje zakupów użytkowników jako łańcuch Markowa pierwszego rzędu, gdzie stanami są kategorie produktów.

\textbf{Struktura danych}:

Łańcuch Markowa przechowuje:
\begin{itemize}
    \item \texttt{transitions} -- słownik słowników: \{stan: \{następny\_stan: licznik\}\}
    \item \texttt{states} -- zbiór wszystkich stanów (48 kategorii produktów)
    \item \texttt{total\_sequences} -- liczba sekwencji użytych do treningu
\end{itemize}

\textbf{Trening modelu}:

Dla każdej sekwencji kategorii zakupowych $[c_1, c_2, ..., c_n]$ algorytm iteruje po parach sąsiadujących stanów $(c_i, c_{i+1})$ i zwiększa licznik przejścia $T[c_i][c_{i+1}]$. Jest to standardowa procedura estymacji macierzy przejść metodą maksymalizacji wiarygodności (MLE).

\textbf{Normalizacja do prawdopodobieństw}:

Prawdopodobieństwo przejścia obliczane jest jako:

\begin{equation}
P(s_j | s_i) = \frac{T[s_i][s_j]}{\sum_{k} T[s_i][s_k]}
\end{equation}

\textbf{Predykcja}:

Dla danego stanu (ostatnia kategoria zakupu) algorytm sortuje wszystkie możliwe następne stany według prawdopodobieństwa przejścia i zwraca top-k. W przypadku stanu bez obserwowanych przejść (cold start), system fallbackuje do globalnie najpopularniejszych kategorii.

\textbf{Generowanie sekwencji}:

Metoda predict\_sequence() generuje sekwencję $n$ przewidywanych kategorii metodą zachłanną (greedy), wybierając w każdym kroku najbardziej prawdopodobny następny stan. Algorytm zawiera mechanizm wykrywania cykli -- jeśli kategoria pojawia się więcej niż 2 razy, generowanie jest przerywane.

\textbf{Rozkład stacjonarny} -- metoda \texttt{get\_stationary\_distribution()}:

Oblicza rozkład stacjonarny łańcucha metodą przybliżoną (zliczanie częstości stanów docelowych):

Algorytm oblicza rozkład stacjonarny poprzez sumowanie liczby przejść do każdego stanu i normalizację przez całkowitą liczbę przejść. Wynik reprezentuje długoterminowe prawdopodobieństwo znalezienia się użytkownika w danej kategorii.

\medskip
\noindent\textbf{Naiwny klasyfikator Bayesa}

\medskip

Klasa \texttt{CustomNaiveBayes} implementuje multinomialny Naive Bayes z wygładzaniem Laplace'a.

\medskip

\textbf{Cechy użytkownika} (features):

\begin{itemize}
    \item \texttt{total\_orders} -- łączna liczba zamówień (dyskretyzowana: 0-2, 3-5, 6-10, 11+)
    \item \texttt{avg\_order\_value} -- średnia wartość zamówienia (low, medium, high, premium)
    \item \texttt{days\_since\_last\_order} -- dni od ostatniego zamówienia (recent, moderate, old, very\_old)
    \item \texttt{favorite\_category} -- najczęściej kupowana kategoria
    \item \texttt{order\_frequency} -- częstość zamówień (rare, occasional, regular, frequent)
\end{itemize}

\textbf{Struktura danych}:

Model przechowuje:
\begin{itemize}
    \item \texttt{class\_priors} -- prawdopodobieństwa \textit{a priori} $P(C)$ dla każdej klasy
    \item \texttt{feature\_likelihoods} -- prawdopodobieństwa warunkowe $P(x_i | C)$
    \item \texttt{feature\_vocabularies} -- unikalne wartości każdej cechy (dla wygładzania Laplace'a)
\end{itemize}

\textbf{Trening modelu}:

Faza treningu obejmuje:
\begin{enumerate}
    \item Zliczenie wystąpień każdej klasy i obliczenie prawdopodobieństw \textit{a priori}: $P(C) = \frac{count(C)}{N}$
    \item Dla każdej próbki treningowej -- aktualizacja słowników cech dla odpowiedniej klasy
    \item Budowa słownika unikalnych wartości cech (vocabulary) potrzebnego do wygładzania Laplace'a
\end{enumerate}

\textbf{Predykcja}:

Predykcja wykorzystuje twierdzenie Bayesa w przestrzeni logarytmicznej (dla stabilności numerycznej):

\begin{equation}
\log P(C | X) = \log P(C) + \sum_{i=1}^{n} \log P(x_i | C)
\end{equation}

Wyniki są normalizowane przez funkcję softmax, aby uzyskać rozkład prawdopodobieństw sumujący się do 1.

\textbf{Wygładzanie Laplace'a}:

Dla cech niewidzianych podczas treningu stosowane jest wygładzanie Laplace'a, które zapobiega zerowaniu prawdopodobieństwa:

\begin{equation}
P(x_i | C) = \frac{count(x_i, C) + 1}{count(C) + |V|}
\end{equation}

gdzie $|V|$ to liczba unikalnych wartości cechy (rozmiar słownika).

\textbf{Ważność cech}:

Ważność cechy jest mierzona entropią rozkładu jej wartości w różnych klasach:

\begin{equation}
H(feature) = -\sum_{v \in V} P(v | C) \cdot \log_2 P(v | C)
\end{equation}

Wyższa entropia oznacza większą zdolność cechy do rozróżniania klas. Typowy ranking ważności cech dla predykcji zakupu: days\_since\_last\_order > total\_orders > avg\_order\_value > favorite\_category.

\medskip
\noindent\textbf{Integracja modeli -- ProbabilisticRecommendationEngine}

\medskip

Klasa \texttt{ProbabilisticRecommendationEngine} łączy oba modele w jeden system rekomendacyjny.

\textbf{Trening}:

System trenuje trzy komponenty:
\begin{enumerate}
    \item \textbf{Markov Chain} -- na sekwencjach kategorii z historii zamówień
    \item \textbf{Purchase NB} -- na cechach użytkowników z etykietami will\_purchase / will\_not\_purchase
    \item \textbf{Churn NB} -- na cechach użytkowników z etykietami will\_churn / will\_not\_churn
\end{enumerate}

\textbf{Predykcja zintegrowana}:

Algorytm generowania rekomendacji w pseudokodzie:

\begin{lstlisting}[style=pseudocode]
FUNCTION generate_recommendations(user, last_category, k=10):
    markov_predictions = Markov.predict_next(last_category, top=5)
    user_features = extract_features(user)
    p_purchase = NB_purchase.predict(user_features)["will_purchase"]

    products = get_products_from_categories(markov_predictions)

    FOR EACH product IN products:
        p_category = max(category_probability from Markov)
        score = 0.6 * p_category + 0.4 * p_purchase
        add(product, score) to recommendations
    END FOR

    RETURN top_k(recommendations, k)
END FUNCTION
\end{lstlisting}

Wagi agregacji (Markov 60\%, NB 40\%) zostały dobrane empirycznie -- Markov Chain lepiej przewiduje następną kategorię, podczas gdy Naive Bayes moduluje wynik na podstawie ogólnego prawdopodobieństwa zakupu użytkownika.

\medskip
\noindent\textbf{API probabilistyczne}

\medskip

System udostępnia dwa główne endpointy w \texttt{probabilistic\_views.py}:

\textbf{MarkovRecommendationsAPI} (\texttt{GET /api/markov-recommendations/}):

\begin{itemize}
    \item Trenuje modele na bieżących danych (10-15 sekund dla pełnego treningu)
    \item Przewiduje następne kategorie zakupów na podstawie ostatniego zamówienia użytkownika
    \item Zwraca top 6 produktów z przewidywanych kategorii
    \item Oblicza prawdopodobieństwo zakupu i oczekiwany czas do następnego zamówienia
\end{itemize}

\textbf{BayesianInsightsAPI} (\texttt{GET /api/bayesian-insights/}):

\begin{itemize}
    \item Preferencje kategorii użytkownika (z Markova)
    \item Ryzyko odejścia klienta (z Naive Bayes)
    \item Wzorce behawioralne (feature importance)
    \item Personalizowane rekomendacje
\end{itemize}

\textbf{Diagram sekwencji: Probabilistic Models}

\begin{figure}[H]
  \centering
  \includegraphics[width=\textwidth]{images/probabilisticMethodsSequenceDiagram.png}
  \caption{Diagram sekwencji - Probabilistic Models.}
  \label{fig:prob_sequence}
\end{figure}

Proces generowania rekomendacji probabilistycznych rozpoczyna się gdy żądanie użytkownika trafia do endpointa API \texttt{/api/markov-recommendations/}. System inicjalizuje i trenuje oba modele na aktualnych danych transakcyjnych: \texttt{CustomMarkovChain} buduje macierz przejść między kategoriami produktowymi na podstawie sekwencji zamówień, podczas gdy \texttt{CustomNaiveBayes} uczy klasyfikatorów prawdopodobieństwa zakupu oraz ryzyka odejścia klienta (churn). Łańcuch Markowa analizuje ostatnie zamówienie użytkownika i przewiduje następne najbardziej prawdopodobne kategorie zakupów wraz z prawdopodobieństwami przejść. Równolegle klasyfikator Naive Bayesa oblicza prawdopodobieństwo zakupu (\texttt{purchase\_probability}) oraz ryzyko odejścia klienta na podstawie cech behawioralnych użytkownika (liczba zamówień, średnia wartość koszyka, czas od ostatniego zakupu). Silnik rekomendacji probabilistycznych (\texttt{ProbabilisticRecommendationEngine}) agreguje wyniki obu modeli w proporcji 60\% wagi dla Markova i 40\% dla Bayesa, wybierając top K produktów o najwyższym prawdopodobieństwie zakupu. Backend zwraca odpowiedź zawierającą listę rekomendacji wraz z metadanymi biznesowymi: przewidywane kategorie (\texttt{predicted\_categories}), prawdopodobieństwo zakupu oraz estymowaną liczbę dni do następnego zamówienia (\texttt{expected\_days\_to\_next\_order}). System zapisuje wygenerowane predykcje do tabeli \texttt{PurchaseProbability} w celu późniejszej analizy trendów zakupowych oraz weryfikacji trafności modelu.

Modele probabilistyczne oferują unikalną wartość biznesową: predykcję przyszłych zakupów (Markov) oraz identyfikację użytkowników zagrożonych odejściem (Naive Bayes), co umożliwia proaktywne działania marketingowe.

\bigskip

W niniejszym rozdziale przedstawiono szczegółową implementację trzech metod rekomendacyjnych bez użycia gotowych bibliotek:

\textbf{Content-Based Filtering (CBF)} -- algorytm oparty na ważonych wektorach cech i podobieństwie cosinusowym, umożliwiający rekomendacje produktów na podstawie ich cech (kategoria, tagi, cena, słowa kluczowe). System efektywnie przetwarza 500 produktów, obliczając i przechowując ~4,000 par podobieństw powyżej progu 20\%.

\textbf{Fuzzy Logic} -- system rozmytego wnioskowania Mamdani z 6 regułami IF-THEN, uwzględniający czynniki ceny, jakości i popularności. Algorytm buduje rozmyty profil użytkownika na podstawie historii zakupów i generuje spersonalizowane rekomendacje z wyjaśnieniem aktywacji reguł.

\textbf{Modele probabilistyczne} -- łańcuch Markowa pierwszego rzędu (predykcja sekwencji zakupowych kategorii) oraz klasyfikator Naive Bayes (prawdopodobieństwo zakupu i odejście klienta). System wykorzystuje historię zamówień do budowy macierzy przejść i modeli predykcyjnych.

Wszystkie trzy metody zostały zintegrowane z systemem Django/React, oferują API REST oraz zaawansowane panele debugowania dla administratorów. Praktyczne funkcjonowanie systemu, interfejsy użytkownika oraz wyniki ewaluacji zostały przedstawione w kolejnych rozdziałach.

\newpage

\section*{Rozdzia\l{} 6}
\addcontentsline{toc}{section}{Rozdział 6: Funkcjonowanie systemu rekomendacji w praktyce}
\section*{Funkcjonowanie systemu rekomendacji w praktyce}

Niniejszy rozdział prezentuje praktyczne aspekty działania zaimplementowanych algorytmów rekomendacyjnych, obejmując interfejsy użytkownika, panele administracyjne oraz panele debugowania. Przedstawiono rzeczywiste przypadki użycia systemu oraz sposób prezentacji rekomendacji użytkownikom końcowym i administratorom.

\subsection*{6.1 Przegląd interfejsu użytkownika}
\addcontentsline{toc}{subsection}{6.1 Przegląd interfejsu użytkownika}

Niniejszy rozdział przedstawia funkcjonowanie systemu rekomendacji z perspektywy użytkownika końcowego oraz administratora. System oferuje trzy komplementarne metody rekomendacyjne: Content-Based Filtering (rekomendacje oparte na podobieństwie cech produktów), Fuzzy Logic (system rozmytego wnioskowania z interpretowalnymi regułami) oraz modele probabilistyczne (łańcuch Markowa i naiwny klasyfikator Bayesa dla predykcji sekwencji zakupowych).

Rekomendacje są wyświetlane użytkownikom w trzech głównych lokalizacjach interfejsu:

\begin{itemize}
\item \textbf{Strona główna sklepu} - sekcja "Recommended For You" prezentująca spersonalizowane sugestie na podstawie aktywnego algorytmu wybranego przez administratora
\item \textbf{Strona produktu} - sekcja "Similar Products" z produktami podobnymi do aktualnie przeglądanego
\item \textbf{Panel klienta} - dedykowane zakładki z rekomendacjami według każdej z trzech metod, umożliwiające użytkownikowi porównanie działania algorytmów
\end{itemize}

Administrator systemu dysponuje zaawansowanymi panelami debugowania dla każdej metody, umożliwiającymi monitorowanie działania algorytmów, analizę jakości rekomendacji oraz konfigurację parametrów systemu. Przepływ danych w systemie został szczegółowo opisany w rozdziale 4.4 (Projekt architektury systemu).

\subsection*{6.2 Rekomendacje Content-Based Filtering}
\addcontentsline{toc}{subsection}{6.2 Rekomendacje Content-Based Filtering}

\subsubsection*{6.2.1 Widok użytkownika}

Metoda Content-Based Filtering jest wykorzystywana jako jedna z opcji sortowania produktów na stronie głównej sklepu. Administrator może wybrać algorytm CBF w ustawieniach systemu rekomendacji, co powoduje wyświetlanie produktów podobnych do tych, które użytkownik wcześniej przeglądał lub kupił.

\begin{figure}[h!]
  \centering
  \includegraphics[width=0.92\textwidth]{images/recomendationSystem.png}
  \caption{Rekomendacje Content-Based Filtering wyświetlane użytkownikowi na stronie głównej.}
  \label{fig:cbf_recommendations}
\end{figure}

\newpage

Rysunek \ref{fig:cbf_recommendations} przedstawia sekcję "Recommended For You (Content-Based)"~wyświetlaną na stronie głównej aplikacji po wyborze algorytmu CBF przez administratora. System prezentuje 4 produkty podobne do wcześniej przeglądanych lub zakupionych artykułów, wybrane na podstawie analizy cech (kategoria, tagi, cena, słowa kluczowe). Każdy produkt zawiera zdjęcie, nazwę, cenę oraz przycisk dodania do koszyka.

Rekomendacje CBF są również dostępne w panelu klienta, gdzie użytkownik może zobaczyć produkty podobne do swoich poprzednich zakupów. System automatycznie identyfikuje produkty z wysokim współczynnikiem podobieństwa (powyżej 20\%) i prezentuje je w sekcji spersonalizowanych rekomendacji. Szczegółowy opis przepływu danych w systemie CBF przedstawiono w rozdziale 5.1.

\newpage

\subsubsection*{6.2.2 Panel administracyjny}

System oferuje zaawansowany panel debugowania dostępny przez endpoint \texttt{/api/content-based-debug/}. Panel prezentuje:

\textbf{Widok ogólny (bez parametru product\_id)}:

\begin{itemize}
    \item \textbf{Szczegóły algorytmu}: nazwa, metoda (Weighted Feature Vectors + Cosine Similarity), status
    \item \textbf{Wagi cech}: category (40\%), tag (30\%), price (20\%), keywords (10\%)
    \item \textbf{Statystyki bazy danych}: liczba produktów, zapisanych podobieństw, procent pokrycia
    \item \textbf{Status pamięci podręcznej}: HIT/MISS, czas wygaśnięcia
    \item \textbf{Top 10 podobieństw}: produkty o najwyższym podobieństwie w systemie
\end{itemize}

\begin{figure}[h!]
  \centering
  \includegraphics[width=0.92\textwidth]{images/contentBasedAdminDebug1.png}
  \caption{Panel debugowania Content-Based Filtering.}
  \label{fig:cbf_debug1}
\end{figure}

\textbf{Widok szczegółowy (z parametrem product\_id)}:

Dla konkretnego produktu panel pokazuje:

\begin{itemize}
    \item Wektor cech produktu z wagami (słownik feature → weight)
    \item Top 10 produktów podobnych z szczegółami obliczeń
    \item Wzór matematyczny dla każdej pary: $\frac{dot\_product}{norm_1 \times norm_2}$
    \item Cechy wspólne między produktami
    \item Breakdown podobieństwa: ile procent z kategorii, ile z tagów, ile z ceny, ile z keywords
\end{itemize}

\begin{figure}[h!]
  \centering
  \includegraphics[width=0.92\textwidth]{images/contentBasedAdminDebug2.png}
  \caption{CBF - szczegółowa analiza podobieństwa produktu.}
  \label{fig:cbf_debug2}
\end{figure}

Panel umożliwia administratorowi:
\begin{itemize}
    \item Monitorowanie pokrycia rekomendacji (ile produktów ma podobieństwa)
    \item Identyfikację produktów bez podobieństw (słabo opisane metadane)
    \item Walidację działania wag (czy kategorie dominują prawidłowo)
    \item Ręczne wyzwalanie przeliczenia macierzy
\end{itemize}

\newpage

\subsection*{6.3 Rekomendacje Fuzzy Logic}
\addcontentsline{toc}{subsection}{6.3 Rekomendacje Fuzzy Logic}

\subsubsection*{6.3.1 Widok użytkownika}

System Fuzzy Logic jest dostępny dla użytkowników w dedykowanej zakładce panelu klienta. Interfejs składa się z trzech podzakładek: "Fuzzy Recommendations"~(rekomendacje produktów), "User Profile"~(profil użytkownika) oraz "Fuzzy Rules"~(reguły wnioskowania). Szczegółowy opis przepływu danych w systemie Fuzzy Logic przedstawiono w rozdziale 5.2.

\newpage

\textbf{Zakładka "Fuzzy Recommendations"}

\begin{figure}[h!]
  \centering
  \includegraphics[width=0.92\textwidth]{images/fuzzyLogicClient1.png}
  \caption{Panel klienta - rekomendacje Fuzzy Logic.}
  \label{fig:fuzzy_client}
\end{figure}

Rysunek \ref{fig:fuzzy_client} przedstawia główny widok rekomendacji Fuzzy Logic. U góry strony znajdują się trzy zakładki nawigacyjne: "Fuzzy Recommendations", "User Profile"~oraz "Fuzzy Rules", umożliwiające przełączanie między widokami. Sekcja "Recommended Products"~wyświetla 4 produkty wybrane przez silnik wnioskowania rozmytego Mamdani. Każdy produkt zawiera:

\begin{itemize}
    \item \textbf{Fuzzy Score}: całkowity wynik rekomendacji wyrażony w procentach (np. 53.9\%, 53.7\%, 53.6\%) - wynik agregacji wszystkich 6 reguł rozmytych z uwzględnieniem ich wag (suma ważona aktywacji reguł)
    \item \textbf{Category Match}: stopień dopasowania kategorii produktu do ulubionych kategorii użytkownika wyrażony w procentach (np. 63.3\%, 62.6\%, 62.4\%)
    \item \textbf{View Rule Activations}: niebieski przycisk umożliwiający podgląd szczegółowej aktywacji wszystkich 6 reguł IF-THEN dla danego produktu wraz z wyjaśnieniem, dlaczego produkt został polecony
\end{itemize}

Wszystkie pokazane produkty mają podobny Fuzzy Score (53-54\%), co oznacza, że system wnioskowania Mamdani ocenił je jako równie dopasowane do profilu użytkownika. Category Match (62-63\%) wskazuje na wysoki stopień dopasowania kategorii produktów do historycznych preferencji zakupowych użytkownika.

\medskip

\textbf{Zakładka "User Profile"}

\begin{figure}[h!]
  \centering
  \includegraphics[width=0.92\textwidth]{images/fuzzyLogicClient2.png}
  \caption{Profil użytkownika Fuzzy Logic.}
  \label{fig:fuzzy_client2}
\end{figure}

Zakładka "User Profile"~(Rysunek \ref{fig:fuzzy_client2}) prezentuje rozmyty profil użytkownika zbudowany przez klasę \texttt{FuzzyUserProfile} na podstawie historii zakupów. Profil jest wykorzystywany przez system wnioskowania Mamdani do obliczania spersonalizowanych rekomendacji. Wyświetlane informacje:

\begin{itemize}
    \item \textbf{Profile Type}: typ profilu użytkownika\\
    -- \textit{authenticated} - użytkownik zalogowany z pełną historią zakupów (jak na zrzucie ekranu)\\
    -- \textit{guest} - użytkownik bez historii zakupów (profil domyślny z globalnych statystyk)

    \item \textbf{Price Sensitivity}: wrażliwość cenowa użytkownika wyrażona w procentach\\
    -- Przykład: 60\% oznacza umiarkowaną wrażliwość cenową ("Moderate price sensitivity")\\
    -- Wartość obliczana na podstawie średniej ceny zakupionych produktów względem średniej w systemie\\
    -- Wpływa na aktywację reguł R3 (Price Sensitive Match) i R5 (Premium Match)

    \item \textbf{Favorite Categories}: lista ulubionych kategorii produktowych z procentowym udziałem w historii zakupów\\
\end{itemize}

System automatycznie aktualizuje profil po każdym nowym zamówieniu użytkownika, co pozwala na dynamiczną adaptację rekomendacji do zmieniających się preferencji. Kategorie są zapisywane w formacie hierarchicznym \\
(kategoria\_główna.podkategoria) i przechowywane w polu JSON modelu \\
\texttt{RecommendationSettings}.

\medskip

\textbf{Zakładka "Fuzzy Rules"}

\begin{figure}[h!]
  \centering
  \includegraphics[width=0.92\textwidth]{images/fuzzyLogicClient3.png}
  \caption{Reguły wnioskowania Fuzzy Logic.}
  \label{fig:fuzzy_client3}
\end{figure}

Zakładka "Fuzzy Rules"~(Rysunek \ref{fig:fuzzy_client3}) prezentuje szczegółowy opis 6 reguł wnioskowania IF-THEN wykorzystywanych przez system Mamdani. Strona wyświetla nagłówek "How Recommendations Are Made"~oraz wyjaśnienie: "Our system uses 6 intelligent rules to find the best products for you. Each rule looks at different aspects like price, quality, and category match."~Każda reguła jest opisana w zrozumiały dla użytkownika sposób.

System używa tych reguł do obliczenia końcowego Fuzzy Score dla każdego produktu zgodnie ze wzorem agregacji przedstawionym w rozdziale 5.2.5. Wagi reguł (0.6-0.9) określają ich wpływ na ostateczną rekomendację - reguła R1 (0.9) ma największy wpływ, podczas gdy R3 (0.6) najmniejszy. Przycisk "View Rule Activations"~w zakładce "Fuzzy Recommendations"~pokazuje, które reguły zostały aktywowane dla konkretnego produktu, z jaką siłą (stopień aktywacji $\alpha_i$) oraz jaki był ich wkład w końcowy wynik.

\medskip

\noindent\textbf{Wyszukiwanie tolerujące błędy (Fuzzy Search)}

\medskip

Wyszukiwanie tolerujące błędy wykorzystuje algorytm odległości Levensteina do wyszukiwania produktów z tolerancją na literówki i błędy pisowni. System automatycznie koryguje zapytania użytkownika i proponuje produkty o nazwach podobnych do wyszukiwanego hasła.

\begin{figure}[h!]
  \centering
  \includegraphics[width=0.92\textwidth]{images/fuzzySearch1.jpg}
  \caption{Wyszukiwarka rozmyta (Fuzzy Search).}
  \label{fig:fuzzy_search}
\end{figure}

Rysunek \ref{fig:fuzzy_search} przedstawia interfejs wyszukiwarki rozmytej z przykładowym zapytaniem "laptop". Użytkownik kontroluje tolerancję wyszukiwania za pomocą suwaka "Fuzzy Threshold"~(wartość 0.3 oznacza próg 30\% podobieństwa). System zwraca produkty z metryką "Fuzzy Match"~(np. 86\%, 45\%), wskazującą stopień dopasowania nazwy do zapytania. Wyszukiwarka znajduje produkty zawierające słowo "Laptop"~(86\% match) oraz bardziej odległe jak "Desktop Hub"~(45\% match), co demonstruje tolerancję na nieprecyzyjne zapytania.

Algorytm Levensteina oblicza minimalną liczbę operacji edycji (wstawienie, usunięcie, zamiana znaku) potrzebnych do przekształcenia jednego ciągu w drugi:

\begin{equation}
lev(a,b) = \begin{cases}
|a| & \text{jeśli } |b| = 0 \\
|b| & \text{jeśli } |a| = 0 \\
lev(tail(a), tail(b)) & \text{jeśli } a[0] = b[0] \\
1 + \min \begin{cases}
lev(tail(a), b) \\
lev(a, tail(b)) \\
lev(tail(a), tail(b))
\end{cases} & \text{w przeciwnym wypadku}
\end{cases}
\end{equation}

System wyszukiwania zwraca produkty, dla których odległość Levensteina między zapytaniem a nazwą produktu jest mniejsza niż ustalony próg (domyślnie 0.5).

\newpage

\subsubsection*{6.3.2 Panel administracyjny}

Panel debugowania dostępny przez endpoint \texttt{/api/fuzzy-debug/} prezentuje szczegółowe informacje o systemie wnioskowania rozmytego:

\textbf{Widok ogólny (bez parametru product\_id)}:

\begin{itemize}
    \item \textbf{Szczegóły algorytmu}: metoda (Mamdani Fuzzy Inference), liczba reguł (6), T-norma (min), T-conorma (max)
    \item \textbf{Funkcje przynależności}: definicje dla price, quality, popularity z progami
    \item \textbf{Statystyki}: średni fuzzy\_score, rozkład wyników, aktywacja reguł
    \item \textbf{Profil użytkownika}: jeśli podany user\_id — szczegóły profilu rozmytego
\end{itemize}

\begin{figure}[h!]
  \centering
  \includegraphics[width=0.92\textwidth]{images/fuzzyLogicAdminDebug1.png}
  \caption{Panel debugowania Fuzzy Logic.}
  \label{fig:fuzzy_debug1}
\end{figure}

\textbf{Widok produktu} (z parametrem product\_id):

\begin{itemize}
    \item Wartości fuzzyfikacji (wszystkie przynależności)
    \item Aktywacja każdej z 6 reguł z wyjaśnieniem
    \item Obliczenie końcowe z breakdownem
    \item Porównanie z innymi produktami
\end{itemize}

\begin{figure}[h!]
  \centering
  \includegraphics[width=0.92\textwidth]{images/fuzzyLogicAdminDebug2.png}
  \caption{Fuzzy Logic - ewaluacja produktu.}
  \label{fig:fuzzy_debug2}
\end{figure}

Panel umożliwia administratorowi monitorowanie działania systemu rozmytego, weryfikację poprawności funkcji przynależności oraz analizę aktywacji reguł dla konkretnych produktów. Szczegółowy opis algorytmu Mamdani przedstawiono w rozdziale 5.2.

\newpage

\subsection*{6.4 Rekomendacje Probabilistic Models}
\addcontentsline{toc}{subsection}{6.4 Rekomendacje Probabilistic Models}

\subsubsection*{6.4.1 Widok użytkownika}

Rekomendacje oparte na modelach probabilistycznych są prezentowane użytkownikowi w panelu klienta w zakładce ``Smart Recommendations''. Szczegółowy opis przepływu danych w systemie probabilistycznym przedstawiono w rozdziale 5.3. System wyświetla dwie podzakładki:
\begin{itemize}
    \item \textbf{Next Purchase (Markov)}: produkty z kategorii przewidywanych przez łańcuch Markowa jako najbardziej prawdopodobne do zakupu
    \item \textbf{Behavior Insights (Bayesian)}: analiza zachowań zakupowych użytkownika z wykorzystaniem Naive Bayes
\end{itemize}

\begin{figure}[h!]
  \centering
  \includegraphics[width=0.92\textwidth]{images/probabilisticMethodsClient1.png}
  \caption{Zakładka "Next Purchase (Markov)".}
  \label{fig:prob_client1}
\end{figure}

Zakładka "Next Purchase (Markov)"~prezentuje:
\begin{itemize}
    \item \textbf{Next Purchase Probability}: prawdopodobieństwo zakupu w ciągu 30 dni (np. 50\%)
    \item \textbf{Expected Days Until Next Purchase}: przewidywany czas do następnego zakupu
    \item \textbf{Likely Next Products}: lista produktów z najwyższym Prediction Score (np. Imou Cruiser 2 5MP: 13\%, A4Tech HD PK-910P: 13\%)
    \item \textbf{Your Shopping Patterns}: najczęstsza sekwencja zakupów i długość cyklu (np. power.strips → laptop.hubs → office.accessories, 10 products per cycle)
\end{itemize}

\begin{figure}[h!]
  \centering
  \includegraphics[width=0.92\textwidth]{images/probabilisticMethodsClient2.png}
  \caption{Zakładka "Behavior Insights (Bayesian)".}
  \label{fig:prob_client2}
\end{figure}

Zakładka ``Behavior Insights (Bayesian)'' wykorzystuje model Naive Bayes do analizy preferencji zakupowych:
\begin{itemize}
    \item \textbf{Purchase Likelihood}: wykres słupkowy prawdopodobieństwa zakupu dla każdej kategorii
    \item Kategorie z najwyższym prawdopodobieństwem: electronics.phones (10\%), power.strips (9\%), accessories.cables (9\%), office.accessories (9\%)
    \item Model uczy się na podstawie historii zakupów wszystkich użytkowników i tworzy profil behawioralny
\end{itemize}

\begin{figure}[h!]
  \centering
  \includegraphics[width=0.92\textwidth]{images/probabilisticMethodsClient3.png}
  \caption{Zakładka "Churn Risk Analysis".}
  \label{fig:prob_client3}
\end{figure}

Zakładka ``Churn Risk Analysis'' (Rysunek \ref{fig:prob_client3}) prezentuje:
\begin{itemize}
    \item \textbf{Churn Risk}: poziom ryzyka rezygnacji klienta (np. 25\% - LOW RISK)
    \item \textbf{Shopping Behavior Analysis}: analiza wzorców zakupowych użytkownika
    \item \textbf{Personalized Suggestions}: spersonalizowane sugestie produktów
\end{itemize}

\newpage

\subsubsection*{6.4.2 Panel administracyjny}

Panel administracyjny systemu rekomendacji probabilistycznych prezentuje zaawansowane analizy dla administratora. System oferuje dwa główne widoki: panel debugowania (dostępny przez endpoint \texttt{/api/probabilistic-debug/}) oraz panele administracyjne Markov Analysis i Bayesian Analysis.

\medskip
\noindent\textbf{Panel debugowania modeli probabilistycznych}

\begin{figure}[H]
  \centering
  \includegraphics[width=0.92\textwidth]{images/probabilisticMethodsAdminDebug1.png}
  \caption{Panel debugowania - statystyki Markov Chain.}
  \label{fig:prob_debug1}
\end{figure}

\textbf{Statystyki Markov Chain} (Rysunek \ref{fig:prob_debug1}):
\begin{itemize}
    \item Rząd łańcucha (Order): 1 (first-order Markov Chain)
    \item Liczba stanów (kategorii): 48
    \item Liczba przejść (transitions): 48
    \item Macierz przejść z prawdopodobieństwami przejść między kategoriami
\end{itemize}

\begin{figure}[h!]
  \centering
  \includegraphics[width=0.92\textwidth]{images/probabilisticMethodsAdminDebug2.png}
  \caption{Panel debugowania - statystyki Naive Bayes.}
  \label{fig:prob_debug2}
\end{figure}

\textbf{Statystyki Naive Bayes} (Rysunek \ref{fig:prob_debug2}):

\textit{Purchase Prediction}:
\begin{itemize}
    \item Trained: Yes
    \item Number of Features: 3 (liczba zamówień, średnia wartość koszyka, czas od ostatniego zakupu)
    \item Classes: will\_not\_purchase
    \item Class Priors: will\_not\_purchase = 1.0
\end{itemize}

\textit{Churn Prediction}:
\begin{itemize}
    \item Trained: Yes
    \item Number of Features: 3
    \item Classes: will\_churn, will\_not\_churn
    \item Class Priors: will\_churn = 0.95, will\_not\_churn = 0.05
\end{itemize}

\medskip
\noindent\textbf{Panele administracyjne - analiza biznesowa}

\begin{figure}[h!]
  \centering
  \includegraphics[width=0.92\textwidth]{images/probabilisticMethodsAdmin1.png}
  \caption{Panel administracyjny - Markov Analysis.}
  \label{fig:prob_admin1}
\end{figure}

\newpage

Rysunek \ref{fig:prob_admin1} prezentuje panel ``Markov Analysis'' zawierający:
\begin{itemize}
    \item \textbf{Markov Chain Analysis}: ogólne statystyki łańcucha Markowa (total predicted units, forecast period, products analyzed, average daily forecast)
    \item \textbf{Sales Forecast}: wykres prognozy sprzedaży w czasie na podstawie macierzy przejść łańcucha Markowa. Prognoza jest wyznaczana poprzez symulację sekwencji zakupowych: algorytm rozpoczyna od bieżącego rozkładu prawdopodobieństw kategorii produktowych, następnie iteracyjnie mnoży go przez macierz przejść, przewidując prawdopodobieństwo zakupu w każdej kategorii dla kolejnych okresów czasowych (dni/tygodni). Łączna sprzedaż jest sumą predykcji dla wszystkich kategorii z uwzględnieniem średniej wartości produktu w każdej kategorii.
    \item \textbf{Detailed Forecast}: tabela z szczegółowymi predykcjami dla poszczególnych okresów
\end{itemize}

\begin{figure}[h!]
  \centering
  \includegraphics[width=0.92\textwidth]{images/probabilisticMethodsAdmin2.png}
  \caption{Panel administracyjny - Bayesian Analysis.}
  \label{fig:prob_admin2}
\end{figure}

\newpage

Rysunek \ref{fig:prob_admin2} prezentuje panel ``Bayesian Analysis'' zawierający zaawansowane narzędzia analityczne dla administratora. Panel umożliwia analizę oczekiwanego popytu na produkty, identyfikację kategorii produktowych preferowanych przez użytkowników oraz monitorowanie metryk wydajności modelu Naive Bayes. System generuje rekomendacje dotyczące poziomu zapasów oraz identyfikuje produkty wymagające uzupełnienia magazynu na podstawie predykcji probabilistycznych.

\bigskip

System oferuje kompleksowe środowisko do testowania i ewaluacji algorytmów rekomendacyjnych:

\begin{itemize}
    \item \textbf{Dla użytkowników}: Spersonalizowane rekomendacje produktów, analiza wzorców zakupowych, predykcje odejścia klienta i prawdopodobieństwa zakupu
    \item \textbf{Dla administratorów}: Panele zarządzania algorytmami, monitorowanie wydajności, prognozy sprzedaży, analiza popytu
    \item \textbf{Dla deweloperów}: Zaawansowane panele debugowania z szczegółowymi informacjami o obliczeniach, aktywacji reguł, wagach i podobieństwach
\end{itemize}

Praktyczne testy wykazały, że system efektywnie obsługuje 500 produktów, 20 użytkowników oraz 265 zamówień, wydajnie generując rekomendacje w czasie rzeczywistym dla wszystkich trzech algorytmów.

\newpage

\section*{Rozdzia\l{} 7}
\addcontentsline{toc}{section}{Rozdział 7: Porównanie i ewaluacja metod rekomendacyjnych}
\section*{Porównanie i ewaluacja metod rekomendacyjnych}

Rozdział przedstawia praktyczne porównanie sześciu metod rekomendacyjnych zaimplementowanych w aplikacji we współpracy dwuosobowej. Analiza opiera się na wynikach testów przeprowadzonych na rzeczywistych danych: 500 produktów w 48 kategoriach, 20 użytkowników oraz historia transakcji (265 zamówień, 569 pozycji) zgromadzona w bazie PostgreSQL. Ewaluacja koncentruje się na trzech parach metod o komplementarnym charakterze: Content-Based Filtering vs Collaborative Filtering (podejście atrybutowe vs behawioralne), Fuzzy Logic vs Sentiment Analysis (personalizacja cenowa vs analiza opinii), oraz modele probabilistyczne vs Apriori (sekwencje czasowe vs współzakupy).

\medskip
\noindent\textbf{Metodyka badania i środowisko testowe}

\medskip

Testy przeprowadzono w środowisku Docker Compose z trzema kontenerami: frontend (React 18), backend (Django 5.1.4), oraz baza danych (PostgreSQL 14). Dane testowe zostały wygenerowane komendą \texttt{python manage.py seed}, która wypełniła system produktami z kategorii elektroniki komputerowej (laptopy, peryferia, komponenty, akcesoria) oraz utworzyła 20 kont użytkowników z symulowaną historią zakupów. Każda metoda była testowana w rzeczywistym środowisku aplikacji z włączonym mechanizmem pamięci podręcznej Django zgodnie z konfiguracją produkcyjną (cache timeout: 5-120 minut w zależności od typu danych).

Metryki ewaluacji obejmowały: (1) \textbf{Pokrycie katalogu} - procent produktów, dla których metoda generuje rekomendacje, (2) \textbf{Czas wykonania} - średni czas generowania rekomendacji obserwowany w panelach administratora i klienta podczas testów manualnych, (3) \textbf{Trafność merytoryczna} - jakościowa ocena poprawności rekomendacji na podstawie powiązań między produktami (komplementarność, podobieństwo kategorii, wzorce zakupowe), (4) \textbf{Interpretowalność} - możliwość wyjaśnienia użytkownikowi dlaczego dany produkt został zaproponowany.

\medskip
\noindent\textbf{Krótka charakterystyka metod współautora projektu}

\medskip

\textbf{Collaborative Filtering (metoda współautora)} analizuje historię zakupów użytkowników i buduje macierz podobieństw produktów używając algorytmu Adjusted Cosine Similarity z centrowaniem średniej użytkownika. Metoda wykrywa powiązania które nie są oczywiste na podstawie samych cech produktów. Jeśli użytkownicy kupujący aparaty fotograficzne często kupują też plecaki turystyczne, filtracja kolaboratywna połączy te produkty mimo różnych kategorii. Wymaga jednak dużo danych historycznych aby działać efektywnie i napotyka problem zimnego startu dla nowych produktów bez historii zakupów. Wyniki podobieństw są zapisywane w modelu \texttt{ProductSimilarity} z oznaczeniem typu \texttt{collaborative}. Metoda jest szczególnie skuteczna w odkrywaniu rekomendacji międzykategorialnych (ang. \textit{cross-category recommendations}), które zwiększają wartość koszyka zakupowego.

\textbf{Sentiment Analysis (metoda współautora)} analizuje opinie użytkowników z pięciu źródeł: treść opinii tekstowych, oceny gwiazdkowe (1-5), opinie zweryfikowane (potwierdzony zakup), opinie administratora oraz szczegółowe parametry produktu. System wykorzystuje dwa leksykony sentymentu: Opinion Lexicon (Hu \& Liu 2004) oraz AFINN-165 (Nielsen 2011) do przypisywania wag emocjonalnych słowom. Agregacja wieloźródłowa łączy wszystkie sygnały w jeden wskaźnik \texttt{sentiment\_score} (zakres -1 do +1), który określa ogólny sentyment wobec produktu. Metoda jest uniwersalna (działa tak samo dla każdego użytkownika) i nie wymaga personalizacji. Głównym ograniczeniem jest wrażliwość na negację, ironię oraz sarkazm w tekście opinii. Wyniki są zapisywane w tabeli \texttt{method\_product\_sentiment\_summary} i wykorzystywane do priorytetyzacji produktów z pozytywnymi opiniami.

\textbf{Apriori (metoda współautora)} odkrywa reguły asocjacyjne typu „klienci kupujący produkt A często kupują również produkt B". Algorytm analizuje koszyki zakupowe i identyfikuje itemsety (zbiory produktów) pojawiające się razem częściej niż przypadkowo. Wykorzystuje metryki support (wsparcie - jak często itemset występuje), confidence (pewność - prawdopodobieństwo że B jest kupowane gdy kupiono A) oraz lift (współczynnik przewyższenia - jak bardzo częste jest łączne występowanie w porównaniu do niezależnego). Reguły są generowane tylko dla itemsetów przekraczających minimalne progi (min\_support = 0.01, min\_confidence = 0.3), co eliminuje powiązania przypadkowe. Metoda jest efektywna w rekomendacji produktów komplementarnych (akcesoria, dodatki) ale wymaga historii współzakupów i napotyka problem zimnego startu dla nowych produktów. Wyniki są zapisywane w tabeli \texttt{method\_productassociation}.

\newpage
\medskip
\noindent\textbf{Porównanie par metod}

\medskip

\noindent\textbf{Content-Based Filtering vs Collaborative Filtering}

\medskip

Tabela \ref{tab:cbf-vs-cf} przedstawia kluczowe różnice między metodami CBF i CF. Content-Based Filtering ekstraktuje cechy bezpośrednio z atrybutów produktów (kategoria 40\%, tagi 30\%, cena 20\%, słowa kluczowe 10\%) bez potrzeby historii zakupów. Collaborative Filtering analizuje historię zakupów i buduje macierz podobieństw produktów używając algorytmu Adjusted Cosine Similarity z centrowaniem średniej użytkownika.

\begin{table}[H]
\centering
\caption{Porównanie metod Content-Based Filtering i Collaborative Filtering}
\label{tab:cbf-vs-cf}
\begin{tabular}{|p{3cm}|p{5.5cm}|p{5.5cm}|}
\hline
\textbf{Kryterium} & \textbf{CBF (Feature-Based)} & \textbf{CF (Item-Based)} \\
\hline
Źródło danych & Atrybuty produktów (cechy) & Historia zakupów (zachowania) \\
\hline
Pokrycie katalogu & Pełne - każdy produkt ma cechy & Częściowe - tylko produkty z historią \\
\hline
Problem zimnego startu & Nie występuje dla nowych produktów & Występuje dla nowych produktów \\
\hline
Typ powiązań & Oczywiste w kategorii (laptop A $\rightarrow$ laptop B) & Nieoczywiste międzykategorialne (aparat $\rightarrow$ plecak) \\
\hline
Algorytm & Weighted TF-IDF + Cosine Similarity & Adjusted Cosine + centrowanie średniej \\
\hline
Przechowywanie & \texttt{ProductSimilarity} (type='content\_based') & \texttt{ProductSimilarity} (type='collaborative') \\
\hline
Złożoność & Niższa - wektory powiązanych produktów & Wyższa - macierz wszystkich produktów \\
\hline
Czas odpowiedzi & Średni (cache MISS: kilka sekund) & Szybki (po wytrenowaniu macierzy) \\
\hline
Interpretowalność & Średnia (wagi cech) & Niska (ukryte wzorce) \\
\hline
\end{tabular}
\end{table}

\textbf{Różnice w działaniu:} CBF rekomenduje laptopy podobne do laptopów - produkty z tej samej kategorii, w podobnym przedziale cenowym, o podobnych tagach. CF wykrywa powiązania które nie są oczywiste na podstawie samych cech produktów. Jeśli użytkownicy kupujący aparaty często kupują też plecaki, CF połączy te produkty mimo różnych kategorii. CBF działa od razu bo każdy produkt ma cechy, podczas gdy CF wymaga dużo danych historycznych aby działać dobrze.

\textbf{Zastosowanie praktyczne:} W projekcie obie metody działają równolegle. CBF jest używany dla nowych produktów bez historii oraz dla gości bez konta - gwarantuje że każdy produkt może być rekomendowany (sekcja „Podobne produkty" na stronie produktu). CF jest używany dla zalogowanych użytkowników z historią zakupów - wyświetla rekomendacje oparte na wzorcach zakupowych społeczności. Dla sklepów z małą liczbą transakcji CBF jest bezpieczniejszym wyborem. Dla dojrzałych sklepów z bogatą historią CF odkrywa ciekawsze powiązania które zwiększają sprzedaż krzyżową.

\textbf{Wyniki testów:} CBF osiągnął 80-85\% pokrycie katalogu (około 400-425 produktów ma podobieństwa powyżej progu 20\%). Czas generowania macierzy wyniósł około 1 minuty dla 500 produktów. CF wymagał większej ilości danych historycznych i osiągnął 60-70\% pokrycie katalogu. Obie metody z cache HIT (<100ms) zapewniają bardzo szybką odpowiedź, ale CBF ma przewagę dla nowych produktów gdzie CF zwraca pustą listę rekomendacji.

\newpage
\noindent\textbf{Fuzzy Logic vs Sentiment Analysis}

\medskip

Tabela \ref{tab:fuzzy-vs-sentiment} zestawia metody spersonalizowaną (Fuzzy Logic) i uniwersalną (Sentiment Analysis). Fuzzy Logic buduje indywidualny \texttt{FuzzyUserProfile} z historii zakupów każdego użytkownika (parametr wrażliwości cenowej, preferowane kategorie). Sentiment Analysis agreguje tekst z pięciu źródeł używając leksykonów Opinion Lexicon i AFINN-165 i tworzy jedną ocenę dla całej społeczności.

\begin{table}[H]
\centering
\caption{Porównanie metod Fuzzy Logic i Sentiment Analysis}
\label{tab:fuzzy-vs-sentiment}
\begin{tabular}{|p{3cm}|p{5.5cm}|p{5.5cm}|}
\hline
\textbf{Kryterium} & \textbf{Fuzzy Logic} & \textbf{Sentiment Analysis} \\
\hline
Personalizacja & Wysoka - profil użytkownika & Brak - uniwersalny wskaźnik \\
\hline
Źródło danych & Historia zakupów (preferencje cenowe) & Opinie tekstowe użytkowników \\
\hline
Kryteria oceny & Cena, jakość (rating), popularność & Sentyment tekstowy + oceny gwiazdkowe \\
\hline
Algorytm & System wnioskowania Mamdaniego (6 reguł IF-THEN) & Leksykony sentymentu (Opinion Lexicon, AFINN-165) \\
\hline
Interpretowalność & Pełna (100\%) - aktywacja reguł & Średnia (wagi słów z leksykonów) \\
\hline
Czas odpowiedzi & Bardzo szybki (milisekundy) & Szybki (preprocessing opinii) \\
\hline
Problem zimnego startu & Częściowy (wymaga historii użytkownika) & Nie występuje (uniwersalny) \\
\hline
Kontekst & Profil cenowy użytkownika & Opinie społeczności \\
\hline
Typ rekomendacji & „Dopasowane do Twojego budżetu" & „Najlepiej oceniane przez innych" \\
\hline
\end{tabular}
\end{table}

\textbf{Różnice w działaniu:} Fuzzy Logic personalizuje rekomendacje według profilu cenowego użytkownika. Ten sam produkt może być polecony użytkownikowi premium jako „wysoka jakość, warta swojej ceny" i odrzucony dla użytkownika budżetowego jako „zbyt drogi". System używa sześciu reguł typu „jeśli-to" (IF-THEN), które oceniają produkty na podstawie trzech kryteriów z funkcjami przynależności trójkątnymi i trapezoidalnymi. Sentiment Analysis natomiast tworzy jeden uniwersalny ranking produktów na podstawie agregacji opinii tekstowych i ocen gwiazdkowych. Produkt z wysokim sentymentem (+0.8) jest rekomendowany wszystkim użytkownikom tak samo.

\textbf{Zastosowanie praktyczne:} Fuzzy Logic jest dostępna w dedykowanej sekcji panelu klienta z trzema zakładkami (preferowane kategorie, dopasowanie cenowe, jakość-popularność). Wymaga zalogowania i historii zakupów. Sentiment Analysis jest dostępna dla wszystkich użytkowników (także gości) i wyświetla uniwersalny ranking „Najlepiej oceniane produkty" na stronie głównej. Fuzzy Logic jest skuteczniejsza dla użytkowników o wyraźnych preferencjach cenowych (tylko tanie lub tylko premium), podczas gdy Sentiment Analysis lepiej sprawdza się jako globalny filtr jakości.

\textbf{Wyniki testów:} Fuzzy Logic osiągnęła najszybszy czas odpowiedzi (<100ms dla 100 produktów) dzięki brakowi konieczności przechowywania macierzy - wszystkie obliczenia wykonywane w locie. Wyniki dotyczą zastosowanego w pracy prostego systemu rozmytego Mamdaniego z 6 regułami IF-THEN i 3 zmiennymi wejściowymi (cena, jakość, popularność). Interpretowalność 100\% - każda rekomendacja ma pełne wyjaśnienie aktywacji konkretnych reguł. Sentiment Analysis wymagała preprocessingu opinii tekstowych (tokenizacja, usuwanie stop words) ale działała uniwersalnie bez potrzeby budowania profili użytkowników. Głównym ograniczeniem Sentiment jest wrażliwość na negację i ironię („nie polecam" może być błędnie sklasyfikowane jako pozytywne jeśli leksykon wykryje słowo „polecam").

\newpage
\noindent\textbf{Modele Probabilistyczne vs Apriori}

\medskip

Tabela \ref{tab:prob-vs-apriori} porównuje metody sekwencyjne (modele probabilistyczne) i asocjacyjne (Apriori). Modele probabilistyczne (łańcuch Markowa + Naive Bayes) przewidują przyszłe zakupy na podstawie sekwencji czasowych oraz profilu behawioralnego użytkownika. Apriori odkrywa reguły współzakupów z koszyków produktowych bez uwzględniania kolejności.

\begin{table}[H]
\centering
\caption{Porównanie modeli probabilistycznych i algorytmu Apriori}
\label{tab:prob-vs-apriori}
\begin{tabular}{|p{3cm}|p{5.5cm}|p{5.5cm}|}
\hline
\textbf{Kryterium} & \textbf{Modele Probabilistyczne} & \textbf{Apriori} \\
\hline
Źródło danych & Sekwencje zamówień w czasie & Koszyki zakupowe (itemsety) \\
\hline
Uwzględnienie czasu & Tak - kolejność zakupów & Nie - współwystępowanie \\
\hline
Algorytm & Markov Chain (60\%) + Naive Bayes (40\%) & Algoritm Apriori (support, confidence, lift) \\
\hline
Personalizacja & Wysoka - profil użytkownika & Brak - uniwersalne reguły \\
\hline
Typ predykcji & „Po laptopie następują akcesoria" & „Z laptopem kupują mysz" \\
\hline
Metryki & Prawdopodobieństwo przejścia, prawdopodobieństwo zakupu & Support, confidence, lift \\
\hline
Problem zimnego startu & Występuje (wymaga historii) & Występuje (wymaga współzakupów) \\
\hline
Czas treningu & Kilka sekund (Markov + NB) & Wysoki (dla dużych katalogów) \\
\hline
Czas predykcji & Bardzo szybki (<10ms) & Średni (wyszukiwanie reguł) \\
\hline
Zastosowanie & Przewidywanie następnej kategorii, analiza odejścia klienta & Rekomendacje produktów komplementarnych \\
\hline
\end{tabular}
\end{table}

\textbf{Różnice w działaniu:} Modele probabilistyczne analizują sekwencje zakupów w czasie i przewidują którą kategorię użytkownik prawdopodobnie kupi następną (np. „po laptopie zwykle następują akcesoria komputerowe"). Łańcuch Markowa pierwszego rzędu używa macierzy przejść między 48 kategoriami produktów trenowanej na rzeczywistych sekwencjach zamówień. Naive Bayes ocenia prawdopodobieństwo zakupu na podstawie pięciu cech użytkownika: łączna liczba zamówień, średnia wartość zamówienia, dni od ostatniego zakupu, ulubiona kategoria oraz częstotliwość zakupów. Apriori natomiast odkrywa reguły typu „klienci kupujący laptop często kupują również mysz bezprzewodową" bez uwzględniania kolejności czasowej. Analizuje koszyki zakupowe i identyfikuje itemsety pojawiające się razem częściej niż przypadkowo.

\textbf{Zastosowanie praktyczne:} Modele probabilistyczne są widoczne w sekcji „Analiza probabilistyczna" panelu klienta z wizualizacją łańcucha przejść między kategoriami oraz predykcją prawdopodobieństwa odejścia klienta. Administrator ma dostęp do paneli „Markov Chain Analysis" i „Bayesian Analysis" z prognozami sprzedaży oraz identyfikacją użytkowników zagrożonych odejściem. Apriori wyświetla sekcję „Często kupowane razem" na stronach produktów oraz w koszyku zakupowym, sugerując produkty komplementarne. Modele probabilistyczne są skuteczniejsze dla użytkowników z regularną historią zakupów (wykrywają cykle zakupowe), podczas gdy Apriori lepiej sprawdza się w rekomendacji akcesoriów i dodatków (cross-selling).

\textbf{Wyniki testów:} Modele probabilistyczne wymagały treningu na 265 zamówieniach (kilka sekund dla Markov Chain, 1-2 sekundy dla Naive Bayes). Po wytrenowaniu predykcje były wykonywane bardzo szybko (<10ms). Macierz przejść Markova zawierała 48 stanów (po jednym dla każdej kategorii) z wygładzaniem Laplace'a ($\alpha$ = 1.0) zapobiegającym zerowemu prawdopodobieństwu. Apriori generował reguły asocjacyjne z progami min\_support = 0.01 i min\_confidence = 0.3, co eliminowało powiązania przypadkowe ale wymagało wystarczającej liczby transakcji dla każdego itemsetu.

\newpage
\medskip
\noindent\textbf{Podsumowanie ewaluacji}

\medskip

Porównanie sześciu metod rekomendacyjnych wykazało, że każda z nich adresuje inny aspekt problemu rekomendacji i najlepsze wyniki uzyskuje się stosując je komplementarnie:

\textbf{Metody oparte na treści vs metody behawioralne:} Content-Based Filtering (cechy produktów) rozwiązuje problem zimnego startu dla nowych produktów i zapewnia pełne pokrycie katalogu. Collaborative Filtering (historia zakupów społeczności) odkrywa nieoczywiste powiązania międzykategorialne zwiększające sprzedaż krzyżową. CBF jest bezpieczniejszym wyborem dla nowych sklepów, CF dla dojrzałych platform z bogatą historią transakcji.

\textbf{Metody personalizowane vs uniwersalne:} Fuzzy Logic (profil cenowy użytkownika) personalizuje rekomendacje według wrażliwości cenowej i oferuje pełną interpretowalność (100\% - aktywacja 6 reguł IF-THEN). Sentiment Analysis (opinie społeczności) tworzy uniwersalny ranking jakości produktów dostępny dla wszystkich użytkowników. Fuzzy jest skuteczniejsza dla użytkowników o wyraźnych preferencjach, Sentiment jako globalny filtr jakości.

\textbf{Metody sekwencyjne vs asocjacyjne:} Modele probabilistyczne (sekwencje czasowe) przewidują przyszłe zakupy na podstawie łańcucha Markowa (60\%) i profilu behawioralnego Naive Bayes (40\%), umożliwiając identyfikację użytkowników zagrożonych odejściem. Apriori (współzakupy) odkrywa reguły asocjacyjne produktów komplementarnych efektywne w sprzedaży krzyżowej (ang. \textit{cross-selling}). Modele probabilistyczne lepsze dla regularnych użytkowników (cykle zakupowe), Apriori dla akcesoriów i dodatków.

Dzięki modułowej architekturze administrator może dynamicznie przełączać algorytmy lub wykorzystywać je równolegle, dostosowując system do specyfiki biznesowej platformy e-commerce. Praktyczne testy wykazały, że system efektywnie obsługuje 500 produktów, 20 użytkowników oraz 265 zamówień, wydajnie generując rekomendacje w czasie rzeczywistym dla wszystkich sześciu algorytmów.

\newpage

\section*{Rozdzia\l{} 8}
\addcontentsline{toc}{section}{Rozdział 8: Podsumowanie i wnioski końcowe}
\section*{Podsumowanie i wnioski końcowe}

Niniejsza praca przedstawiła proces implementacji oraz analizy kompletnego systemu e-commerce wyposażonego w mechanizmy rekomendacji produktów. Zaimplementowano trzy metody rekomendacyjne: Content-Based Filtering oparty na ważonych wektorach cech, system logiki rozmytej Mamdani oraz modele probabilistyczne wykorzystujące łańcuch Markowa i klasyfikator Naive Bayesa.

\medskip
\noindent\textbf{Ograniczenia systemu}

\medskip

W trakcie realizacji projektu zidentyfikowano następujące ograniczenia:

\textbf{Problem zimnego startu} -- algorytmy Markov Chain oraz Naive Bayes wymagają historycznych danych o interakcjach użytkowników z produktami. Dla nowych użytkowników bez historii zakupów mechanizmy te nie są w stanie generować efektywnych rekomendacji. Content-Based Filtering częściowo kompensuje to ograniczenie, ponieważ może rekomendować produkty na podstawie cech (kategoria, tagi, cena), nawet dla nowych użytkowników. Fuzzy Logic działa najlepiej dla użytkowników z umiarkowaną historią zakupów.

\textbf{Efekt ,,filter bubble'' w CBF} -- użytkownik otrzymuje rekomendacje podobnych produktów, nie odkrywając nowych kategorii. Rozwiązanie: hybrydyzacja z innymi metodami rekomendacyjnymi.

\textbf{Skalowalność dla bardzo dużych katalogów} -- dla katalogów produktów przekraczających 10\,000 pozycji mogą wystąpić wyzwania wydajnościowe. Obecne optymalizacje (przycinanie progowe, pamięć podręczna, operacje zbiorcze) są wystarczające dla katalogów do 1\,000-2\,000 produktów, ale większe wymagałyby zastosowania przybliżonego wyszukiwania najbliższych sąsiadów (algorytmy LSH, HNSW) lub partycjonowania tabel PostgreSQL.

\textbf{Brak obsługi kontekstu czasowego i sezonowości} -- system nie uwzględnia czynników sezonowych (np. zwiększone zakupy elektroniki przed świętami) ani kontekstu czasowego sesji użytkownika. Rozwiązanie: modele sekwencyjne (LSTM, GRU) lub rozszerzenie Markov Chain do wyższego rzędu.

\medskip
\noindent\textbf{Kierunki dalszego rozwoju}

\medskip

Zidentyfikowano następujące kierunki rozwoju systemu:

\textbf{Zastosowanie głębokiego uczenia maszynowego} -- obecny system wykorzystuje klasyczne algorytmy rekomendacyjne. Zastosowanie sieci neuronowych, takich jak autoencodery czy sieci rekurencyjne, mogłoby umożliwić automatyczne uczenie się ukrytych wzorców w danych bez konieczności ręcznego definiowania reguł rozmytych czy wag cech.

\textbf{Mechanizm hybrydowy} -- obecnie administrator przełącza między metodami ręcznie. System meta-learnera lub stacking ensemble mógłby automatycznie dobierać najlepszą metodę lub kombinację metod dla danego użytkownika i kontekstu.

\textbf{Rekomendacje w czasie rzeczywistym} -- obecny system wykorzystuje pamięć podręczną z okresem ważności 5-120 minut. Implementacja systemu aktualizującego rekomendacje w czasie rzeczywistym po każdej akcji użytkownika (przeglądanie produktów, dodawanie do koszyka) mogłaby zwiększyć trafność sugestii, ale wiązałaby się z istotnymi konsekwencjami dla wydajności systemu. Aktualizacje w czasie rzeczywistym wymagałyby ponownego przeliczania profilu użytkownika oraz rekomendacji przy każdej akcji, co dla algorytmu CBF oznaczałoby obliczanie podobieństw dla dziesiątek produktów, dla Fuzzy Logic -- ewaluację 6 reguł rozmytych dla setek produktów, a dla modeli probabilistycznych -- aktualizację macierzy przejść Markova i ponowny trening klasyfikatorów Naive Bayes. W scenariuszu intensywnego przeglądania (użytkownik otwiera 20-30 produktów w ciągu 5 minut) generowałoby to setek żądań obliczeniowych, potencjalnie zwiększając obciążenie serwera 10-krotnie. Rozwiązaniem kompromisowym mogłoby być wykorzystanie kolejek zadań asynchronicznych (np. Celery + Redis) do aktualizacji rekomendacji w tle z opóźnieniem 30-60 sekund, co łączyłoby korzyści personalizacji z zachowaniem akceptowalnej wydajności systemu.

\textbf{Zaawansowane metody obsługi zimnego startu} -- zastosowanie technik faktoryzacji macierzy (SVD) lub wstępnej ankiety preferencji dla nowych użytkowników mogłoby poprawić jakość rekomendacji w pierwszych sesjach.

\medskip
\noindent\textbf{Wnioski końcowe}

\medskip

Zrealizowany system stanowi kompletne rozwiązanie e-commerce z zaawansowanymi mechanizmami rekomendacji produktów. Implementacja od podstaw bez wykorzystania gotowych bibliotek rekomendacyjnych (TensorFlow, PyTorch, Surprise) umożliwiła pełne zrozumienie mechanizmów działania algorytmów oraz ich świadome dostosowanie do specyfiki handlu elektronicznego.

\textbf{Content-Based Filtering} okazał się najbardziej uniwersalną metodą, rozwiązującą problem zimnego startu dla nowych produktów. Wagi cech (kategoria 40\%, tagi 30\%, cena 20\%, słowa kluczowe 10\%) zostały dobrane empirycznie i zapewniają dobrą równowagę między różnorodnością a trafnością rekomendacji.

\textbf{Logika rozmyta} oferuje najwyższą interpretowalność spośród zaimplementowanych metod. Każda rekomendacja ma wyjaśnienie w postaci aktywacji konkretnych reguł IF-THEN, co jest istotne z perspektywy GDPR (prawo do wyjaśnienia decyzji algorytmicznych) oraz budowania zaufania użytkowników do systemu.

\textbf{Modele probabilistyczne} umożliwiają najgłębszą personalizację dla użytkowników z bogatą historią zakupów. Łańcuch Markowa przewiduje sekwencje zakupowe na poziomie kategorii produktów, Naive Bayes ocenia prawdopodobieństwo zakupu i ryzyko odejścia klienta.

Komplementarność zastosowanych metod -- Content-Based Filtering dla nowych produktów, Fuzzy Logic dla personalizacji z interpretowal nością, modele probabilistyczne dla głębokiej analizy behawioralnej -- zapewnia wszechstronne wsparcie procesu decyzyjnego użytkownika. Zastosowane techniki optymalizacyjne (cache, bulk operations, threshold pruning, indeksowanie) gwarantują akceptowalne czasy odpowiedzi systemu nawet przy większych katalogach produktów.

Praca wykazała, że implementacja systemu rekomendacyjnego od podstaw jest możliwa i celowa w kontekście edukacyjnym oraz w sytuacjach wymagających pełnej kontroli nad logiką biznesową. Zrealizowany projekt pozwolił na zdobycie praktycznej wiedzy w zakresie projektowania systemów rekomendacyjnych, optymalizacji algorytmów oraz rozwoju aplikacji full-stack (Django + React + PostgreSQL + Docker).

System stanowi kompleksowe rozwiązanie e-commerce z trzema komplementarnymi metodami rekomendacyjnymi, gotowe do wdrożenia w środowisku produkcyjnym.

\newpage

\begin{thebibliography}{99}

\bibitem{pazzani2007content}
Pazzani, M. J., \& Billsus, D. (2007). Content-Based Recommendation Systems. \textit{The Adaptive Web}, Springer, pp. 325-341.

\bibitem{zadeh1965fuzzy}
Zadeh, L. A. (1965). Fuzzy Sets. \textit{Information and Control}, 8(3), pp. 338-353.

\bibitem{mamdani1975experiment}
Mamdani, E. H., \& Assilian, S. (1975). An Experiment in Linguistic Synthesis with a Fuzzy Logic Controller. \textit{International Journal of Man-Machine Studies}, 7(1), pp. 1-13.

\bibitem{rabiner1989tutorial}
Rabiner, L. R. (1989). A Tutorial on Hidden Markov Models and Selected Applications in Speech Recognition. \textit{Proceedings of the IEEE}, 77(2), pp. 257-286.

\bibitem{murphy2012machine}
Murphy, K. P. (2012). \textit{Machine Learning: A Probabilistic Perspective}. MIT Press.

\bibitem{ricci2015recommender}
Ricci, F., Rokach, L., \& Shapira, B. (2015). \textit{Recommender Systems Handbook}. Springer.

\bibitem{gomez2016netflix}
Gomez-Uribe, C. A., \& Hunt, N. (2016). The Netflix Recommender System: Algorithms, Business Value, and Innovation. \textit{ACM Transactions on Management Information Systems}, 6(4), pp. 1-19.

\bibitem{klement2000triangular}
Klement, E. P., Mesiar, R., \& Pap, E. (2000). \textit{Triangular Norms}. Springer.

\bibitem{salton1989automatic}
Salton, G., \& Buckley, C. (1988). Term-Weighting Approaches in Automatic Text Retrieval. \textit{Information Processing \& Management}, 24(5), pp. 513-523.

\bibitem{ross2010fuzzy}
Ross, T. J. (2010). \textit{Fuzzy Logic with Engineering Applications}. Wiley, 3rd Edition.

\bibitem{mckinsey2013}
McKinsey \& Company. (2013). Big Data: The Next Frontier for Innovation, Competition, and Productivity.

\bibitem{linden2003amazon}
Linden, G., Smith, B., \& York, J. (2003). Amazon.com Recommendations: Item-to-Item Collaborative Filtering. \textit{IEEE Internet Computing}, 7(1), pp. 76-80.

\bibitem{sarwar2001item}
Sarwar, B., Karypis, G., Konstan, J., \& Riedl, J. (2001). Item-Based Collaborative Filtering Recommendation Algorithms. \textit{Proceedings of WWW}, pp. 285-295.

\bibitem{agrawal1994fast}
Agrawal, R., \& Srikant, R. (1994). Fast Algorithms for Mining Association Rules. \textit{Proceedings of VLDB}, pp. 487-499.

\bibitem{mendel2001uncertain}
Mendel, J. M. (2001). \textit{Uncertain Rule-Based Fuzzy Logic Systems: Introduction and New Directions}. Prentice Hall.

\end{thebibliography}


\newpage

% Wykaz rysunków
\section*{Wykaz rysunków i tabel}
\addcontentsline{toc}{section}{Wykaz rysunków i tabel}
\small
\listoffigures

% Spis tabel
{
\addcontentsline{toc}{section}{Spis tabel}
\small
\listoftables
}

\newpage

\section*{Streszczenie}
\addcontentsline{toc}{section}{Streszczenie}

\noindent
\textbf{Tytuł pracy w języku polskim:}\\
System rekomendacji produktów wykorzystujący filtrację opartą na treści, logikę rozmytą i modele probabilistyczne

\noindent
\textbf{Tytuł pracy w języku angielskim:}\\
Product Recommendation System Utilizing Content-Based Filtering, Fuzzy Logic, and Probabilistic Models

\vspace{0.5cm}

\noindent
\textbf{Streszczenie:}

\vspace{0.3cm}

Niniejsza praca inżynierska przedstawia projekt oraz implementację systemu rekomendacji produktów dla platformy e-commerce, łączącego trzy komplementarne metody: filtrację opartą na treści (Content-Based Filtering), logikę rozmytą (Fuzzy Logic) oraz modele probabilistyczne (Markov Chain i Naive Bayes). Celem było zaprojektowanie rozwiązania eliminującego problem przeładowania informacyjnego w sklepach internetowych poprzez dostarczanie użytkownikom spersonalizowanych rekomendacji.

Część teoretyczna obejmuje przegląd systemów rekomendacyjnych oraz analizę rozwiązań alternatywnych (Amazon Personalize, Google Recommendations AI, Apache Mahout) wraz z uzasadnieniem implementacji dedykowanego systemu. Przedstawiono fundament matematyczny wykorzystanych algorytmów: podobieństwo kosinusowe dla ważonych wektorów cech w Content-Based Filtering, funkcje przynależności trójkątne i trapezoidalne z systemem wnioskowania Mamdaniego dla logiki rozmytej oraz macierz przejść stanów i twierdzenie Bayesa dla modeli probabilistycznych.

Część projektowa obejmuje szczegółowy projekt architektury systemu w modelu trójwarstwowym (warstwa prezentacji React 18, warstwa logiki biznesowej Django 5.1.4, warstwa danych PostgreSQL 14), projekt struktury bazy danych z tabelami dla prekalkulowanych wyników algorytmów, projekt interfejsów użytkownika (widoki użytkownika końcowego i panele administracyjne) oraz projekt mechanizmów optymalizacyjnych (pamięć podręczna, indeksowanie, operacje zbiorcze).

Część implementacyjna przedstawia realizację aplikacji webowej w architekturze Django REST Framework (backend) oraz React 18 (frontend). System integruje trzy metody działające komplementarnie w różnych kontekstach: Content-Based Filtering dla rozwiązania problemu zimnego startu nowych produktów, logikę rozmytą dla personalizacji z pełną interpretowalnością reguł IF-THEN oraz modele probabilistyczne dla predykcji sekwencji zakupowych i prawdopodobieństwa odejścia klienta. Zaimplementowano kompletny interfejs z narzędziami debugowania oraz panel administracyjny umożliwiający dynamiczne przełączanie metod rekomendacyjnych. Aplikacja została skonteneryzowana w Docker Compose, co zapewnia powtarzalność środowiska deweloperskiego i produkcyjnego.

Wartością pracy jest implementacja algorytmów od podstaw, co umożliwiło głębokie zrozumienie mechanizmów oraz świadome dostosowanie do specyfiki e-commerce.

\clearpage

\newpage

% % filepath: /SmartRecommender-Project-Django-React/.docs/latex/olko/attachment.tex

\newpage
\section*{Załącznik A}
\addcontentsline{toc}{section}{Załącznik A: Szczegółowe scenariusze przypadków użycia}
\section*{Szczegółowe scenariusze przypadków użycia}

Poniżej przedstawiono szczegółowe opisy scenariuszy użycia systemu odpowiadających przypadkom użycia zaprezentowanym na diagramie w rozdziale 3.3. Scenariusze zostały opisane zgodnie z notacją: Aktor, Warunki początkowe, Przebieg scenariusza głównego, Warunki końcowe.

\subsection*{Scenariusz 1: Przeglądanie i wyszukiwanie produktów}

\noindent
\textbf{Aktor:} Gość (użytkownik niezalogowany)

\noindent
\textbf{Warunki początkowe:} Użytkownik znajduje się na stronie głównej aplikacji

\noindent
\textbf{Przebieg scenariusza głównego:}
\begin{enumerate}
\item Użytkownik otwiera stronę główną aplikacji w przeglądarce
\item Użytkownik otwiera pasek wyszukiwania z standardową opcją 
\item Użytkownik wpisuje szukaną frazę
\item System sortuje produkty według wartości \texttt{sentiment\_score} malejąco
\item Użytkownik przegląda listę produktów
\item Użytkownik klika na wybrany produkt
\item System wyświetla szczegółową stronę produktu z opisem, specyfikacjami, zdjęciami i opiniami
\end{enumerate}

\noindent
\textbf{Warunki końcowe:} Użytkownik widzi szczegółowe informacje o wybranym produkcie

\subsection*{Scenariusz 2: Dodawanie produktów do koszyka}

\noindent
\textbf{Aktor:} Gość (użytkownik niezalogowany)

\noindent
\textbf{Warunki początkowe:} Użytkownik przegląda szczegółową stronę produktu (laptop)

\noindent
\textbf{Przebieg scenariusza głównego:}
\begin{enumerate}
\item Użytkownik klika przycisk ,,Dodaj do koszyka''
\item System dodaje produkt do koszyka i wyświetla powiadomienie o sukcesie
\item Użytkownik klika ikonę koszyka w nawigacji
\item System wyświetla widok koszyka z dodanymi produktami
\item System automatycznie wywołuje algorytm Apriori dla produktów w koszyku
\item System wyświetla sekcję ,,Frequently Bought Together'' z trzema produktami rekomendowanymi przez algorytm Apriori (torba na laptopa, mysz bezprzewodowa, hub USB-C)
\item Każda rekomendacja zawiera metryki: lift, confidence oraz przycisk ,,Dodaj do koszyka''
\item Użytkownik przegląda rekomendacje
\item Użytkownik klika ,,Dodaj do koszyka'' przy torbie i myszy
\item System dodaje wybrane produkty do koszyka i aktualizuje sumę całkowitą
\end{enumerate}

\textbf{Warunki końcowe:} Koszyk zawiera laptop, torbę i mysz, wyświetlona jest zaktualizowana suma

\subsection*{Scenariusz 3: Rejestracja nowego użytkownika}

\noindent
\textbf{Aktor:} Gość (użytkownik niezalogowany)

\noindent
\textbf{Warunki początkowe:} Użytkownik znajduje się na stronie głównej

\noindent
\textbf{Przebieg scenariusza głównego:}
\begin{enumerate}
\item Użytkownik klika przycisk ,,Zarejestruj się'' w nawigacji
\item System wyświetla formularz rejestracji
\item Użytkownik wypełnia pola: imię, nazwisko, adres e-mail, hasło, potwierdzenie hasła
\item Użytkownik klika przycisk ,,Zarejestruj''
\item System waliduje dane: sprawdza czy e-mail jest unikalny, czy hasła się zgadzają, czy hasło spełnia wymagania bezpieczeństwa
\item System tworzy nowe konto użytkownika w bazie danych
\item System wyświetla komunikat o pomyślnej rejestracji
\item System automatycznie loguje użytkownika i generuje token JWT
\item System przekierowuje użytkownika do panelu klienta
\end{enumerate}

\textbf{Warunki końcowe:} Użytkownik jest zalogowany i posiada aktywne konto w systemie

\subsection*{Scenariusz 4: Logowanie do systemu}

\noindent
\textbf{Aktor:} Gość (użytkownik niezalogowany, posiadający konto)

\noindent
\textbf{Warunki początkowe:} Użytkownik znajduje się na stronie głównej

\noindent
\textbf{Przebieg scenariusza głównego:}
\begin{enumerate}
\item Użytkownik klika przycisk ,,Zaloguj się'' w nawigacji
\item System wyświetla formularz logowania
\item Użytkownik wpisuje adres e-mail i hasło
\item Użytkownik klika przycisk ,,Zaloguj''
\item System weryfikuje dane logowania w bazie danych
\item System generuje token JWT i zwraca go do klienta
\item Frontend zapisuje token w localStorage przeglądarki
\item System przekierowuje użytkownika do panelu klienta
\end{enumerate}

\noindent
\textbf{Warunki końcowe:} Użytkownik jest zalogowany, posiada ważny token JWT

\subsection*{Scenariusz 5: Śledzenie statusu zamówienia}

\noindent
\textbf{Aktor:} Klient (użytkownik zalogowany)

\noindent
\textbf{Warunki początkowe:} Użytkownik jest zalogowany, posiada co najmniej jedno zamówienie

\noindent
\textbf{Przebieg scenariusza głównego:}
\begin{enumerate}
\item Użytkownik otwiera panel klienta
\item Użytkownik klika zakładkę ,,My Orders''
\item System pobiera wszystkie zamówienia użytkownika z bazy danych
\item System wyświetla listę zamówień posortowaną od najnowszych
\item Dla każdego zamówienia wyświetlane są: numer zamówienia, status (oczekujące/w realizacji/zakończone/anulowane)
\end{enumerate}

\noindent
\textbf{Warunki końcowe:} Użytkownik widzi szczegółowe informacje o statusie wybranego zamówienia

\subsection*{Scenariusz 6: Wystawianie opinii o produkcie}

\noindent
\textbf{Aktor:} Klient (użytkownik zalogowany)

\noindent
\textbf{Warunki początkowe:} Użytkownik posiada zakończone zamówienie zawierające produkt

\noindent
\textbf{Przebieg scenariusza głównego:}
\begin{enumerate}
\item Użytkownik otwiera panel klienta, zakładkę ,,My Orders''
\item Użytkownik znajduje zakończone zamówienie
\item Użytkownik klika przycisk ,,Add Review'' przy zakupionym produkcie
\item System wyświetla formularz opinii
\item Użytkownik wybiera ocenę gwiazdkową (1-5 gwiazdek)
\item Użytkownik wpisuje treść tekstową opinii
\item Użytkownik klika ,,Wyślij opinię''
\item System zapisuje opinię w bazie danych z powiązaniem: użytkownik, produkt, zamówienie
\item System automatycznie wywołuje algorytm analizy sentymentu dla nowej opinii
\item Algorytm tokenizuje tekst, wyszukuje słowa w słowniku pozytywnym/negatywnym
\item Algorytm oblicza sentiment\_score dla opinii zgodnie ze wzorem (2)
\item System przelicza zagregowany sentyment produktu zgodnie ze wzorem (3) (wieloźródłowa agregacja z 5 źródeł)
\item System aktualizuje pole \texttt{sentiment\_score} w rekordzie produktu
\item System wyświetla komunikat o pomyślnym dodaniu opinii
\item System odświeża widok produktu - zaktualizowany sentyment jest widoczny
\end{enumerate}

\noindent
\textbf{Warunki końcowe:} Opinia została dodana, sentyment produktu zaktualizowany

\subsection*{Scenariusz 7: Zarządzanie kontem użytkownika}

\noindent
\textbf{Aktor:} Klient (użytkownik zalogowany)

\noindent
\textbf{Warunki początkowe:} Użytkownik jest zalogowany

\noindent
\textbf{Przebieg scenariusza głównego:}
\begin{enumerate}
\item Użytkownik otwiera panel klienta
\item Użytkownik klika zakładkę ,,Konto''
\item System wyświetla formularz z danymi użytkownika: imię, nazwisko, e-mail
\item Użytkownik edytuje wybrane pola (np. zmienia nazwisko)
\item Użytkownik klika ,,Zapisz zmiany''
\item System waliduje dane (sprawdza czy e-mail jest unikalny, jeśli został zmieniony)
\item System aktualizuje rekord użytkownika w bazie danych
\item System wyświetla komunikat o pomyślnym zapisaniu zmian
\end{enumerate}

\noindent
\textbf{Warunki końcowe:} Dane użytkownika zostały zaktualizowane

\subsection*{Scenariusz 8: Administrator zarządza produktami}

\noindent
\textbf{Aktor:} Administrator

\noindent
\textbf{Warunki początkowe:} Administrator jest zalogowany, posiada uprawnienia administratora

\noindent
\textbf{Przebieg scenariusza głównego - dodawanie produktu:}
\begin{enumerate}
\item Administrator otwiera panel administracyjny
\item Administrator klika sekcję ,,Products''
\item System wyświetla listę wszystkich produktów
\item Administrator klika przycisk ,,Add New Product''
\item System wyświetla formularz dodawania produktu
\item Administrator wypełnia pola: nazwa, opis, cena, kategorie, specyfikacje techniczne, przesyła zdjęcie
\item Administrator klika ,,Zapisz produkt''
\item System waliduje dane (sprawdza czy wszystkie wymagane pola są wypełnione)
\item System tworzy nowy rekord produktu w bazie danych
\item System automatycznie wywołuje algorytm analizy sentymentu dla nowego produktu
\item Algorytm oblicza sentiment\_score na podstawie nazwy i opisu (brak jeszcze opinii)
\item System zapisuje obliczony sentyment w rekordzie produktu
\item System wyświetla komunikat o pomyślnym dodaniu produktu
\item System odświeża listę produktów
\end{enumerate}

\noindent
\textbf{Warunki końcowe:} Nowy produkt został dodany do katalogu z obliczonym sentymentem

\subsection*{Scenariusz 9: Administrator zarządza zamówieniami}

\noindent
\textbf{Aktor:} Administrator

\noindent
\textbf{Warunki początkowe:} Administrator jest zalogowany

\noindent
\textbf{Przebieg scenariusza głównego:}
\begin{enumerate}
\item Administrator otwiera panel administracyjny
\item Administrator klika sekcję ,,Orders''
\item System wyświetla listę wszystkich zamówień w systemie
\item Dla każdego zamówienia wyświetlane są: numer, aktualny status
\item Administrator klika na wybrane zamówienie ze statusem ,,oczekujące''
\item Administrator klika przycisk ,,Zmień status''
\item System wyświetla listę dostępnych statusów (w realizacji, zakończone, anulowane)
\item Administrator wybiera ,,w realizacji''
\item System aktualizuje status zamówienia w bazie danych
\item System zapisuje zdarzenie zmiany statusu z timestampem
\item System wyświetla komunikat o pomyślnej zmianie statusu
\item Użytkownik właściciel zamówienia widzi zaktualizowany status w swoim panelu
\end{enumerate}

\noindent
\textbf{Warunki końcowe:} Status zamówienia został zmieniony, użytkownik ma dostęp do aktualnej informacji

\subsection*{Scenariusz 10: Administrator zarządza użytkownikami}

\noindent
\textbf{Aktor:} Administrator

\noindent
\textbf{Warunki początkowe:} Administrator jest zalogowany

\noindent
\textbf{Przebieg scenariusza głównego:}
\begin{enumerate}
\item Administrator otwiera panel administracyjny
\item Administrator klika sekcję ,,Users''
\item System wyświetla listę wszystkich użytkowników
\item Dla każdego użytkownika wyświetlane są: ID, imię, nazwisko, e-mail, rola, data rejestracji
\item Administrator może filtrować użytkowników według roli (klient/administrator)
\item Administrator klika na wybranego użytkownika
\item System wyświetla szczegóły użytkownika: profil, historia zamówień, statystyki
\item Administrator może zmienić rolę użytkownika (nadać uprawnienia administratora) lub usunąć konto
\end{enumerate}

\noindent
\textbf{Warunki końcowe:} Administrator ma pełny wgląd w dane użytkowników i może zarządzać uprawnieniami

\subsection*{Scenariusz 11: Administrator przegląda statystyki i dashboard}

\noindent
\textbf{Aktor:} Administrator

\noindent
\textbf{Warunki początkowe:} Administrator jest zalogowany

\noindent
\textbf{Przebieg scenariusza głównego:}
\begin{enumerate}
\item Administrator otwiera panel administracyjny
\item System automatycznie wyświetla dashboard ze statystykami
\item Dashboard zawiera następujące kluczowe metryki:
  \begin{itemize}
  \item Całkowita liczba produktów w katalogu
  \item Całkowita liczba zarejestrowanych użytkowników
  \item Całkowita liczba zamówień (wszystkie statusy)
  \item Całkowity przychód (suma wartości zamówień zakończonych)
  \end{itemize}
\item System wyświetla wykresy:
  \begin{itemize}
  \item Wykres słupkowy: miesięczny obrót (ostatnie 12 miesięcy)
  \item Wykres kołowy: rozkład zamówień według kategorii produktów
  \item Wykres liniowy: liczba nowych użytkowników (ostatnie 30 dni)
  \end{itemize}
\item Administrator klika sekcję ,,Statistics'' dla bardziej szczegółowych analiz
\item System wyświetla widok do zmiany rekomentacji, odświeżanie reguł asocjacyjnych
\end{enumerate}

\noindent
\textbf{Warunki końcowe:} Administrator ma pełny wgląd w kluczowe wskaźniki biznesowe systemu

\subsection*{Scenariusz 12: Administrator debuguje algorytmy rekomendacji}

\noindent
\textbf{Aktor:} Administrator

\noindent
\textbf{Warunki początkowe:} Administrator jest zalogowany, algorytmy rekomendacji zostały przeliczone

\noindent
\textbf{Przebieg scenariusza głównego - Collaborative Filtering:}
\begin{enumerate}
\item Administrator otwiera panel administracyjny
\item Administrator klika sekcję ,,Debug''
\item Administrator wybiera zakładkę ,,Collaborative Filtering''
\item System wyświetla panel debugowania Collaborative Filtering zawierający:
  \begin{itemize}
  \item Wzór matematyczny Adjusted Cosine Similarity
  \item Statystyki: liczba par produktów, średnie podobieństwo, czas ostatniego przeliczenia
  \item Tabelę z przykładowymi podobieństwami (product\_a, product\_b, similarity\_score)
  \end{itemize}
\item Administrator może wpisać ID produktu i zobaczyć top 10 najbardziej podobnych produktów
\item System pobiera dane z tabeli ProductSimilarity i wyświetla wyniki
\end{enumerate}

\noindent
\textbf{Warunki końcowe:} Administrator zweryfikował poprawność działania algorytmu Collaborative Filtering

\subsection*{Scenariusz 13: Przeglądanie rekomendacji na stronie głównej}

\noindent
\textbf{Aktor:} Klient (użytkownik zalogowany)

\noindent
\textbf{Warunki początkowe:} Użytkownik jest zalogowany, posiada historię zakupów

\noindent
\textbf{Przebieg scenariusza głównego:}
\begin{enumerate}
\item Użytkownik loguje się do systemu
\item System przekierowuje użytkownika do strony głównej
\item System wywołuje algorytm Collaborative Filtering dla zalogowanego użytkownika
\item Algorytm analizuje historię zakupów użytkownika
\item Algorytm pobiera produkty podobne do wcześniej zakupionych (na podstawie macierzy podobieństw)
\item System wyświetla sekcję ,,Rekomendowane dla Ciebie'' z 6 produktami
\item Użytkownik przegląda rekomendacje
\item Użytkownik klika na wybrany produkt
\item System wyświetla szczegółową stronę produktu
\end{enumerate}

\noindent
\textbf{Warunki końcowe:} Użytkownik widzi spersonalizowane rekomendacje oparte na swojej historii zakupów

\subsection*{Scenariusz 14: Przeglądanie podobnych produktów (Content-Based Filtering)}

\noindent
\textbf{Aktor:} Klient lub Gość

\noindent
\textbf{Warunki początkowe:} Użytkownik przegląda szczegółową stronę produktu (np. laptop Dell XPS 15)

\noindent
\textbf{Przebieg scenariusza głównego:}
\begin{enumerate}
\item Użytkownik otwiera stronę szczegółową produktu
\item System automatycznie wywołuje endpoint \texttt{GET /api/\allowbreak recommendations/\allowbreak content-based/\allowbreak ?product\_id=\{id\}}
\item Backend wywołuje metodę \texttt{ContentBasedAPI} z parametrem \texttt{product\_id}
\item System wykonuje zapytanie do tabeli \texttt{ProductSimilarity} filtrując po \texttt{product1\_id = \{id\}} oraz \texttt{type=\allowbreak 'content\_based'}
\item System sortuje wyniki malejąco według \texttt{similarity\_score}
\item System pobiera Top 10 najbardziej podobnych produktów z bazy danych
\item Backend zwraca JSON z listą produktów zawierającą: \texttt{product\_id}, \texttt{name}, \texttt{price}, \\\texttt{similarity\_score}
\item Frontend wyświetla sekcję „Podobne produkty" z kafelkami produktów
\item Każdy kafelek zawiera zdjęcie, nazwę, cenę oraz wskaźnik podobieństwa (np. „85\% podobny")
\item Użytkownik przegląda rekomendacje
\item Użytkownik klika na wybrany podobny produkt
\item System przekierowuje do strony szczegółowej wybranego produktu
\end{enumerate}

\noindent
\textbf{Warunki końcowe:} Użytkownik widzi produkty podobne do oglądanego, obliczone na podstawie cech (kategorie 40\%, tagi 30\%, cena 20\%, słowa kluczowe 10\%)

\subsection*{Scenariusz 15: Generowanie macierzy podobieństw Content-Based Filtering}

\noindent
\textbf{Aktor:} Administrator

\noindent
\textbf{Warunki początkowe:} Administrator jest zalogowany, katalog zawiera produkty

\noindent
\textbf{Przebieg scenariusza głównego:}
\begin{enumerate}
\item Administrator otwiera panel administracyjny
\item Administrator klika sekcję „Admin Tools"
\item Administrator wybiera zakładkę „Content-Based Filtering"
\item System wyświetla panel zarządzania CBF z przyciskiem „Generate Similarity Matrix"
\item Administrator klika „Generate Similarity Matrix"
\item System wywołuje endpoint \texttt{POST /api/admin/\allowbreak content-based/\allowbreak generate}
\item Backend uruchamia algorytm \texttt{ContentBased\allowbreak Recommendation\allowbreak Engine.\allowbreak generate\_\allowbreak \\similarities()}
\item System pobiera wszystkie produkty z bazy danych
\item Dla każdego produktu system ekstraktuje cechy:
  \begin{itemize}
  \item Kategorie (waga 40\%) - konwersja do wektora one-hot
  \item Tagi (waga 30\%) - ekstrakcja unikalnych tagów
  \item Przedział cenowy (waga 20\%) - normalizacja do zakresu [0,1]
  \item Słowa kluczowe z opisu (waga 10\%) - tokenizacja i TF-IDF
  \end{itemize}
\item System oblicza podobieństwo cosinusowe między każdą parą produktów zgodnie ze wzorem Weighted Cosine Similarity
\item System filtruje pary z podobieństwem $>$ 20\% (próg minimalny)
\item System usuwa stare rekordy z tabeli \texttt{ProductSimilarity} dla \texttt{type='content\_based'}
\item System zapisuje nowe podobieństwa do tabeli \texttt{ProductSimilarity}
\item System wyświetla komunikat o sukcesie: „Wygenerowano 1247 podobieństw dla 500 produktów w czasie 58 sekund"
\item Administrator widzi zaktualizowane statystyki: liczba par produktów, średnie podobieństwo, timestamp ostatniego przeliczenia
\end{enumerate}

\noindent
\textbf{Warunki końcowe:} Macierz podobieństw została wygenerowana i zapisana w bazie danych, rekomendacje CBF są aktualne

\subsection*{Scenariusz 16: Debugowanie algorytmu Content-Based Filtering}

\noindent
\textbf{Aktor:} Administrator

\noindent
\textbf{Warunki początkowe:} Administrator jest zalogowany, macierz podobieństw została wygenerowana

\noindent
\textbf{Przebieg scenariusza głównego:}
\begin{enumerate}
\item Administrator otwiera panel administracyjny
\item Administrator klika sekcję „Debug"
\item Administrator wybiera zakładkę „Content-Based Debug"
\item System wyświetla panel debugowania CBF zawierający:
  \begin{itemize}
  \item Wzór matematyczny Weighted Cosine Similarity z opisem wag cech
  \item Statystyki globalne: liczba produktów w katalogu, liczba wygenerowanych par, średnie podobieństwo, czas ostatniego przeliczenia
  \item Histogram rozkładu podobieństw (przedziały: 20-40\%, 40-60\%, 60-80\%, 80-100\%)
  \end{itemize}
\item Administrator wprowadza ID produktu w pole „Product ID for analysis"
\item Administrator klika „Analyze Product"
\item System wywołuje endpoint \texttt{GET /api/admin/\allowbreak content-based/\allowbreak debug/\allowbreak ?product\_id=\{id\}}
\item System pobiera wybrany produkt oraz jego Top 10 najbardziej podobnych produktów
\item System wyświetla szczegółową analizę dla wybranego produktu:
  \begin{itemize}
  \item Podstawowe dane: nazwa, kategorie, tagi, cena
  \item Ekstrahowane cechy: wektor kategorii, lista tagów, znormalizowana cena, słowa kluczowe
  \item Tabelę Top 10 podobnych produktów z kolumnami: ID, nazwa, similarity\_score, wspólne kategorie, wspólne tagi
  \end{itemize}
\item Administrator analizuje wyniki i weryfikuje poprawność podobieństw
\item Administrator może wyeksportować dane do CSV klikając „Export to CSV"
\end{enumerate}

\noindent
\textbf{Warunki końcowe:} Administrator zweryfikował poprawność działania algorytmu CBF dla wybranego produktu

\subsection*{Scenariusz 17: Przeglądanie rekomendacji Fuzzy Logic w panelu klienta}

\noindent
\textbf{Aktor:} Klient (użytkownik zalogowany)

\noindent
\textbf{Warunki początkowe:} Użytkownik jest zalogowany, posiada historię zakupów

\noindent
\textbf{Przebieg scenariusza głównego:}
\begin{enumerate}
\item Użytkownik loguje się do systemu
\item Użytkownik otwiera panel klienta
\item Użytkownik klika zakładkę „Fuzzy Logic Recommendations"
\item System wywołuje endpoint \texttt{GET /api/\allowbreak fuzzy-logic-\allowbreak recommendations/\allowbreak ?limit=10}
\item Backend wywołuje \texttt{FuzzyLogicAPI} z parametrem \texttt{user\_id} (z tokenu JWT)
\item System inicjalizuje silnik rozmyty \texttt{FuzzyEngine.\allowbreak initialize\_\allowbreak fuzzy\_system(\allowbreak user\_id)}
\item System buduje profil użytkownika \texttt{FuzzyUserProfile}:
  \begin{itemize}
  \item Oblicza parametr \texttt{price\_sensitivity} na podstawie średniej ceny zakupionych produktów
  \item Ekstraktuje preferowane kategorie z historii zakupów (top 3 kategorie)
  \end{itemize}
\item System pobiera wszystkie produkty z bazy danych z metrykami: cena, średni rating, liczba zamówień
\item Dla każdego produktu system wykonuje wnioskowanie rozmyte:
  \begin{itemize}
  \item Fuzzyfikacja - oblicza przynależność ceny, jakości i popularności do zbiorów rozmytych (cheap/medium/expensive, low/medium/high)
  \item Aplikacja 6 reguł IF-THEN typu Mamdani (R1: preferred category match, R2: quality-price balance, R3: price sensitive match, R4: premium match, R5: popular high quality, R6: budget friendly)
  \item Agregacja wyników reguł z wagami (R1: 0.9, R2: 0.85, R3: 0.6, R4: 0.75, R5: 0.8, R6: 0.7)
  \item Defuzzyfikacja - oblicza końcowy \texttt{fuzzy\_score} metodą średniej ważonej
  \end{itemize}
\item System sortuje produkty malejąco według \texttt{fuzzy\_score}
\item System zwraca JSON z Top 10 rekomendacji zawierający: \texttt{product\_id}, \texttt{name}, \texttt{price}, \\\texttt{fuzzy\_score}, \texttt{activated\_rules}
\item Frontend wyświetla trzy zakładki:
  \begin{itemize}
  \item „Fuzzy Recommendations" - lista Top 10 produktów z fuzzy\_score i aktywowanymi regułami
  \item „User Profile" - wyświetla \texttt{price\_sensitivity} (np. 60\% - moderate) oraz preferowane kategorie
  \item „Fuzzy Rules" - opis 6 reguł IF-THEN z wyjaśnieniem logiki
  \end{itemize}
\item Użytkownik przegląda rekomendacje dopasowane do jego profilu cenowego
\item Użytkownik klika przycisk „View Rule Activations" przy wybranym produkcie
\item System wyświetla szczegóły aktywacji reguł: które reguły zostały aktywowane, z jaką siłą ($\alpha_i$), jaki był ich wkład w końcowy wynik
\end{enumerate}

\noindent
\textbf{Warunki końcowe:} Użytkownik widzi spersonalizowane rekomendacje dopasowane do jego wrażliwości cenowej oraz preferowanych kategorii z pełną przejrzystością procesu decyzyjnego

\subsection*{Scenariusz 18: Wyszukiwanie tolerujące błędy (Fuzzy Search)}

\noindent
\textbf{Aktor:} Klient lub Gość

\noindent
\textbf{Warunki początkowe:} Użytkownik znajduje się na stronie głównej

\noindent
\textbf{Przebieg scenariusza głównego:}
\begin{enumerate}
\item Użytkownik otwiera pasek wyszukiwania
\item Użytkownik przełącza opcję wyszukiwania na „Fuzzy Search"
\item System wyświetla suwak „Fuzzy Threshold" (domyślnie 0.5 = 50\%)
\item Użytkownik wpisuje frazę z literówką: „loptap" (zamiast „laptop")
\item Użytkownik ustawia próg podobieństwa na 0.3 (30\%)
\item Użytkownik klika przycisk „Search"
\item System wywołuje endpoint \texttt{GET /api/search/\allowbreak fuzzy/\allowbreak ?query=\allowbreak loptap\&\allowbreak threshold=0.3}
\item Backend iteruje po wszystkich produktach w katalogu
\item Dla każdego produktu system oblicza odległość Levenshteina między zapytaniem a nazwą produktu:
  \begin{itemize}
  \item Algorytm rekurencyjny oblicza minimalną liczbę operacji edycji (wstawienie, usunięcie, zamiana znaku)
  \item System normalizuje odległość do zakresu [0,1] dzieląc przez maksymalną długość
  \item Oblicza \texttt{fuzzy\_match\_score} = 1 - normalized\_distance
  \end{itemize}
\item System filtruje produkty z \texttt{fuzzy\_match\_score} $\geq$ 0.3 (próg użytkownika)
\item System sortuje wyniki malejąco według \texttt{fuzzy\_match\_score}
\item Backend zwraca JSON z produktami: \texttt{product\_id}, \texttt{name}, \texttt{price}, \texttt{fuzzy\_match\_score}
\item Frontend wyświetla wyniki wyszukiwania:
  \begin{itemize}
  \item Laptop Dell XPS 15 - 86\% match (zawiera „Laptop")
  \item Laptop HP Pavilion - 86\% match (zawiera „Laptop")
  \item Desktop Hub USB-C - 45\% match (częściowe dopasowanie)
  \end{itemize}
\item Użytkownik widzi produkty mimo błędu w zapytaniu
\item Użytkownik klika na wybrany produkt
\end{enumerate}

\noindent
\textbf{Warunki końcowe:} Użytkownik znalazł poszukiwane produkty mimo literówki dzięki algorytmowi Levenshteina

\subsection*{Scenariusz 19: Debugowanie algorytmu Fuzzy Logic}

\noindent
\textbf{Aktor:} Administrator

\noindent
\textbf{Warunki początkowe:} Administrator jest zalogowany, system posiada użytkowników z historią zakupów

\noindent
\textbf{Przebieg scenariusza głównego:}
\begin{enumerate}
\item Administrator otwiera panel administracyjny
\item Administrator klika sekcję „Debug"
\item Administrator wybiera zakładkę „Fuzzy Logic Debug"
\item System wyświetla widok ogólny panelu debugowania zawierający:
  \begin{itemize}
  \item Szczegóły algorytmu: metoda Mamdani Fuzzy Inference, liczba reguł (6), T-norma (min), T-conorma (max)
  \item Definicje funkcji przynależności dla price (cheap: 0-500 PLN, medium: 300-1500 PLN, expensive: 1000+ PLN)
  \item Definicje funkcji przynależności dla quality (low: 0-2.5, medium: 2.0-4.0, high: 3.5-5.0)
  \item Definicje funkcji przynależności dla popularity (low: 0-10 zamówień, medium: 5-50, high: 30+ zamówień)
  \item Listę profili użytkowników z parametrem price\_sensitivity
  \end{itemize}
\item Administrator wprowadza ID produktu w pole „Product ID for detailed analysis"
\item Administrator klika „Analyze Product"
\item System wywołuje endpoint \texttt{GET /api/admin/\allowbreak fuzzy-debug/\allowbreak ?product\_id=\{id\}}
\item System pobiera wybrany produkt oraz wykonuje szczegółową ewaluację rozmytą
\item System wyświetla widok szczegółowy dla produktu:
  \begin{itemize}
  \item Dane produktu: nazwa, cena, średni rating, liczba zamówień
  \item Wartości fuzzyfikacji:
    \begin{itemize}
    \item Price: cheap=0.3, medium=0.7, expensive=0.0
    \item Quality: low=0.0, medium=0.2, high=0.8
    \item Popularity: low=0.0, medium=0.6, high=0.4
    \end{itemize}
  \item Aktywacja wszystkich 6 reguł z siłą aktywacji $\alpha_i$:
    \begin{itemize}
    \item R1 (Preferred Category Match): $\alpha_1$ = 0.85, waga = 0.9, wkład = 0.765
    \item R2 (Quality-Price Balance): $\alpha_2$ = 0.7, waga = 0.85, wkład = 0.595
    \item R3 (Price Sensitive Match): $\alpha_3$ = 0.3, waga = 0.6, wkład = 0.18
    \item R4 (Premium Match): $\alpha_4$ = 0.0, waga = 0.75, wkład = 0.0
    \item R5 (Popular High Quality): $\alpha_5$ = 0.6, waga = 0.8, wkład = 0.48
    \item R6 (Budget Friendly): $\alpha_6$ = 0.2, waga = 0.7, wkład = 0.14
    \end{itemize}
  \item Agregacja: suma wkładów = 2.16, fuzzy\_score = 2.16 / 4.6 = 0.47 (47\%)
  \end{itemize}
\item Administrator analizuje które reguły dominują dla danego produktu
\item Administrator może dostroić wagi reguł klikając „Edit Rule Weights"
\item System zapisuje nowe wagi w konfiguracji
\end{enumerate}

\noindent
\textbf{Warunki końcowe:} Administrator zweryfikował proces wnioskowania rozmytego i zoptymalizował wagi reguł

\subsection*{Scenariusz 20: Przeglądanie rekomendacji probabilistycznych w panelu klienta}

\noindent
\textbf{Aktor:} Klient (użytkownik zalogowany)

\noindent
\textbf{Warunki początkowe:} Użytkownik jest zalogowany, posiada historię zakupów, modele probabilistyczne zostały wytrenowane

\noindent
\textbf{Przebieg scenariusza głównego:}
\begin{enumerate}
\item Użytkownik loguje się do systemu
\item Użytkownik otwiera panel klienta
\item Użytkownik klika zakładkę „Probabilistic Recommendations"
\item System wywołuje endpoint \texttt{GET /api/\allowbreak probabilistic-\allowbreak recommendations/\allowbreak \\?user\_id=\{id\}}
\item Backend wywołuje \texttt{ProbabilisticAPI} z parametrem \texttt{user\_id}
\item System pobiera historię zamówień użytkownika z bazy danych wywołując \texttt{get\_user\_\allowbreak \\orders(\allowbreak user\_id)}
\item System ekstraktuje sekwencję kategorii produktów z zamówień uporządkowanych chronologicznie
\item System wywołuje \texttt{MarkovChain.\allowbreak predict\_next\_\allowbreak products(\allowbreak order\_\allowbreak sequence)}:
  \begin{itemize}
  \item Identyfikuje ostatnią zakupioną kategorię (np. „Laptopy")
  \item Odczytuje macierz przejść dla kategorii „Laptopy"
  \item Pobiera Top 3 kategorie z najwyższym prawdopodobieństwem przejścia (np. Akcesoria komputerowe: 0.45, Peryferia: 0.30, Torby: 0.15)
  \item Dla każdej kategorii pobiera Top produkty
  \end{itemize}
\item System wywołuje \texttt{NaiveBayes.\allowbreak predict\_\allowbreak purchase\_\allowbreak probability(\allowbreak user\_profile)}:
  \begin{itemize}
  \item Ekstraktuje 5 cech użytkownika: liczba zamówień, średnia wartość, dni od ostatniego zakupu, ulubiona kategoria, częstotliwość
  \item Stosuje twierdzenie Bayesa dla klasyfikacji binarnej (kupi/nie kupi)
  \item Oblicza prawdopodobieństwo zakupu w najbliższym okresie
  \end{itemize}
\item System agreguje wyniki: 60\% waga dla Markov, 40\% waga dla Naive Bayes
\item Backend zwraca JSON z Top 10 rekomendacji zawierający: \texttt{product\_id}, \texttt{name}, \texttt{price}, \\\texttt{markov\_probability}, \texttt{bayes\_probability}, \texttt{combined\_score}
\item Frontend wyświetla trzy zakładki:
  \begin{itemize}
  \item „Probabilistic Recommendations" - lista Top 10 produktów z prawdopodobieństwami
  \item „Markov Chain Visualization" - graf przejść między kategoriami z prawdopodobieństwami
  \item „Purchase Probability Analysis" - wykres prawdopodobieństwa zakupu oraz analiza cech użytkownika
  \end{itemize}
\item Użytkownik przegląda rekomendacje oparte na sekwencjach zakupowych
\item Użytkownik klika zakładkę „Markov Chain Visualization"
\item System wyświetla interaktywny graf pokazujący możliwe przejścia z obecnej kategorii do kolejnych
\item Użytkownik klika zakładkę „Purchase Probability Analysis"
\item System wyświetla:
  \begin{itemize}
  \item Prawdopodobieństwo zakupu: 73\% (high probability)
  \item Breakdown cech: liczba zamówień: 12 (pozytywny wpływ +0.15), średnia wartość: 450 PLN (pozytywny +0.10), dni od ostatniego: 15 (pozytywny +0.20)
  \item Rekomendacja: „Użytkownik jest aktywny, wysokie prawdopodobieństwo konwersji"
  \end{itemize}
\end{enumerate}

\noindent
\textbf{Warunki końcowe:} Użytkownik widzi rekomendacje przewidujące jego przyszłe zakupy oparte na łańcuchach Markowa i klasyfikatorze Bayesa

\subsection*{Scenariusz 21: Trenowanie modeli probabilistycznych przez administratora}

\noindent
\textbf{Aktor:} Administrator

\noindent
\textbf{Warunki początkowe:} Administrator jest zalogowany, system posiada historię zamówień

\noindent
\textbf{Przebieg scenariusza głównego:}
\begin{enumerate}
\item Administrator otwiera panel administracyjny
\item Administrator klika sekcję „Admin Tools"
\item Administrator wybiera zakładkę „Probabilistic Models"
\item System wyświetla panel zarządzania modelami probabilistycznymi
\item Administrator widzi status modeli:
  \begin{itemize}
  \item Markov Chain: ostatni trening 2024-01-15, liczba stanów: 48 kategorii, wygładzanie Laplace'a: $\alpha$ = 1.0
  \item Naive Bayes: ostatni trening 2024-01-15, liczba próbek treningowych: 200 zamówień, accuracy: 78\%
  \end{itemize}
\item Administrator klika przycisk „Train Markov Chain Model"
\item System wywołuje endpoint \texttt{POST /api/admin/\allowbreak probabilistic/\allowbreak train-markov}
\item Backend uruchamia \texttt{MarkovChainEngine.\allowbreak train()}:
  \begin{itemize}
  \item Pobiera wszystkie zamówienia z bazy danych uporządkowane chronologicznie per użytkownik
  \item Ekstraktuje sekwencje kategorii dla każdego użytkownika
  \item Buduje macierz przejść 48x48 (48 kategorii produktowych)
  \item Oblicza prawdopodobieństwa przejść: $P(s_j | s_i) = \frac{count(s_i \rightarrow s_j) + \alpha}{\sum_k count(s_i \rightarrow s_k) + \alpha \cdot N}$
  \item Stosuje wygładzanie Laplace'a ($\alpha$ = 1.0) zapobiegające zerowemu prawdopodobieństwu
  \item Zapisuje macierz w cache Django z timeout 24h
  \end{itemize}
\item System wyświetla komunikat: „Markov Chain trained successfully. 48 states, 2304 transitions, training time: 3.2s"
\item Administrator klika przycisk „Train Naive Bayes Model"
\item System wywołuje endpoint \texttt{POST /api/admin/\allowbreak probabilistic/\allowbreak train-bayes}
\item Backend uruchamia \texttt{NaiveBayesEngine.\allowbreak train()}:
  \begin{itemize}
  \item Pobiera wszystkich użytkowników z zamówieniami
  \item Dla każdego użytkownika ekstraktuje 5 cech: liczba zamówień, średnia wartość, dni od ostatniego, ulubiona kategoria, częstotliwość
  \item Tworzy etykiety binarne (aktywny/nieaktywny) na podstawie heurystyki (zakup w ostatnich 30 dniach)
  \item Oblicza prawdopodobieństwa warunkowe dla każdej cechy: $P(feature | class)$
  \item Oblicza prawdopodobieństwa a priori: $P(active)$, $P(inactive)$
  \item Waliduje model na zbiorze testowym (80\%/20\% split)
  \item Zapisuje model w tabeli cache
  \end{itemize}
\item System wyświetla komunikat: „Naive Bayes trained successfully. Training samples: 200, Test accuracy: 78\%, training time: 1.8s"
\item Administrator widzi zaktualizowane timestampy i metryki
\end{enumerate}

\noindent
\textbf{Warunki końcowe:} Administrator zweryfikował poprawność działania modeli probabilistycznych (Markov Chain i Naive Bayes)


%\begin{figure}[H]
    %\centering
    %\includegraphics[width=\textwidth]{Oświadczenie.pdf}
%\end{figure}

\end{document}
